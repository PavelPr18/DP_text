\chapter*{Úvod}
\phantomsection
\addcontentsline{toc}{chapter}{Úvod}

Tématem této semestrální práce je návrh rozšíření \acs{JTAG} front-endu o podporu dvou a čtyřvodičového protokolu a návrh rozšíření pro přístup na systémovou sběrnici \acs{RISC-V} procesoru.

Cílem semestrální práce je navrhnout modul podporující komunikaci pomocí dvouvodičové varianty \acs{JTAG} protokolu s možností přepínat mezi touto a čtyřvodičovou variantou. Dalším cílem je návrh časově optimalizovaného protokolu pro přístup na systémovou sběrnici \acs{RISC-V} procesoru využívající \acs{JTAG} rozhraní.

V úvodní části práce je obecně přestaveno testovací rozhraní \acs{JTAG} a základní princip fungování řídicího stavového automatu dle příslušného standardu IEEE 1149.1. Dále je uveden testovací systém pro procesory \acs{RISC-V}, který je použitý pro přístup na systémovou sběrnici procesoru. \hl{Princip fungování systémové sběrnice použitého procesorového jádra je uveden na konci této kapitoly.}

Druhá kapitola se věnuje principu komunikace dvouvodičovou variantou \acs{JTAG} protokolu, způsobu aktivace dvouvodičového režimu a také možnosti přepnutí na původní čtyřvodičovou variantu. Dále je v této kapitole popsán způsob návrhu a implementace modulu zajišťujícího uvedenou funkcionalitu.

V poslední části práce je popsán navržený protokol pro přístup na systémovou sběrnici procesoru. Tento protokol bude využitý jako vedlejší možnost přístupu na systémovou sběrnici, která je časově efektivnější oproti stávajícímu způsobu využívajícím systém pro testování \acs{RISC-V} procesorů. Uveden je také způsob zkrácení přenášené adresy registru, ke kterému je přistupováno, a možnosti volby délky přenášených dat. Protokol je navržený ve dvou variantách, které jsou popsány detailněji v jednotlivých částech této kapitoly. Následně jsou také zhodnoceny jejich vlastnosti.

