%\chapter{Návrh a implementace rozšiřujícího modulu pro přístup na systémovou sběrnici pomocí JTAG protokolu}	\label{jtag_ap}
\chapter{Popis modulu pro přístup na systémovou sběrnici pomocí \acs{JTAG} protokolu}	\label{jtag_ap}
Pro přístup na systémovou sběrnici procesoru jsou navrženy dvě varianty komunikačního protokolu přenášeného pomocí JTAG rozhraní. Obě varianty protokolu jsou popsány pro čtyřvodičovou variantu \acs{JTAG} komunikace mezi debuggerem (aplikační prostředí pro testování obvodů) a \acs{JTAG} systémem (navržený systém pro testování). Navržený protokol bude sloužit jako vedlejší možnost přístupu na systémovou sběrnici, která je časově optimálnější.

\section{Návrh optimalizovaného protokolu pro přístup na systémovou sběrnici}	\label{sec:protokoly}
%\section{Návrh časově optimalizovaného protokolu pro přístup na systémovou sběrnici}	\label{sec:protokoly}
Pro přístup na systémovou sběrnici byly navrženy dvě varianty protokolu využívajícího JTAG rozhraní. Obě varianty se od sebe liší především ve způsobu signalizace nepřístupnosti systémové sběrnice z důvodu probíhající operace na sběrnici. První variantou je využití obdobného principu jak u stávajícího řešení. Tedy pokud se operace na systémovou sběrnici nestihne provést do okamžiku dalšího požadavku, je skrze JTAG komunikaci odeslán zpět status \textit{busy}. Druhou variantou je využití výstupního \texttt{\acs{TDO}} portu k signalizaci dokončení operace na systémové sběrnici dynamicky.

\subsection{Zkrácení adresy a přenášených dat}
V případě obou variant protokolu je adresa zkrácena na 21 bitů namísto původních 32 bitů. Důvodem pro zkrácení adresy je velikost adresního prostoru celého systému, ve kterém není z adresy využito 11 bitů. Z tohoto důvodu je výhodné nevyužívané bity adresy neodesílat a adresu korektně rozšířit na původních 32 bitů až po jejím příjmu. Tato optimalizace ušetří 11 taktů testovacího hodinového signálu \texttt{\acs{TCK}} pro každý přístup na sběrnici. V případě většího adresního prostoru systému by se pouze odesílaná adresa rozšířila.

Délka přenášených dat může být volena jako \textit{Word} (32-bitů), \textit{Half-Word} (16-bitů) a \textit{Byte} (8-bitů). Obvyklým způsobem je přístup na sběrnici v celé šířce 32 bitů. Pro specifické požadavky je možné využít také kratších variant. Příkladem může být potřeba přepsání logické hodnoty určitého bitu registru, či určité skupiny bitů. Délka přenesených dat je dekódována podle adresy na signál \texttt{be} systémové sběrnice.

\subsection{Varianta protokolu Busy-wait} \label{subsec:busy-wait}
Varianta protokolu \textbf{Busy-wait} je založena na principu signalizace probíhající operace na sběrnici před odesíláním požadovaných dat. Tato informace je předávána debuggeru jako 3-bitová hodnota \textit{status}, prostřednictvím signálu \texttt{\acs{TDO}}. \textit{Status} může nabývat hodnot dle tabulky \ref{tab:status_vals}, kde je vidět, že je využitý pouze jeden bit ze tří. Důvodem je to, že pole \textit{status} je vysíláno současně s hodnotou \textit{op}, která je vysílána debuggerem, a ta již 3-bitová být musí. Tato varianta protokolu využívá toho, že debugger prochází řídicím stavovým automatem (\acs{TAP}) po spuštění požadavku na systémové sběrnici vždy přes stav \textit{Run-Test/Idle}, kde čeká po několik taktů hodinového signálu \texttt{\acs{TCK}}. Tato prodleva dává možnost systémové sběrnici požadavek dokončit. V případě nedokončeného požadavku je při dalším požadavku signalizováno pomocí hodnoty \textit{status} = 1 (\textit{busy}), že nebude akceptován. Oproti přístupu na systémovou sběrnici pomocí stávajícího testovacího systému není třeba tento stav nijak resetovat.

\begin{table}[!h]
  \caption{Tabulka možných návratových \textit{status} hodnot navrženého protokolu}
  \begin{center}
  	\small
	  \begin{tabular}{!{\vrule width 1.2pt}M{1.5cm}|M{1.8cm}|M{9cm}!{\vrule width 1.2pt}}
	    \noalign{\hrule height 1.2pt}
	    Status & Význam & Popis\\
	    \noalign{\hrule height 1.2pt}
	    0 & no\_error & Systémová sběrnice je připravena k přístupu\\
			\hline
			1 & busy & Na systémové sběrnici probíhá komunikace\\
			\hline
			2 - 7 & res & Rezervováno pro budoucí využití\\
			\hline
			\noalign{\hrule height 1.2pt}
		\end{tabular}
  \end{center}
	\label{tab:status_vals}
\end{table}

Význam hodnot \textit{op} (operace) je popsán v tabulce \ref{tab:op_vals}. Výběr, zda bude následovat zápis nebo čtení, je určen nejnižším bitem pole \textit{op}, kde hodnota log. \texttt{0} určuje čtení a log. \texttt{1} zápis, což je shodné s významem signálu \texttt{we} systémové sběrnice popsaného v tabulce \ref{tab:ri5cy_bus}. Hodnota \textit{op} určuje také délku přenášených dat, kterou je tak možné kdykoliv změnit.

\begin{table}[!h]
  \caption{Tabulka možných hodnot operace navrženého protokolu}
  \begin{center}
  	\small
	  \begin{tabular}{!{\vrule width 1.2pt}M{0.7cm}|M{2.5cm}|M{10.4cm}!{\vrule width 1.2pt}}
	    \noalign{\hrule height 1.2pt}
	    \textit{op} & Význam & Popis\\
	    \noalign{\hrule height 1.2pt}
	    0 & byte / R & Čtení bajtu (následuje odesílání adresy)\\
			\hline	    
			1 & byte / W & Zápis bajtu (následuje odesílání adresy a zapisovaných dat)\\
			\hline
			2 & half-word / R & Čtení půlslova (následuje odesílání adresy)\\
			\hline	    
			3 & half-word / W & Zápis půlslova (následuje odesílání adresy a zapisovaných dat)\\
			\hline
			4 & word / R & Čtení slova (následuje odesílání adresy)\\
			\hline	   
			5 & word / W & Zápis slova (následuje odesílání adresy a zapisovaných dat)\\
			\hline
			6 & data & Uvozuje přenos vyčtených dat (pro Busy-mode). Rezervováno (pro Dynamic-mode)\\
			\hline
			7 & cmd & Uvozuje změnu módu\\
			\hline
			\noalign{\hrule height 1.2pt}
		\end{tabular}
  \end{center}
	\label{tab:op_vals}
\end{table}

%\subsubsection{Busy-wait mód - zápis} 
\subsubsection{Zápis na systémovou sběrnici prostřednictvím Busy-wait módu} 
Časový průběh JTAG komunikace pro zápis na systémovou sběrnici je zobrazen na obrázku \ref{fig:busy_w}. Po přechodu řídicího stavového automatu do stavu \textit{Shift-DR} je nejdříve odeslána 3-bitová hodnota \textit{op}, která může v případě zápisu nabývat hodnot 1, 3 nebo 5, podle délky zapisovaných dat. Následuje odesílání 21-bitové adresy, na kterou se budou data zapisovat. Po odeslání adresy následuje přenos požadovaných dat k zápisu, jejichž délka je určena hodnotou \textit{op}. Pro zápis na další adresu je třeba provést průchod řídicím stavovým automatem. Pokud je navrácena hodnota \textit{status} = 1 (\textit{busy}), nebudou následující data zapsána, protože se musí nejdříve dokončit probíhající zápis na systémovou sběrnici. Debugger by v tomto případě měl požadavek pro zapsání dat opakovat, dokud nebude navrácen \textit{status} = 0 (\textit{no\_err}).

\begin{figure}[H]
  \begin{center}
    \includegraphics[scale=0.575]{obrazky/busy_w.pdf}
  \end{center}
  \caption{Průběh zápisu na systémovou sběrnici v Busy-wait módu}
	\label{fig:busy_w}
\end{figure}

%\subsubsection{Busy-wait mód - čtení} 
\subsubsection{Čtení ze systémové sběrnice prostřednictvím Busy-wait módu} 
Časový průběh JTAG komunikace v případě čtení ze systémové sběrnice je zobrazen na obrázku \ref{fig:busy_r}. Nejprve je odeslána hodnota \textit{op} = 0, 2 nebo 4, která dle tabulky \ref{tab:op_vals} uvozuje odesílání adresy, ze které bude vyčtena hodnota registru a délka čtených dat. Jakmile je adresa odeslána, spustí se čtení ze systémové sběrnice. Debugger projde pomocí signálu \texttt{\acs{TMS}} řídicím stavovým automatem dle stavového diagramu na obrázku \ref{fig:tap_controller} zpět do stavu \textit{Shift-DR} a odešle hodnotu \textit{op} = 6, kterou je požadováno odeslání dat vyčtených z požadované adresy zpět prostřednictvím \texttt{\acs{TDO}} pinu. Délka odesílaných dat je určena předchozí hodnotou \textit{op} odesílanou v rámci přenosu adresy. V případě, že je navrácen \textit{status} = 1 (busy), odpovídající nedokončenému čtení ze systémové sběrnice, nejsou následující data platná. V dalším průchodu má debugger možnost žádat data znovu (\textit{op} = 6), nebo zažádat o vyčtení z jiné adresy.

\begin{figure}[H]
  \begin{center}
    \includegraphics[scale=0.68]{obrazky/busy_r.pdf}
  \end{center}
  \caption{Průběh čtení ze systémové sběrnice v Busy-wait módu}
	\label{fig:busy_r}
\end{figure}

\subsection{Varianta protokolu Dynamic-wait}	\label{subsec:dyn-wait}
Varianta protokolu \textbf{Dynamic-wait} je založena na principu signalizace probíhající operace na sběrnici prostřednictvím \texttt{\acs{TDO}} signálu v rámci stavu \textit{Shift-DR} řídicího stavového automatu. V případě probíhající operace na systémové sběrnici je tato informace signalizována úrovní log. \texttt{0} na \texttt{\acs{TDO}} pinu v definovaném úseku komunikace, tedy po spuštění požadavku na systémovou sběrnici. Jakmile je požadavek na systémovou sběrnici obsloužen dochází k nastavení \texttt{\acs{TDO}} pinu na úroveň log. \texttt{1} po jeden takt hodinového signálu \texttt{\acs{TCK}}. Poté následuje další komunikace, jak je dále popsáno pro zápis a čtení.

Použití tohoto komunikačního módu má velkou výhodu z hlediska časové optimalizace přenosu dat. Tato výhoda spočívá v nejkratší možné prodlevě způsobené obsluhou požadavku systémovou sběrnicí, protože je tato informace signalizována dynamicky ihned po dokončení požadavku. Další aspekt, že je tento způsob časově optimálnější, spočívá v setrvávání řídicího stavového automatu ve stavu \textit{Shift-DR} pro přístup k libovolnému počtu registrů. Při komunikaci se tak neztrácí čas průchodem stavového automatu přes stav \textit{Run-Test/Idle}. Další výhodou je možnost libovolně střídat zápis a čtení v rámci jednoho přechodu do stavu \textit{Shift-DR}. Tato vlastnost může výrazně časově optimalizovat případy komunikace, kdy je zapotřebí přečíst hodnotu registru a změnit v něm hodnoty pouze některých bitů.

K výběru, zda jde o zápis nebo čtení a délky odesílaných dat, je využita hodnota \textit{op} (operace), stějně jako v případě \textbf{Busy-wait} módu popsaného v podkapitole \ref{subsec:busy-wait}. Význam hodnot \textit{op} je popsán v tabulce \ref{tab:op_vals}. Rozdílem je nevyužití hodnoty \textit{op} = 6, protože v dynamickém režimu není potřeba, a zůstává tak volná pro budoucí využití.

%\subsubsection{Dynamic-wait mód - zápis} 
\subsubsection{Zápis na systémovou sběrnici prostřednictvím Dynamic-wait módu} 
Na obrázku \ref{fig:wait_w} je znázorněn časový průběh \acs{JTAG} komunikace popisující způsob zápisu na systémovou sběrnici v dynamickém  režimu. Po přechodu řídicího stavového automatu do stavu \textit{Shift-DR} je odeslána 3-bitová hodnota \textit{op}, která může nabývat hodnot 1, 3 nebo 5, podle délky zapisovaných dat, stejně jako při zápisu v \textbf{Busy-wait} módu. Následuje odeslání 21-bitové adresy, na kterou je požadováno data zapsat a ta se poté odesílají dle zvolené délky. Jakmile jsou data odeslána, následuje fáze zpracování zápisu systémovou sběrnicí, kdy je \acs{JTAG} systémem vystavena hodnota log. \texttt{0} na signálu \texttt{\acs{TDO}} a je držena, dokud zápis na systémové sběrnici není proveden. Po dokončení zápisu \acs{JTAG} systém nastaví signál \texttt{\acs{TDO}} na úroveň log. \texttt{1} po dobu jednoho taktu. Po takto zapsaných datech může být proveden zápis na další adresu okamžitě bez nutnosti průchodu stavovým automatem, jak je uvedeno na obrázku nebo může být přenos ukončen přechodem do stavu \textit{Exit1-DR}, což je zobrazeno na konci průběhu. V případě ukončení přenosu by měla být poslední hodnota na \texttt{\acs{TDO}} nastavena na hodnotu log. \texttt{0}. Důvodem je, že \acs{JTAG} systém po přechodu ze stavu \textit{Shift-DR} získává informaci právě v tomto taktu, a tudíž by pro případ pokračování v komunikaci hodnota log. \texttt{1} v tomto taktu znamenala \textit{status} = 1 (\textit{busy}), který v tomto dynamickém módu nemůže nastat.

\begin{figure}[!h]
  \begin{center}
    \includegraphics[scale=0.56]{obrazky/wait_w.pdf}
  \end{center}
  \caption{Průběh zápisu na systémovou sběrnici v Dynamic-wait módu}
	\label{fig:wait_w}
\end{figure}

%\subsubsection{Dynamic-wait mód - čtení} 
\subsubsection{Čtení ze systémové sběrnice prostřednictvím Dynamic-wait módu}
Časový průběh JTAG komunikace v případě čtení v dynamickém režimu je zobrazen na obrázku \ref{fig:wait_r}. Komunikace začíná odesláním hodnoty operace, která může nabývat hodnot \textit{op} = 0, 2 nebo 4, stejně jako pro čtení v \textbf{Busy-wait} módu. Následuje odesílání 21-bitové adresy. Po odeslání adresy je spuštěn požadavek čtení ze systémové sběrnice a následuje signalizace dokončení čtení. Jakmile je operace dokončena \acs{JTAG} systém nastaví signál \texttt{\acs{TDO}} na úroveň log. 1 a začne vysílat přečtená data. Po odeslání dat může debugger pokračovat v dalším čtením či zápisem s možností volby délky dat dle hodnoty \textit{op}, nebo komunikaci ukončit. 

\begin{figure}[!h]
  \begin{center}
    \includegraphics[scale=0.6]{obrazky/wait_r.pdf}
  \end{center}
  \caption{Průběh čtení ze systémové sběrnice v Dynamic-wait módu}
	\label{fig:wait_r}
\end{figure}

\section{Zhodnocení návrhu optimalizovaného protokolu pro přístup na systémovou sběrnici}
V této kapitole byly popsány dvě varianty protokolu pro přístup na systémovou sběrnici pomocí \acs{JTAG} rozhraní. Obě varianty jsou časově optimalizované, oproti přístupu na systémovou sběrnici pomocí původního testovacího systému popsaného v podkapitole \ref{sec:risc-v_dbg}.

První varianta je inspirována řešením původního testovacího systému, kde se využívá signalizace probíhajícího přístupu na systémovou sběrnici prostřednictvím \textit{status} kódu. Tato varianta je oproti původnímu přístupu časově optimálnější, protože není třeba procházet řídicím stavovým automatem v takové míře. Počet průchodů je snížen zejména, protože adresa registru a data se odesílají přímo, zatímco v případě použití původního přístupu je třeba zapisovat adresu registru a data postupně do registrů testovacího systému.

Druhá varianta protokolu je nejvíce časově optimální, protože využívá signalizace probíhajícího přístupu na systémovou sběrnici prostřednictvím \texttt{\acs{TDO}}. Díky této vlastnosti také není třeba procházet řídicím stavovým automatem během komunikace a po zpracování požadavku na systémové sběrnici nevzniká výrazná prodleva před započetím další komunikace.

\section{Implementace modulu pro přístup na systémovou sběrnici pomocí \acs{JTAG} protokolu}
Pro komunikaci s čipem pomocí navrženého optimalizovaného protokolu byla přidána nová \acs{JTAG} instrukce do testovacího systému \acs{RISC-V}. Zápisem instrukce \texttt{0x02} je aktivován modul pro přístup na systémovou sběrnici pomocí navrženého protokolu. Navržené řešení bylo popsáno v jazyce \acs{VHDL} a funkcionalita byla ověřena v simulátoru \texttt{Xcelium}, stejně jako modul pro dvouvodičový \acs{JTAG} protokol popsaný v podkapitole \ref{sec:cJTAG_adapter}.

\subsection{Základní popis návrhu}	\label{subsec:jtag_ap_impl}
Modul pro komunikaci pomocí navrženého protokolu je rozdělen na dvě části, z důvodu přechodu mezi hodinovými doménami \acs{JTAG} rozhraní (\texttt{\acs{TCK}}) a systémové sběrnice procesoru (\texttt{CLK}).

Část obvodu pro zpracování optimalizovaného protokolu je realizována implementací stavového automatu, který je taktován na nástupnou hranu \texttt{\acs{TCK}}. Tento stavový automat je rozdělený na tři části, pro zpracování varianty protokolu \textbf{Busy-wait}, \textbf{Dynamic-wait} a také pro změnu varianty. Všechny tyto části stavového automatu mají společné stavy \textit{S\_IDLE} a \textit{S\_OP} (operace). Přechody ze stavu \textit{S\_OP} rozdělují stavový automat podle vybrané varianty protokolu a také podle přijaté hodnoty \textit{op} definované tabulkou \ref{tab:op_vals}. Rozdělení stavového automatu v rámci stavu \textit{S\_OP} vyplývá z potřeby vyhodnocení hodnoty \textit{op} = 7 uvozující změnu varianty protokolu. Přechod mezi stavy \textit{S\_IDLE} a \textit{S\_OP} nastane vždy při přechodu řídicího stavového automatu do stavu \textit{Shift-DR} a zároveň za předpokladu, že je instrukce nastavena na hodnotu \texttt{0x02}.

Tento způsob implementace stavového automatu byl vybrán jako kompromis mezi variantou samostatných stavových automatů pro jednotlivé varianty protokolu a mezi jedním sdíleným stavovým automatem pro obě varianty. Pokud by bylo zvoleno řešení se samostatnými stavovými automaty, byly by stavy ve kterých nejsou přenášena data zbytečně duplikovány (například stav \textit{S\_IDLE}). Toto řešení by také zbytečně navyšovalo počet klopných obvodů potřebný pro definování stavů. V případě sloučení všech stavů pro oba protokoly, by byla kombinační logika stavového automatu výrazně složitější a pravděpodobně i celkově obsáhlejší. Toto řešení by mělo za následek výrazně náročnější implementaci případných změn či vylepšení (například přidání dalšího protokolu). 

\subsection{Varianta protokolu Busy-wait - stavový automat} \label{subsec:busy-wait-fsm}
Princip fungování části stavového automatu pro komunikaci pomocí \textbf{Busy-wait} varianty protokolu je znázorněn stavovým diagramem na obrázku \ref{fig:busy_wait_fsm}. Pokud je zvolena varianta protokolu \textbf{Busy-wait}, přechází automat ze stavu \textit{S\_OP} na základě přijaté hodnoty \textit{op} do stavu \textit{S\_ADDR\_SHIFT} nebo \textit{S\_DATA\_SHIFT}.

V případě zápisu na systémovou sběrnici stavový automat obsluhuje příjem adresy a dat podle časového průběhu na obrázku \ref{fig:busy_w}. Jednotlivým částem průběhu odpovídají jednotlivé stavy, mezi kterými automat přechází na základě jejich délky, která je stanovena hodnotou čítače \texttt{shift\_cnt}. Tento čítač je při každé změně stavu resetován. Ze stavu \textit{S\_OP} přechází automat do stavu \textit{S\_ADDR\_SHIFT}, jakmile je přijata hodnota \textit{op}, která je přidružena přenosu adresy. Po přijetí adresy přechází automat do stavu \textit{S\_DATA\_SHIFT} pro příjem zapisovaných dat, a to za předpokladu, že hodnota \textit{op} odpovídá zápisu (\texttt{we} = 1). Po přijetí zapisovaných dat dané délky, která byla určena na základě hodnoty \textit{op} podle tabulky \ref{tab:op_vals}, přechází automat do stavu \textit{S\_REQUEST}. Ve stavu \textit{S\_REQUEST} automat setrvá vždy pouze jeden hodinový takt \texttt{\acs{TCK}} je vystaven požadavek na systémovou sběrnici, pokud je předchozí požadavek dokončen (\texttt{rq\_start} = 1).

Průchody stavovým automatem pro čtení ze systémové sběrnice odpovídají časovému průběhu na obrázku \ref{fig:busy_r}. Při prvním průchodu je přijata adresa registru, ze kterého je požadováno data vyčíst. Tato fáze je obdobná, jako v případě zápisu s rozdílem, že ze stavu \textit{S\_ADDR\_SHIFT} se přechází přímo do stavu \textit{S\_REQUEST} (\texttt{we} = 0), kde je vyvolán požadavek na vyčtení dat z adresy (\texttt{rq\_start} = 1). Při druhém průchodu je očekávána hodnota \textit{op} = 6, čímž je podmíněn přechod ze stavu \textit{S\_OP} přímo do stavu \textit{S\_DATA\_SHIFT}, kde jsou vyčtená data odeslána zpět prostřednictvím \texttt{\acs{TDO}} pinu. Stavem \textit{S\_REQUEST} se po odeslání dat pouze prochází do stavu \textit{S\_IDLE}, a požadavek na systémovou sběrnici není iniciován.

\begin{figure}[H]
  \begin{center}
    \includegraphics[scale=0.22]{obrazky/busy_fsm_simple.pdf}
  \end{center}
  \caption{Stavový diagram části stavového automatu pro busy-wait mód}
	\label{fig:busy_wait_fsm}
\end{figure}
\pagebreak

\subsection{Varianta protokolu Dynamic-wait - stavový automat} \label{subsec:dyn-wait-fsm}
Stavový diagram pro část stavového automatu realizující komunikaci pomocí \textbf{Dynamic-wait} varianty protokolu je znázorněn na obrázku \ref{fig:dyn_wait_fsm}. Stavy \textit{S\_ADDR\_SHIFT}, \textit{S\_DATA\_SHIFT} a \textit{S\_REQUEST} nejsou stejnými stavy jako v případě obrázku \ref{fig:busy_wait_fsm}, ale jsou stejně pojmenované pro názornost. Oproti komunikaci \textbf{Busy-wait} variantou protokolu automat přechází ze stavu \textit{S\_OP} do stavu \textit{S\_ADDR\_SHIFT} v případě čtení i zápisu.

V případě příjmu hodnoty \textit{op} odpovídající zápisu na systémovou sběrnici (\texttt{we} = 1) přechází stavový automat ze stavu \textit{S\_ADDR\_SHIFT} do stavu \textit{S\_DATA\_SHIFT} po přijetí adresy na základě hodnoty čítače \texttt{shift\_cnt} obdobně jako v případě zápisu v \textbf{Busy-wait} režimu. Komunikace odpovídá časovému průběhu na obrázku \ref{fig:wait_w}. Po příjmu zapisovaných dat přechází automat do stavu \textit{S\_REQUEST}, kdy je spuštěn požadavek zápisu na systémovou sběrnici, který není oproti předchozí variantě protokolu podmíněn hodnotou \textit{status} = 1 (\textit{busy}), protože tato situace z principu tohoto protokolu nenastane. V tomto stavu již začíná, z pohledu této varianty protokolu, fáze čekání na zpracování požadavku na systémové sběrnici, proto musí být na \texttt{\acs{TDO}} pinu nastavena hodnota log. \texttt{0}. Z tohoto důvodu jsou nastaveny signály \texttt{tdo\_dyn} = 0 a \texttt{tdo\_dyn\_en} = 1 již v tomto stavu. Poté automat přechází do stavu \textit{S\_WAIT}, kde setrvá dokud není zápis na systémovou sběrnici dokončen. Jakmile je zápis dokončen přechází automat do stavu \textit{S\_ACK}, kde je nastaven signál \texttt{tdo\_dyn} na hodnotu log. \texttt{1}. S dalším hodinovým taktem přechází automat zpět do stavu \textit{S\_OP} a komunikace může pokračovat nebo ji debugger může ukončit přechodem ze stavu \textit{Shift-DR} řídicího stavového automatu.

Pro čtení ze systémové sběrnice je třeba po přijetí adresy čteného registru zadat požadavek na systémovou sběrnici, jak je vidět z časového průběhu na obrázku \ref{fig:wait_r}. Stavový automat tedy přejde ze stavu \textit{S\_ADDR\_SHIFT} do stavu \textit{S\_REQUEST} (\texttt{we} = 0). Poté následuje čekání na provedení operace systémovou sběrnicí ve stavu \textit{S\_WAIT} a následně \textit{S\_ACK} stejně jako v případě zápisu. Jakmile je operace dokončena přechází automat do stavu \textit{S\_DATA\_SHIFT} a data jsou odeslána zpět prostřednictvím \texttt{\acs{TDO}} pinu. Po přenosu požadovaných dat přechází automat do stavu \textit{S\_OP} a debugger může pokračovat v komunikaci nebo komunikaci ukončit přechodem ze stavu \textit{Shift-DR} řídicího stavového automatu.

\begin{figure}[H]
  \begin{center}
    \includegraphics[scale=0.19]{obrazky/dyn_fsm_simple.pdf}
  \end{center}
  \caption{Stavový diagram části stavového automatu pro dynamic-wait mód.}
	\label{fig:dyn_wait_fsm}
\end{figure}

\subsection{Výběr varianty protokolu - stavový automat} \label{subsec:cmd-fsm}
Diagram části stavového automatu pro změnu používané varianty protokolu je znázorněn na obrázku \ref{fig:cmd_fsm}. Pokud je přijata hodnota \textit{op} = 7 uvozující změnu módu přechází stavový automat do stavu \textit{S\_CMD\_SHIFT}, kde je přijata hodnota \textit{cmd}, která odpovídá danému režimu podle tabulky \ref{tab:cmd_vals}. Stavový automat po přijetí 4-bitové hodnoty \textit{cmd} přechází zpět do stavu \textit{S\_IDLE} a hodnota je zapsána do příslušného registru.

\begin{figure}[!h]
  \begin{center}
    \includegraphics[scale=0.3]{obrazky/cmd_fsm_simple.pdf}
  \end{center}
  \caption{Stavový diagram části stavového automatu pro změnu módu}
	\label{fig:cmd_fsm}
\end{figure}

\begin{table}[!h]
  \caption{Tabulka možných hodnot \textit{cmd} registru}
  \begin{center}
  	\small
	  \begin{tabular}{!{\vrule width 1.2pt}c|c|c!{\vrule width 1.2pt}}
	    \noalign{\hrule height 1.2pt}
	    \textit{cmd} & Význam & Popis\\
	    \noalign{\hrule height 1.2pt}
	    0 & \textbf{Busy-wait} & Následující komunikace může probíhat v \textbf{Busy-wait} módu\\
			\hline
			1 & \textbf{Dynamic-wait} & Následující komunikace může probíhat v \textbf{Dynamic-wait} módu\\
			\hline
			2 - 15 & res & Rezervováno pro budoucí využití\\
			\hline
			\noalign{\hrule height 1.2pt}
		\end{tabular}
  \end{center}
	\label{tab:cmd_vals}
\end{table}