% Pro sazbu seznamu literatury použijte jednu z následujících možností

%%%%%%%%%%%%%%%%%%%%%%%%%%%%%%%%%%%%%%%%%%%%%%%%%%%%%%%%%%%%%%%%%%%%%%%%%
%1) Seznam citací definovaný přímo pomocí prostředí literatura / thebibliography

\begin{thebibliography}{99}
	
\bibitem{sr72/2017}
	VYSOKÉ UČENÍ TECHNICKÉ V~BRNĚ.
	\emph{Směrnice č.\,72/2017, Úprava, odevzdávání a~zveřejňování závěrečných prací.}
	Online. Brno: VUT v~Brně, 2017.
	Úplné znění ke dni 11.\,4.\,2022.
	Dostupné z:\\
	{\small
	\url{https://www.vut.cz/uredni-deska/vnitrni-predpisy-a-dokumenty/smernice-c-72-2017-uprava-odevzdavani-a-zverejnovani-zaverecnych-praci-d161410}.}
	[cit.\ 2023-09-27].

\bibitem{CSN_ISO_690-2022}
    ÚŘAD PRO TECHNICKOU NORMALIZACI, METROLOGII A~STÁTNÍ ZKUŠEBNICTVÍ.
    ČSN ISO 690:2022 (01 0197), \emph{Informace a dokumentace -- Pravidla pro bibliografické odkazy a~citace informačních zdrojů.}
    Čtvrté vydání. Praha, 2022.

\bibitem{CSN_ISO_7144-1997}
    ÚŘAD PRO TECHNICKOU NORMALIZACI, METROLOGII A~STÁTNÍ ZKUŠEBNICTVÍ.
    ČSN ISO 7144 (010161), \emph{Dokumentace -- Formální úprava disertací a~podobných dokumentů.}
%    24 stran.
    Praha, 1997.

\bibitem{CSN_ISO_31-11}
    ÚŘAD PRO TECHNICKOU NORMALIZACI, METROLOGII A~STÁTNÍ ZKUŠEBNICTVÍ.
    ČSN ISO 31-11, \emph{Veličiny a~jednotky -- část 11: Matematické znaky a~značky používané ve fyzikálních vědách a~v~technice.}
    Praha, 1999.

\bibitem{RajmicSysel2002}
    RAJMIC, P. a SYSEL, P.
    Wavelet Spectrum Thresholding Rules.
    In: \emph{Proceedings of the International Conference Research in Telecommunication Technology}.
    Žilina: Žilina University, 2002. s.\,60--63. ISBN 80-7100-991-1.

\bibitem{IEEE_1149-1}
IEEE Standard for Test Access Port and Boundary-Scan Architecture. Online. In: . S.~1-422. ISBN 978-0-7381-8263-6. Dostupné z: \url{https://doi.org/10.1109/IEEESTD.2013.6515989}. [cit. 2023-11-02].

\bibitem{JTAG_TAP_diagram}
\textit{Joint Test Action Group}. Online. In: Wikipedia: the free encyclopedia. San Francisco (CA): Wikimedia Foundation, 2001-. Dostupné z: \url{https://de.m.wikipedia.org/wiki/Joint\_Test\_Action\_Group}. [cit. 2023-10-31].

\bibitem{JTAG}
\textit{JTAG}. Online. In: Wikipedia: the free encyclopedia. San Francisco (CA): Wikimedia Foundation, 2001-. Dostupné z: \url{https://en.wikipedia.org/wiki/JTAG}. [cit. 2023-11-02].

\bibitem{risc-v_dbg}
\textit{RISC-V External Debug Support Version 0.13.2}. Online. San Mateo, California, U.S.: SiFive, 2019. Dostupné z: \url{https://riscv.org/wp-content/uploads/2019/03/riscv-debug-release.pdf}. [cit. 2023-08-01].

\bibitem{ri5cy}
\textit{RI5CY: User Manual}. Online. 2019. Dostupné z: \url{https://www.pulp-platform.org/docs/ri5cy\_user\_manual.pdf}. [cit. 2023-11-05].

\end{thebibliography}


%%%%%%%%%%%%%%%%%%%%%%%%%%%%%%%%%%%%%%%%%%%%%%%%%%%%%%%%%%%%%%%%%%%%%%%%%
%%2) Seznam citací pomocí BibTeXu
%% Při použití je nutné v TeXnicCenter ve výstupním profilu aktivovat spouštění BibTeXu po překladu.
%% Definice stylu seznamu
%\bibliographystyle{unsrturl}
%% Pro českou sazbu lze použít styl czechiso.bst ze stránek
%% http://www.fit.vutbr.cz/~martinek/latex/czechiso.tar.gz
%%\bibliographystyle{czechiso}
%% Vložení souboru se seznamem citací
%\bibliography{text/literatura}
%
%% Následující příkaz je pouze pro ukázku sazby literatury při použití BibTeXu.
%% Způsobí citaci všech zdrojů v souboru literatura.bib, i když nejsou citovány v textu.
%\nocite{*}