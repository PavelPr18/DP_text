\chapter{Komunikační protokol JTAG}
%Tato kapitola popisuje princip přenosu dat pomocí komunikačního protokolu JTAG a jeho použití pro testování obvodů obsahující procesor architektury \acs{RISC} (\textit{\acl{RISC}}) s instrukční sadou \acs{RISC-V}. V kapitole je také přiblížen princip fungování systémové sběrnice použitého \acs{RISC-V} procesoru.
Tato kapitola popisuje princip přenosu dat pomocí komunikačního protokolu JTAG a jeho použití pro testování obvodů \hl{obsahujících procesor s instrukční sadou {\acs{RISC-V}}}. V kapitole je také přiblížen princip fungování systémové sběrnice použitého \acs{RISC-V} procesoru.

\section{Základní popis \acs{JTAG} protokolu}
\acs{JTAG} (\textit{\acl{JTAG}}) je standardizovaná komunikační sběrnice určená pro testování integrovaných obvodů a \acs{DPS} (\textit{\acl{DPS}}).
\acs{JTAG} rozhraní definuje tzv. \acs{TAP} (\textit{\acl{TAP}}), který je skupinou vstupů a výstupů určených k testování. Jelikož \acs{JTAG} je synchronní sběrnice, její rozhraní tedy zahrnuje samostatný vodič pro hodinový signál \texttt{\acs{TCK}} (\textit{\acl{TCK}}). Dalším vodičem je \texttt{\acs{TMS}} (\textit{\acl{TMS}}), kterým je přenášen řídicí signál pro stavový automat. Vodiče zajišťující sériový přenos dat jsou označeny \texttt{\acs{TDI}} (\textit{\acl{TDI}}) a \texttt{\acs{TDO}} (\textit{\acl{TDO}}). Jako volitelný vodič základního rozhraní je možné připojit také \texttt{\acs{TRST}} (\textit{\acl{TRST}}), který provádí reset stavového automatu. Reset se využívá především k inicializaci automatu během připojení napájení. \cite {IEEE_1149-1} \cite{JTAG}

\section{Řídicí stavový automat - \acs{TAP}}
Standard IEEE 1149.1 definuje stavový automat, který je hlavní součástí \acs{TAP} (\textit{\acl{TAP}}). Stavový diagram stavového automatu je zobrazen na obrázku \ref{fig:tap_controller}. Všechny přechody mezi stavy toho automatu jsou plně synchronní s náběžnou hranou vstupního hodinového signálu \texttt{\acs{TCK}} a jsou určeny řídicím signálem \texttt{\acs{TMS}}. Důležitou vlastností tohoto stavového automatu je způsob resetování jeho stavu, tedy návrat do stavu \textit{Test-Logic-Reset}. Reset lze provést minimálně pěti po sobě jdoucími hodinovými takty, kdy hodnota řídicího signálu \texttt{\acs{TMS}} setrvává v log. \texttt{1}. Tento způsob resetování je možné provést ze kteréhokoliv stavu automatu. Příkladem může být provedení resetu v případě, že se automat nachází ve stavu \textit{Shift-DR}. Automat tedy projde postupně stavy \textit{Exit1-DR}, \textit{Update-DR}, \textit{Select-DR-Scan} a \textit{Select-IR-Scan} až do stavu \textit{Test-Logic-Reset}. Hlavními stavy, při kterých probíhá přenos dat signály \texttt{\acs{TDI}} a \texttt{\acs{TDO}}, jsou \textit{Shift-DR} a \textit{Shift-IR}, tedy přenos dat a instrukce do datového a instrukčního posuvného registru. \cite {IEEE_1149-1}

%TODO: vymenit spojovnik v nazvech stavů za pomlčku??

\begin{figure}[H]
  \begin{center}
    \includegraphics[scale=0.35]{obrazky/JTAG_TAP_Controller_State_Diagram.png}
  \end{center}
  \caption{Stavový diagram řídicího stavového automatu JTAGu \cite{JTAG_TAP_diagram}}
	\label{fig:tap_controller}
\end{figure}

\section{Systém pro testování \acs{RISC-V}}		\label{sec:risc-v_dbg}
Pro procesory architektury \acs{RISC-V} je definován systém pro jejich testování a ladění, dle specifikace \textit{RISC-V External Debug Support}. Blokové schéma systému je zobrazené na obrázku \ref{fig:blok_sch_risc-v_dbg}. Nadřazeným systémem pro testování integrovaných obvodů je dle schématu \textit{debugger}, kterým je zpravidla testovací software v PC, \hl{například OpenOCD, TRACE32 a další}. Pro jeho spojení s integrovaným obvodem je třeba zpravidla použít převodník z USB na JTAG protokol. Systém pro přístup k procesorovému jádru se nazývá \ac{DM}. Součástí Debug modulu je blok realizující přístup na systémovou sběrnici (\textit{System Bus Access}), který je popsán v podkapitole \ref{subsec:dm_sba}. Přechod mezi hodinovými doménami \acs{JTAG} rozhraní a systémové sběrnice v tomto systému realizuje rozhraní \ac{DMI}. \cite{risc-v_dbg}

%Systém umožňuje vykonávat krátký program 

\begin{figure}[!h]
  \begin{center}
    \includegraphics[scale=0.9]{obrazky/risc-v_debug_system_blok_sch.png}
  \end{center}
  \caption{Blokové schéma systému použitého pro testování \acs{RISC-V} procesorů \cite{risc-v_dbg}}
	\label{fig:blok_sch_risc-v_dbg}
\end{figure}

\subsection{Instrukce testovacího systému}	\label{subsec:jtag_instr}
Systém pro testování \acs{RISC-V} definuje základní \acs{JTAG} instrukce dle tabulky \ref{tab:jtag_instr}. Délka instrukčního registru je v tomto případě 5 bitů. Výchozí hodnotou instrukčního registru je \textit{IDCODE}, což umožňuje vyčíst z datového registru informace o zařízení. V případě, že je zvolena instrukce \textit{BYPASS}, funguje datový registr jako jednobitový posuvný registr mezi piny \texttt{TDI} a \texttt{TDO}, a jeho hodnota nemá z hlediska systému žádnou funkci. Této možnosti se využívá při zapojení několika \acs{JTAG} zařízení do série (\textit{Daisy-Chain}). Pro testování \acs{RISC-V} je specifikována instrukce \textit{DTMCS} umožňující především reset testovacího systému a instrukce \textit{DMI Access} umožňující přístup k registrům testovacího systému popsaného v podkapitole \ref{subsec:dm_reg_access}. \cite {IEEE_1149-1} \cite{risc-v_dbg}

\begin{table}[!h]
  \caption{Tabulka možných hodnot instrukčního registru. \cite{risc-v_dbg}}
  \begin{center}
  	\small
	  \begin{tabular}{!{\vrule width 1.2pt}M{1.5cm}|M{2.5cm}|M{9cm}!{\vrule width 1.2pt}}
	    \noalign{\hrule height 1.2pt}
	    Hodnota & Význam & Popis\\
	    \noalign{\hrule height 1.2pt}
	    \texttt{0x00} & BYPASS & Komunikace se zařízením je deaktivována\\
			\hline	    
			\texttt{0x01} & IDCODE & Možnost přečíst identifikační kód zařízení\\
			\hline
			\texttt{0x10} & DTMCS & Řídicí a stavový registr testovacího systému\\
			\hline	    
			\texttt{0x11} & DMI Access & Přístup k registrům testovacího systému\\
			\hline
			\texttt{0x1F} & BYPASS & Komunikace se zařízením je deaktivována\\
			\hline
			\noalign{\hrule height 1.2pt}
		\end{tabular}
  \end{center}
	\label{tab:jtag_instr}
\end{table}

\subsubsection{Změna instrukce testovacího systému}
Časový průběh změny instrukce testovacího systému je znázorněn na obrázku \ref{fig:jtag_instr}. Řídicí stavový automat je nejdříve doveden do stavu \textit{Shift-IR}, kde je odeslána 5-ti bitová hodnota požadované instrukce. Následným přechodem do stavu \textit{Update-IR} (na obrázku \texttt{Up-I}) je požadovaná instrukce nastavena a v systému je vybrán požadovaný datový registr. 

\begin{figure}[!h]
  \begin{center}
    \includegraphics[scale=0.76]{obrazky/instr_general.pdf}
  \end{center}
  \caption{Průběh přepnutí \acs{JTAG} instrukce.}
	\label{fig:jtag_instr}
\end{figure}

\subsection{Přístup k registrům testovacího systému}		\label{subsec:dm_reg_access}
Modul sloužící k ladění procesorového jádra (\textit{\acl{DM}}) obsahuje sadu 32-bitových registrů. K těmto registrům může debugger přistoupit prostřednictvím \acs{JTAG} protokolu a následně \acs{DMI} rozhraní, jak je možné vidět na blokovém schématu na obrázku \ref{fig:blok_sch_risc-v_dbg}. Rozhraní \acs{DMI} je definováno jako 41-bitový registr, který obsahuje 7-bitovou adresu zvoleného registru, data pro zápis či čtení z registru a dvoubitovou hodnotu \textit{op} (operace). Bitová pole registru jsou znázorněna tabulkou \ref{tab:dmi_access}. \cite{risc-v_dbg}

\begin{table}[H]
  \caption{Formát datového registru pro přístup k registrům testovacího systému. \cite{risc-v_dbg}}
  \begin{center}
  	\small
	  \begin{tabular}{!{\vrule width 1.2pt}M{2cm}|M{10.5cm}|M{1cm}!{\vrule width 1.2pt}}
	    \noalign{\hrule height 1.2pt}
				\multicolumn{3}{!{\vrule width 1.2pt}c!{\vrule width 1.2pt}}{DMI register}\\
				\hline
				address [7] & data [32] & op [2]\\
				\hline
			\noalign{\hrule height 1.2pt}
		\end{tabular}
  \end{center}
	\label{tab:dmi_access}
\end{table}

Tento registr funguje jako posuvný registr pro komunikaci prostřednictvím \acs{JTAG} protokolu. Pro aktivaci přístupu k registrům popisovaným způsobem je třeba přepnout instrukci zápisem hodnoty \texttt{0x11} (\textit{DMI Access}) do instrukčního registru, jak je uvedeno v podkapitole \ref{subsec:jtag_instr}. Poté může debugger zahájit komunikaci s testovacím systémem.
 
Obecný časový průběh přístupu na danou adresu testovacího systému je znázorněn na obrázku \ref{fig:dmi}. Kromě signálů \acs{JTAG} rozhraní je zobrazen také příslušný stav řídicího stavového automatu. Komunikace začíná vždy přechodem stavového automatu do stavu \textit{Shift-DR} a odesláním požadované hodnoty na vstup \texttt{TDI}. \hl{Hodnota DMI registru je vždy odeslána v pořadí od nejméně významného bitu ({\acs{LSB}}) po nejvíce významný bit ({\acs{MSB}}).} Přičemž zpracování požadavku je vždy zahájeno přechodem automatu do stavu \textit{Update-DR} (na obrázku \texttt{Up-D}). Hodnota odeslaná zpět prostřednictvím pinu \texttt{TDO} je do posuvného registru přiřazena v rámci průchodu stavem \textit{Capture-DR} (na obrázku \texttt{Ca-D}). \cite{risc-v_dbg}

%TODO: prepnuti instrukce??

\begin{figure}[!h]
  \begin{center}
    \includegraphics[scale=0.61]{obrazky/dmi.pdf}
  \end{center}
  \caption{Průběh \acs{JTAG} komunikace pro přístup k registru testovacího systému.}
	\label{fig:dmi}
\end{figure}

\subsubsection{Význam hodnoty operace pro přístup k registrům testovacího systému}
Hodnota operace (\textit{op}) při odeslání požadavku pro přístup k registru testovacího systému může nabývat hodnot 0, 1 nebo 2 dle tabulky \ref{tab:dmi_access_op}.

\begin{table}[H]
  \caption{Tabulka možných hodnot operace \acs{DMI} rozhraní. \cite{risc-v_dbg}}
  \begin{center}
  	\small
	  \begin{tabular}{!{\vrule width 1.2pt}M{1.5cm}|M{2cm}|M{8cm}!{\vrule width 1.2pt}}
	    \noalign{\hrule height 1.2pt}
	    op & Význam & Popis\\
	    \noalign{\hrule height 1.2pt}
	    0 & nop & Následující data a adresa jsou ignorovány\\
			\hline	    
			1 & read & Čtení hodnoty registru z adresy\\
			\hline
			2 & write & Zápis přijatých dat na registr\\
			\hline	    
			3 & res & Rezervováno pro budoucí využití\\
			\hline
			\noalign{\hrule height 1.2pt}
		\end{tabular}
  \end{center}
	\label{tab:dmi_access_op}
\end{table}

V případě odpovědi \acs{JTAG} systému mohou být hodnoty přijaté debuggerem 0, 2 nebo 3, jejichž význam je popsaný tabulkou \ref{tab:dmi_access_op_response}. V případě příjmu hodnoty 0 je debugger informován, že následující požadavek na přístup k registru bude akceptován, protože předchozí operace byla dokončena. V opačném případě je \acs{JTAG} systémem odeslána hodnota 3, což indikuje probíhající operaci a aktuální požadavek nemůže být vykonán. Pokud nastane tato situace, je hodnota 3 navrácena při každém dalším požadavku. Pro vymazání tohoto stavu je třeba \hl{tento stav} resetovat zápisem do řídicího registru \textit{DTMCS}. \hl{Tato vlastnost je značnou nevýhodou tohoto řešení, protože změna instrukce pro zápis do \textit{DTMCS} registru, zápis do registru a změna instrukce zpět prodlužuje čas přístupu k požadovanému registru testovacího systému.} \cite{risc-v_dbg}

Jelikož požadavek je vždy odstartován přechodem stavového automatu do stavu \textit{Update-DR} a návratová hodnota operace je vyhodnocena při průchodu stavem\\ \textit{Capture-DR}, bude čas na vykonání požadavku vždy určený počtem taktů \texttt{TCK} hodinového signálu ve stavu \textit{Run-Test/Idle}.

\begin{table}[!h]
  \caption{Tabulka možných návratových hodnot operace \acs{DMI} rozhraní. \cite{risc-v_dbg}}
  \begin{center}
  	\small
	  \begin{tabular}{!{\vrule width 1.2pt}M{1.5cm}|M{2cm}|M{8cm}!{\vrule width 1.2pt}}
	    \noalign{\hrule height 1.2pt}
	    op & Význam & Popis\\
	    \noalign{\hrule height 1.2pt}
	    0 & ok & Předchozí operace proběhla v pořádku\\
			\hline	    
			1 & res & Rezervováno pro budoucí využití\\
			\hline
			2 & err & Chyba Debug modulu\\
			\hline	    
			3 & busy & Předchozí požadavek nebyl dokončen\\
			\hline
			\noalign{\hrule height 1.2pt}
		\end{tabular}
  \end{center}
	\label{tab:dmi_access_op_response}
\end{table}

\subsection{Přístup na systémovou sběrnici pomocí Debug modulu}		\label{subsec:dm_sba}
Pro přístup	na systémovou sběrnici jsou v debug modulu vyhrazeny registry popsané tabulkou \ref{tab:sba_regs}. Řídicí a stavový registr \textit{sbcs} umožňuje nastavení režimu zápisu nebo čtení ze systémové sběrnice, případně také volbu možnosti automatické inkrementace adresy. Registr \textit{sbcs} také indikuje probíhající operaci na systémové sběrnici bitem \textit{sbbusy}. Další důležitou informací uloženou v tomto stavovém registru je šířka systémové sběrnice procesoru, přičemž možné hodnoty jsou 8, 16, 32, 64 nebo 128 bitů. Registr \textit{sbaddress0} slouží k nastavení požadované adresy, na kterou se bude přistupovat, a registr \textit{sbdata0} slouží pro zápis nebo čtení dat. \cite{risc-v_dbg}

\begin{table}[!h]
  \caption{Tabulka registrů systému pro přístup na systémovou sběrnici. \cite{risc-v_dbg}}
  \begin{center}
  	\small
	  \begin{tabular}{!{\vrule width 1.2pt}M{1.6cm}|M{1cm}|M{11cm}!{\vrule width 1.2pt}}
	    \noalign{\hrule height 1.2pt}
	    Označení & Adresa & Popis\\
	    \noalign{\hrule height 1.2pt}
	    \textit{sbcs} & \texttt{0x38} & Řídicí a stavový registr bloku pro přístup na systémovou sběrnici\\
			\hline
			\textit{sbaddress0} & \texttt{0x39} & Registr obsahující adresu pro přístup na systémovou sběrnici\\
			\hline
			\textit{sbdata0} & \texttt{0x3C} & Registr obsahující data pro čtení/zápis na systémovou sběrnici\\
			\hline			
			\noalign{\hrule height 1.2pt}
		\end{tabular}
  \end{center}
	\label{tab:sba_regs}
\end{table}

\subsubsection{Zápis na systémovou sběrnici pomocí Debug modulu}
%Pro možnost zápisu na systémovou sběrnici je třeba nejdříve zapsat do řídicího registru \textit{sbcs} hodnotu \texttt{0x00040000}. Tato hodnota nastaví přístup ke sběrnici o šířce 32 bitů a režim zápisu. \cite{risc-v_dbg}
\hl{Pro možnost zápisu na systémovou sběrnici je třeba nejdříve zapsat do řídicího registru \textit{sbcs} příslušnou hodnotu pro nastavení přístupu ke sběrnici o šířce 32 bitů a režimu zápisu.} \cite{risc-v_dbg}

Pokud je režim zápisu již nastaven, je možné provést samotný zápis, jehož časový průběh je znázorněn na obrázku \ref{fig:dmi_w}. Zápis na systémovou sběrnici tedy obsahuje zápis požadované adresy do registru \textit{sbaddress0} (\texttt{0x39}) a zapisovaných dat \textit{sbdata0} (\texttt{0x3C}). Jelikož jde v obou případech o zápis do registru, obsahuje návratová hodnota \acs{DMI} registru vždy předchozí data a adresu.
 
\begin{figure}[!h]
  \begin{center}
    \includegraphics[scale=0.6]{obrazky/dmi_w_split.pdf}
  \end{center}
  \caption{Průběh \acs{JTAG} komunikace pro zápis na systémovou sběrnici.}
	\label{fig:dmi_w}
\end{figure}

\subsubsection{Čtení ze systémové sběrnice pomocí Debug modulu}
Pro aktivaci režimu čtení ze systémové sběrnice je třeba zapsat do řídicího registru \textit{sbcs} hodnotu \texttt{0x00140000}. Oproti zápisu se tak nastaví i bit \textit{sbreadonaddr}, což způsobí spuštění čtení ze systémové sběrnice po zápisu dané adresy do registru \textit{sbaddress0}.

Časový průběh čtení ze systémové sběrnice je zobrazen na obrázku \ref{fig:dmi_r}. V prvním datovém rámci je odeslána hodnota požadované adresy registru pro čtení, což čtení ze systémové sběrnice spustí. Druhým rámcem je odeslán požadavek o vyčtení dat z registru \textit{sbdata0} (\texttt{0x3C}). Platná data jsou do posuvného registru přiřazena až v následujícím průchodu řídicím stavovým automatem, a proto musí být odeslán třetí datový rámec pro přenesení platných dat z \acs{DMI} registru.

\begin{figure}[!h]
  \begin{center}
    \includegraphics[scale=0.6]{obrazky/dmi_r_split.pdf}
  \end{center}
  \caption{Průběh \acs{JTAG} komunikace pro zápis na systémovou sběrnici.}
	\label{fig:dmi_r}
\end{figure}

\section{Systémová sběrnice PULP RISC-V}
Jako procesorové jádro je v integrovaném obvodu, pro který je rozšíření JTAG front-endu navrženo, použito procesorové jádro typu \acs{RISC-V}. Konkrétní typ procesorového jádra je \acs{PULP} (\acl{PULP}) RI5CY 32-bitový procesor. Systémová sběrnice tohoto procesorového jádra je založena na jednoduchém \textbf{Request-Grant} protokolu. Na sběrnici mohou být obecně připojeny \textit{master} a \textit{slave} moduly. Jako master modul je možné uvažovat dříve popsaný testovací systém nebo nově navržený rozšiřující modul popsaný v kapitole \ref{jtag_ap}. Slave modulem může být například paměť.

\subsection{Signály systémové sběrnice}
V tabulce \ref{tab:ri5cy_bus} jsou \hl{obecně} popsány jednotlivé signály systémové sběrnice procesory PULP RICS-V.

\begin{table}[H]
	\FloatBarrier
  \caption{Tabulka popisu signálů systémové sběrnice PULP RICS-V. \cite{ri5cy}}
  \begin{center}
  	\small
	  \begin{tabular}{!{\vrule width 1.2pt}M{2.2cm}|M{1.5cm}|M{1.8cm}|M{7cm}!{\vrule width 1.2pt}}
	    \noalign{\hrule height 1.2pt}
	    Název signálu & Označení signálu & Počet bitů & Význam signálu\\
	    \noalign{\hrule height 1.2pt}
	    Address & \texttt{addr} & 32 & Adresa registru ke kterému se bude přistupovat\\
			\hline
			Write data & \texttt{wdata} & 32 & Data zapisovaná do registru dle adresy\\
			\hline
			Request & \texttt{req} & 1 & Požadavek přístupu na sběrnici\\
			\hline
			Grant & \texttt{gnt} & 1 & Potvrzení příjmu požadavek přístupu na sběrnici\\
			\hline			
			Read data valid & \texttt{rvalid} & 1 & Potvrzení platnosti čtených dat\\
			\hline
			Read data & \texttt{rdata} & 32 & Data čtená z registru dle adresy\\
			\hline
			Write enable & \texttt{we} & 1 & Určuje zápis(log. \texttt{1})/čtení(log. \texttt{0})\\
			\hline
			Byte enable & \texttt{be} & 4 & Vybírá bajty, které budou zapsány/čteny\\
			\hline
			\noalign{\hrule height 1.2pt}
		\end{tabular}
  \end{center}
	\label{tab:ri5cy_bus}
\end{table}

\subsection{Průběh přístupu na systémovou sběrnici}
Obecný průběh přístupu na systémovou sběrnici je zobrazen na obrázku \ref{fig:ri5cy_bus_basic}, \hl{kde je použitý příklad přístupu k paměti, přičemž pro jiné \textit{slave} moduly bude protokol sběrnice fungovat totožně.} Pokud chce \textit{master} modul přistoupit na sběrnici, vystaví adresu na \texttt{data\_addr\_o}, nastaví signály \texttt{data\_we\_o}, \texttt{data\_be\_o} \hl{a v případě zápisu vystaví také zapisovaná data \texttt{data\_wdata\_o}. Poté \textit{master} nastaví request \texttt{data\_req\_o} na hodnotu log. \texttt{1}, čímž podává požadavek na přístup.} Jakmile je \textit{slave} modul připravený obsloužit požadavek, nastaví \texttt{data\_gnt\_i} na hodnotu log. \texttt{1}. Po přijetí \texttt{data\_gnt\_i} může \textit{master} v dalším taktu změnit adresu, data a signály \texttt{data\_we\_o} a \texttt{data\_be\_o}. \textit{Slave} také nastaví signál \texttt{data\_rvalid\_i} na hodnotu log. \texttt{1}, a to jeden nebo více taktů po nastavení signálu \texttt{data\_gnt\_i}. V případě čtení má signál \texttt{data\_rvalid\_i} význam platnosti čtených dat, které jsou vystaveny na \texttt{data\_rdata\_i}. V případě zápisu musí být signál \texttt{data\_rvalid\_i} nastaven také, přestože čtená data nemají žádný význam. \cite{ri5cy}
 
\begin{figure}[t]
  \begin{center}
    \includegraphics[scale=0.79]{obrazky/ri5cy_bus_basic.png}
  \end{center}
  \caption{Základní sekvence přístupu na systémovou sběrnici \cite{ri5cy}}
	\label{fig:ri5cy_bus_basic}
\end{figure}

Na sběrnici může nastat také případ, kdy \hl{\textit{slave}} nemůže odpovědět okamžitě a signál \texttt{data\_gnt\_i} je nastaven až po několika hodinových taktech, jak je vidět na obrázku \ref{fig:ri5cy_bus_slow_res}. V tomto případě musí být pouze zaručeno, aby \textit{master} neměnil signály \texttt{data\_we\_o}, \texttt{data\_be\_o}, a \texttt{data\_wdata\_o}. 

\begin{figure}[H]
  \begin{center}
    \includegraphics[scale=0.99]{obrazky/ri5cy_bus_slow_res.png}
  \end{center}
  \caption{Sekvence přístupu na systémovou sběrnici se zpožděnou odezvou \textit{slave} modulu \cite{ri5cy}}
	\label{fig:ri5cy_bus_slow_res}
\end{figure}

%!!!!!!!TODO: popsat back-to-back a slow response

%TODO: kratke zhodnoceni přistupu přes debug modul a toho co bylo v kapitole vysvetleno