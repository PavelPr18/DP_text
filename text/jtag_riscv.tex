\chapter{Komunikační protokol JTAG}
Tato kapitola popisuje princip přenosu dat na pomocí komunikačního protokolu JTAG a jeho použití pro testování obvodů s procesorem architektury \acs{RISC-V} (\acl{RISC-V}).

\section{Základní popis \acs{JTAG} protokolu}
\acs{JTAG} (\textit{\acl{JTAG}}) je standardizovaná komunikační sběrnice určená pro testování integrovaných obvodů a \acs{DPS} (\textit{\acl{DPS}}).
\acs{JTAG} rozhraní definuje tzv. \acs{TAP} (\textit{\acl{TAP}}), který je skupinou vstupů a výstupů určených k testování. Jelikož \acs{JTAG} je synchronní sběrnice, její rozhraní zahrnuje samostatný vodič pro hodinový signál \texttt{\acs{TCK}} (\textit{\acl{TCK}}). Dalším vodičem je \texttt{\acs{TMS}} (\textit{\acl{TMS}}), kterým je přenášen řídicí signál pro stavový automat. Vodiče zajišťující přenos dat jsou označeny \texttt{\acs{TDI}} (\textit{\acl{TDI}}) a \texttt{\acs{TDO}} (\textit{\acl{TDO}}). Jako volitelný vodič základního rozhraní je možné připojit také \texttt{\acs{TRST}} (\textit{\acl{TRST}}), který provádí reset stavového automatu. Reset se využívá především k inicializaci automatu během připojení napájení. \cite {IEEE_1149-1} \cite{JTAG}      

\section{Řídicí stavový automat - \acs{TAP}}
Standard IEEE 1149.1 definuje stavový automat, který je hlavní součástí \acs{TAP} (\textit{\acl{TAP}}). Stavový diagram stavového automatu je zobrazen na obrázku \ref{fig:tap_controller}. Všechny přechody mezi stavy toho automatu jsou plně synchronní se vstupním hodinovým signálem \texttt{\acs{TCK}} a jsou určeny řídicím signálem \texttt{\acs{TMS}}. Důležitou vlastností tohoto stavového automatu je způsob resetování jeho stavu, tedy návrat do stavu \textit{Test-Logic-Reset}. Reset lze provést pomocí minimálně pěti po sobě jdoucích hodinových taktů, kdy hodnota řídicího signálu \texttt{\acs{TMS}} setrvává v log. \texttt{1}. Tento způsob resetování je možný provést ze kteréhokoliv stavu automatu. Příkladem může být provedení resetu, když se automat nachází ve stavu \textit{Shift-DR}. Automat tedy projde postupně stavy \textit{Exit1-DR}, \textit{Update-DR}, \textit{Select-DR-Scan} a \textit{Select-DR-Scan} až do stavu \textit{Test-Logic-Reset}. V případě kratší cesty do \textit{Test-Logic-Reset} stavu setrvává při hodnotě log. 1 na signálu \texttt{\acs{TMS}} v tomto stavu. Hlavními stavy při kterých probíhá přenos signály \texttt{\acs{TDI}} a \texttt{\acs{TDO}} jsou \textit{Shift-DR} a \textit{Shift-IR}, tedy přenos dat a instrukce. \cite {IEEE_1149-1}

\begin{figure}[H]
  \begin{center}
    \includegraphics[scale=0.35]{obrazky/JTAG_TAP_Controller_State_Diagram.png}
  \end{center}
  \caption{Stavový diagram řídicího stavového automatu JTAGu \cite{JTAG_TAP_diagram}}
	\label{fig:tap_controller}
\end{figure}

\section{Systém pro testování \acs{RISC-V}}		\label{sec:risc-v_dbg}
Pro procesory architektury \acs{RISC-V} je definován systém pro jejich testování a ladění, dle specifikace \textit{RISC-V External Debug Support}. Blokové schéma systému je zobrazené na obrázku \ref{fig:blok_sch_risc-v_dbg}. Nadřazeným systémem pro testování integrovaných obvodů je dle schématu \textbf{debugger}, kterým je zpravidla testovací software v PC. Pro jeho spojení s integrovaným obvodem je třeba zpravidla použít převodník z USB na JTAG protokol. Systém pro přístup k procesorovému jádru se nazývá \ac{DM}. Součástí Debug modulu je blok realizující přístup na systémovou sběrnici (\textit{System Bus Access}), který je popsaný v podkapitole \ref{subsec:dm_sba}. Přechod mezi hodinovými doménami \acs{JTAG} rozhraní a systémové sběrnice v tomto systému realizuje rozhraní \ac{DMI}. \cite{risc-v_dbg}

%Systém umožňuje vykonávat krátký program 

\begin{figure}[!h]
  \begin{center}
    \includegraphics[scale=0.9]{obrazky/risc-v_debug_system_blok_sch.png}
  \end{center}
  \caption{Blokové schéma systému použitého pro testování \acs{RISC-V} procesorů \cite{risc-v_dbg}}
	\label{fig:blok_sch_risc-v_dbg}
\end{figure}

\subsection{Přístup k registrům testovacího systému}		\label{subsec:dm_reg_access}
Modul sloužící k ladění procesorového jádra (\textit{\acl{DM}}) obsahuje sadu 32-bitových registrů. sloužících k . K těmto registrům může debugger přistoupit prostřednictvím \acs{JTAG} protokolu a následně \acs{DMI} rozhraní, jak je možné vidět na blokovém schématu na obrázku \ref{fig:blok_sch_risc-v_dbg}. Rozhraní \acs{DMI} je definováno jako 41-bitový registr, který obsahuje 7-bitovou adresu zvoleného registru, data pro zápis či čtení z registru a dvoubitovou hodnotu \textit{op} (operace). Bitová pole registru jsou znázorněna tabulkou \ref{tab:dmi_access}.   \cite{risc-v_dbg}

Tento registr funguje jako posuvný registr pro komunikaci prostřednictvím \acs{JTAG} protokolu. Pro aktivaci přístupu k registrům popisovaným způsobem je třeba přepnout instrukci zápisem hodnoty 0x11 do instrukčního registru. Poté může debugger zahájit komunikaci přechodem stavového automatu do stavu \textit{Shift-DR} a odesláním požadované hodnoty do posuvného registru. Přičemž zpracování požadavku je vždy zahájeno přechodem do stavu \textit{Update-DR}.

\begin{table}[!h]
  \caption{Formát registru pro přístup k registrům testovacího systému. \cite{risc-v_dbg}}
  \begin{center}
  	\small
	  \begin{tabular}{!{\vrule width 1.2pt}c|c|c!{\vrule width 1.2pt}}
	    \noalign{\hrule height 1.2pt}
				address [7] & data [32] & op [2]\\
			\noalign{\hrule height 1.2pt}
		\end{tabular}
  \end{center}
	\label{tab:dmi_access}
\end{table}

\subsubsection{Význam hodnoty operace pro přístup k registrům testovacího systému}

\begin{table}[!h]
  \caption{Tabulka možných hodnot operace registru \acs{DMI} rozhraní. \cite{risc-v_dbg}}
  \begin{center}
  	\small
	  \begin{tabular}{!{\vrule width 1.2pt}M{1.5cm}|M{2.5cm}|M{9cm}!{\vrule width 1.2pt}}
	    \noalign{\hrule height 1.2pt}
	    op & Význam & Popis\\
	    \noalign{\hrule height 1.2pt}
	    0 & nop & Následující data a adresa jsou ignorovány\\
			\hline	    
			1 & read & Čtení hodnoty registru z adresy\\
			\hline
			2 & write & Zápis přijatých dat na registr\\
			\hline	    
			3 & res & Rezervováno pro budoucí využití\\
			\hline
			\noalign{\hrule height 1.2pt}
		\end{tabular}
  \end{center}
	\label{tab:dmi_access_op}
\end{table}

\begin{table}[!h]
  \caption{Tabulka možných návratových hodnot operace registru \acs{DMI} rozhraní. \cite{risc-v_dbg}}
  \begin{center}
  	\small
	  \begin{tabular}{!{\vrule width 1.2pt}M{1.5cm}|M{2.5cm}|M{9cm}!{\vrule width 1.2pt}}
	    \noalign{\hrule height 1.2pt}
	    op & Význam & Popis\\
	    \noalign{\hrule height 1.2pt}
	    0 & ok & Předchozí operace proběhla v pořádku\\
			\hline	    
			1 & res & Rezervováno pro budoucí využití\\
			\hline
			2 & err & Chyba Debug modulu\\
			\hline	    
			3 & busy & Předchozí operace stále probíhá\\
			\hline
			\noalign{\hrule height 1.2pt}
		\end{tabular}
  \end{center}
	\label{tab:dmi_access_op_response}
\end{table}

\subsection{Přístup na systémovou sběrnici pomocí Debug modulu}		\label{subsec:dm_sba}
Pro přístup	na systémovou sběrnici jsou v debug modulu vyhrazeny registry 			. \cite{risc-v_dbg}

\section{Systémová sběrnice PULP RISC-V}
Jako procesorové jádro je v integrovaném obvodu, pro který je rozšíření JTAG front-endu navrženo, použito procesorové jádro typu \acs{RISC-V}. Konkrétní typ procesorového jádra je použit \acs{PULP} (\acl{PULP}) C32E40P 32-bitový procesor. Předchozí verze jádra je nazvána RI5CY. Rozdíly mezi těmito verzemi nejsou pro popis systémové sběrnice a návrh rozšiřujícího JTAG front-endu podstatné, proto je možné se v této práci odkazovat na starší verzi RI5CY. Systémová sběrnice tohoto procesorového jádra je založena na jednoduchém \textbf{Request-Grant} protokolu. Na sběrnici mohou být obecně připojeni \textit{master} a \textit{slave} moduly. V tomto případě bude uvažován jako master modul navržený modul rozšiřující JTAG front-end a jako slave například paměť.

\subsection{Signály systémové sběrnice}
V tabulce \ref{tab:ri5cy_bus} jsou popsány jednotlivé signály systémové sběrnice procesory PULP RICS-V.

\begin{table}[H]
	\FloatBarrier
  \caption{Tabulka popisu signálů systémové sběrnice PULP RICS-V. \cite{ri5cy}}
  \begin{center}
  	\small
	  \begin{tabular}{!{\vrule width 1.2pt}M{2.2cm}|M{1.5cm}|M{1.8cm}|M{7cm}!{\vrule width 1.2pt}}
	    \noalign{\hrule height 1.2pt}
	    Název signálu & Označení signálu & Počet bitů & Význam signálu\\
	    \noalign{\hrule height 1.2pt}
	    Address & \texttt{addr} & 32 & Adresa registru ke kterému se bude přistupovat\\
			\hline
			Write data & \texttt{wdata} & 32 & Data zapisovaná do registru dle adresy\\
			\hline
			Request & \texttt{req} & 1 & Požadavek přístupu na sběrnici\\
			\hline
			Grant & \texttt{gnt} & 1 & Potvrzení příjmu požadavek přístupu na sběrnici\\
			\hline			
			Read data valid & \texttt{rvalid} & 1 & Potvrzení platnosti čtených dat\\
			\hline
			Read data & \texttt{rdata} & 32 & Data čtená z registru dle adresy\\
			\hline
			Write enable & \texttt{we} & 1 & Určuje zápis(log. 1)/čtení(log. 0)\\
			\hline
			Byte enable & \texttt{be} & 4 & Vybírá bajty, které budou zapsány/čteny\\
			\hline
			\noalign{\hrule height 1.2pt}
		\end{tabular}
  \end{center}
	\label{tab:ri5cy_bus}
\end{table}

\subsection{Průběh přístupu na systémovou sběrnici}
Obecný průběh přístupu na systémovou sběrnici je zobrazen na obrázku \ref{fig:ri5cy_bus_basic}. Pokud chce master modul přistoupit na sběrnici vystaví adresu na \texttt{data\_addr\_o}, nastaví signály \texttt{data\_we\_o}, \texttt{data\_be\_o}, v případě zápisu vystaví také zapisovaná data \texttt{data\_wdata\_o} a nastaví request \texttt{data\_req\_o} na hodnotu log. 1. Jakmile je slave modu připravený obsloužit požadavek nastaví \texttt{data\_gnt\_i} na hodnotu log. 1. Po přijetí gnt může master v dalším taktu adresu, data a signály \texttt{data\_we\_o}, \texttt{data\_be\_o} změnit. Slave také nastaví signál \texttt{data\_rvalid\_i} na hodnotu log. 1 jeden nebo více taktů po nastavení signálu \texttt{data\_gnt\_i}. V případě čtení má signál \texttt{data\_rvalid\_i} význam platnosti čtených dat, které jsou vystaveny na \texttt{data\_rdata\_i}. V případě zápisu musí být signál \texttt{data\_rvalid\_i} nastaven také, i když čtená data nemají žádný význam. \cite{ri5cy}

\begin{figure}[!h]
  \begin{center}
    \includegraphics[scale=0.65]{obrazky/ri5cy_bus_basic.png}
  \end{center}
  \caption{Základní sekvence přístupu na systémovou sběrnici \cite{ri5cy}}
	\label{fig:ri5cy_bus_basic}
\end{figure}

!!!!!!!TODO: popsat back-to-back a slow response