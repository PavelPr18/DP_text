% Soubory musí být v kódování, které je nastaveno v příkazu \usepackage[...]{inputenc}

\documentclass[%        Základní nastavení
%  draft,    				  % Testovací překlad
  12pt,       				% Velikost základního písma je 12 bodů
  a4paper,    				% Formát papíru je A4
  %oneside,      			% Jednostranný tisk
	twoside,      			% Dvoustranný tisk (kapitoly a další důležité části tedy začínají na lichých stranách)
	unicode,						% Záložky a metainformace ve výsledném  PDF budou v kódování unicode
]{report}				    	% Dokument třídy 'zpráva', vhodná pro sazbu závěrečných prací s kapitolami

\usepackage[utf8]		  %	Kódování zdrojových souborů je UTF-8
	{inputenc}					% Balíček pro nastavení kódování zdrojových souborů

\usepackage[				% Nastavení geometrie stránky
	bindingoffset=10mm,		% Hřbet pro vazbu
	hmargin={25mm,25mm},	% Vnitřní a vnější okraj  (jsou nehezky shodné; jakási úroveň estetiky je dosažena pomocí hřbetu)
	vmargin={25mm,34mm},	% Horní a dolní okraj
	footskip=17mm,			  % Velikost zápatí
	nohead,					      % Bez záhlaví
	marginparsep=2mm,		  % Vzdálenost marginálií
	marginparwidth=18mm,	% Šířka marginálií
]{geometry}

\usepackage{sectsty}
	%přetypuje nadpisy všech úrovní na bezpatkové, kromě \chapter, která je přenastavena zvlášť v thesis.sty
	\allsectionsfont{\sffamily}

\usepackage{graphicx} % Balíček 'graphicx' pro vkládání obrázků
											% Nutné pro vložení logotypů školy a fakulty

\usepackage[          % Balíček 'acronym' pro sazby zkratek a symbolů
	nohyperlinks				% Nebudou tvořeny hypertextové odkazy do seznamu zkratek
]{acronym}						
											% Nutné pro použití prostředí 'acronym' balíčku 'thesis'

\usepackage[
	breaklinks=true,		% Hypertextové odkazy mohou obsahovat zalomení řádku
	hypertexnames=false % Názvy hypertext. odkazů budou tvořeny nezávisle na názvech TeXu
]{hyperref}						% Balíček 'hyperref' pro sazbu hypertextových odkazů
											% Nutné pro použití příkazu 'pdfsettings' balíčku 'thesis'

\usepackage{pdfpages} % Balíček umožňující vkládat stránky z PDF souborů
                      % Nutné při vkládání titulních listů a zadání přímo
                      % ve formátu PDF z informačního systému

\usepackage{enumitem} % Balíček pro nastavení mezerování v odrážkách
  \setlist{topsep=0pt,partopsep=0pt,noitemsep} % konkrétní nastavení

\usepackage{cmap} 		% Balíček cmap zajišťuje, že PDF vytvořené `pdflatexem' je
											% plně "prohledávatelné" a "kopírovatelné"

%\usepackage{upgreek}	% Balíček pro sazbu stojatých řeckých písmem
											%% např. stojaté pí: \uppi
											%% např. stojaté mí: \upmu (použitelné třeba v mikrometrech)
											%% pozor, grafická nekompatibilita s fonty typu Computer Modern!
                      
%\usepackage{amsmath} %balíček pro sabu náročnější matematiky                 

\usepackage{dirtree}	% sazba adresářové struktury
                      % vhodné pro prezentaci obsahu elektronické přílohy (např. CD)

\usepackage[formats]{listings}	% Balíček pro sazbu zdrojových textů
\lstset{              % nastavení
%	Definice jazyka použitého ve výpisech
%    language=[LaTeX]{TeX},	% LaTeX
%	language={Matlab},		% Matlab
	language={C},           % jazyk C
    basicstyle=\ttfamily,	% definice základního stylu písma
    tabsize=2,			% definice velikosti tabulátoru
    inputencoding=utf8,         % pro soubory uložené v kódování UTF-8
		columns=fixed,  %fixed nebo flexible,
		fontadjust=true %licovani sloupcu
    extendedchars=true,
    literate=%  definice symbolů s diakritikou
    {á}{{\'a}}1
    {č}{{\v{c}}}1
    {ď}{{\v{d}}}1
    {é}{{\'e}}1
    {ě}{{\v{e}}}1
    {í}{{\'i}}1
    {ň}{{\v{n}}}1
    {ó}{{\'o}}1
    {ř}{{\v{r}}}1
    {š}{{\v{s}}}1
    {ť}{{\v{t}}}1
    {ú}{{\'u}}1
    {ů}{{\r{u}}}1
    {ý}{{\'y}}1
    {ž}{{\v{z}}}1
    {Á}{{\'A}}1
    {Č}{{\v{C}}}1
    {Ď}{{\v{D}}}1
    {É}{{\'E}}1
    {Ě}{{\v{E}}}1
    {Í}{{\'I}}1
    {Ň}{{\v{N}}}1
    {Ó}{{\'O}}1
    {Ř}{{\v{R}}}1
    {Š}{{\v{S}}}1
    {Ť}{{\v{T}}}1
    {Ú}{{\'U}}1
    {Ů}{{\r{U}}}1
    {Ý}{{\'Y}}1
    {Ž}{{\v{Z}}}1
}

%%%%%%%%%%%%%%%%%%%%%%%%%%%%%%%%%%%%%%%%%%%%%%%%%%%%%%%%%%%%%%%%%
%%%%%%      Definice informací o dokumentu             %%%%%%%%%%
%%%%%%%%%%%%%%%%%%%%%%%%%%%%%%%%%%%%%%%%%%%%%%%%%%%%%%%%%%%%%%%%%

% V tomto souboru se nastavují téměř veškeré informace, proměnné mezi studenty:
% jméno, název práce, pohlaví atd.
% Tento soubor je SDÍLENÝ mezi textem práce a prezentací k obhajobě -- netřeba něco nastavovat na dvou místech.

\usepackage[
%%% Z následujících voleb jazyka lze použít pouze jednu
  czech-english,		% originální jazyk je čeština, překlad je anglicky (výchozí)
  %english-czech,	% originální jazyk je angličtina, překlad je česky
  %slovak-english,	% originální jazyk je slovenština, překlad je anglicky
  %english-slovak,	% originální jazyk je angličtina, překlad je slovensky
%
%%% Z následujících voleb typu práce lze použít pouze jednu
  semestral,		  % semestrální práce (výchozí)
  %bachelor,			%	bakalářská práce
  %master,			  % diplomová práce
  %treatise,			% pojednání o disertační práci
  %doctoral,			% disertační práce
%
%%% Z následujících voleb zarovnání objektů lze použít pouze jednu
%  left,				  % rovnice a popisky plovoucích objektů budou zarovnány vlevo
	center,			    % rovnice a popisky plovoucích objektů budou zarovnány na střed (vychozi)
%
]{thesis}   % Balíček pro sazbu studentských prací


%%% Jméno a příjmení autora ve tvaru
%  [tituly před jménem]{Křestní}{Příjmení}[tituly za jménem]
% Pokud osoba nemá titul před/za jménem, smažte celý řetězec '[...]'
\author[Bc.]{Pavel}{Prášil}

%%% Identifikační číslo autora (VUT ID)
\butid{216847}

%%% Pohlaví autora/autorky
% (nepoužije se ve variantě english-czech ani english-slovak)
% Číselná hodnota: 1...žena, 0...muž
\gender{0}

%%% Jméno a příjmení vedoucího/školitele včetně titulů
%  [tituly před jménem]{Křestní}{Příjmení}[tituly za jménem]
% Pokud osoba nemá titul před/za jménem, smažte celý řetězec '[...]'
\advisor[Ing.]{Petr}{Petyovský}[Ph.D.]

%%% Jméno a příjmení oponenta včetně titulů
%  [tituly před jménem]{Křestní}{Příjmení}[tituly za jménem]
% Pokud osoba nemá titul před/za jménem, smažte celý řetězec '[...]'
% Nastavení oponenta se uplatní pouze v prezentaci k obhajobě;
% v případě, že nechcete, aby se na titulním snímku prezentace zobrazoval oponent, pouze příkaz zakomentujte;
% u obhajoby semestrální práce se oponent nezobrazuje (jelikož neexistuje)
% U dizertační práce jsou typicky dva až tři oponenti. Pokud je chcete mít na titulním slajdu, prosím ručně odkomentujte a upravte jejich jména v definici "VUT title page" v souboru thesis.sty.
\opponent[doc.\ Mgr.]{Křestní}{Příjmení}[Ph.D.]

%%% Název práce
%  Parametr ve složených závorkách {} je název v originálním jazyce,
%  parametr v hranatých závorkách [] je překlad (podle toho jaký je originální jazyk).
%  V případě, že název Vaší práce je dlouhý a nevleze se celý do zápatí prezentace, použijte příkaz
%  \def\insertshorttitle{Zkác.\ náz.\ práce}
%  kde jako parametr vyplníte zkrácený název. Pokud nechcete zkracovat název, budete muset předefinovat,
%  jak se vytváří patička slidu. Viz odkaz: https://bit.ly/3EJTp5A
\title[Implementation of system for IC testing via JTAG interface]{Implementace systému pro testování integrovaných obvodů pomocí JTAG rozhraní}

%%% Označení oboru studia
%  Parametr ve složených závorkách {} je název oboru v originálním jazyce,
%  parametr v hranatých závorkách [] je překlad
\specialization[Cybernetics, Control and Measurements]{Kybernetika, automatizace a měření}

%%% Označení ústavu
%  Parametr ve složených závorkách {} je název ústavu v originálním jazyce,
%  parametr v hranatých závorkách [] je překlad
\department[Department of Control and Instrumentation]{Ústav automatizace a měřicí techniky}
%\department[Department of Biomedical Engineering]{Ústav biomedicínského inženýrství}
%\department[Department of Electrical Power Engineering]{Ústav elektroenergetiky}
%\department[Department of Electrical and Electronic Technology]{Ústav elektrotechnologie}
%\department[Department of Physics]{Ústav fyziky}
%\department[Department of Foreign Languages]{Ústav jazyků}
%\department[Department of Mathematics]{Ústav matematiky}
%\department[Department of Microelectronics]{Ústav mikroelektroniky}
%\department[Department of Radio Electronics]{Ústav radioelektroniky}
%\department[Department of Theoretical and Experimental Electrical Engineering]{Ústav teoretické a experimentální elektrotechniky}
%\department[Department of Telecommunications]{Ústav telekomunikací}
%\department[Department of Power Electrical and Electronic Engineering]{Ústav výkonové elektrotechniky a elektroniky}

%%% Označení fakulty
%  Parametr ve složených závorkách {} je název fakulty v originálním jazyce,
%  parametr v hranatých závorkách [] je překlad
%\faculty[Faculty of Architecture]{Fakulta architektury}
\faculty[Faculty of Electrical Engineering and~Communication]{Fakulta elektrotechniky a~komunikačních technologií}
%\faculty[Faculty of Chemistry]{Fakulta chemická}
%\faculty[Faculty of Information Technology]{Fakulta informačních technologií}
%\faculty[Faculty of Business and Management]{Fakulta podnikatelská}
%\faculty[Faculty of Civil Engineering]{Fakulta stavební}
%\faculty[Faculty of Mechanical Engineering]{Fakulta strojního inženýrství}
%\faculty[Faculty of Fine Arts]{Fakulta výtvarných umění}
%
%Nastavení logotypu (v hranatych zavorkach zkracene logo, ve slozenych plne):
\facultylogo[logo/FEKT_zkratka_barevne_PANTONE_CZ]{logo/UAMT_color_PANTONE_CZ}

%%% Rok odevzdání práce
\graduateyear{2024}
%%% Akademický rok odevzdání práce
\academicyear{2023/24}

%%% Datum obhajoby (uplatní se pouze v prezentaci k obhajobě)
\date{11.\,11.\,1980} 

%%% Místo obhajoby
% Na titulních stránkách bude automaticky vysázeno VELKÝMI písmeny (pokud tyto stránky sází šablona)
\city{Brno}

%%% Abstrakt
\abstract[%
This semestral thesis deals with testing IC's containing \acs{RISC-V} processor core by \acs{JTAG} protocol. This thesis objective is to design an IP block for 2-wire \acs{JTAG} protocol support xand design of extended protocol for \acs{RISC-V} processor system bus access. Designed IP block will be used for the IC testing via 2-wire \acs{JTAG} interface, for the purpose of pin count reduction. The extended protocol will be used for IC testing time optimization. The thesis includes descriptions of the \acs{RISC-V} testing system, designed IP block for 2-wire \acs{JTAG} protocol handling and description of its implementation, and design of optimized protocol. 
]{%
Tato semestrální práce se zabývá testováním integrovaných obvodů s procesorem \acs{RISC-V} pomocí \acs{JTAG} protokolu. Cílem práce je návrh modulu pro podporu dvouvodičové varianty \acs{JTAG} protokolu a návrh rozšiřujícího protokolu pro přístup na systémovou sběrnici \acs{RISC-V} procesoru pomocí \acs{JTAG} rozhraní. Navržený modul bude použit pro testování integrovaného obvodu pomocí dvouvodičového \acs{JTAG} rozhraní, za účelem redukce počtu pinů. Rozšiřující protokol bude sloužit pro časovou optimalizaci testování integrovaného obvodu. Práce obsahuje popisy systému pro testování \acs{RISC-V} procesorů, navrženého modulu pro dvouvodičový \acs{JTAG} protokol, včetně způsobu jeho implementace, a návrhu optimalizovaného protokolu.
}

%%% Klíčová slova
\keywrds[%
JTAG, FPGA, VHDL, RISC-V, IC testing
]{%
JTAG, FPGA, VHDL, RISC-V, testování integrovaných obvodů
}

%%% Poděkování
\acknowledgement{%
Rád bych poděkoval vedoucímu diplomové práce
panu Ing.~Petru Petyovskému, Ph.D.\ za odborné vedení,
konzultace, trpělivost a~podnětné návrhy k~práci.
}%  % do tohoto souboru doplňte údaje o sobě, druhu práce, názvu...

%%%%%%%%%%%%%%%%%%%%%%%%%%%%%%%%%%%%%%%%%%%%%%%%%%%%%%%%%%%%%%%%%%%%%%%%

%%%%%%%%%%%%%%%%%%%%%%%%%%%%%%%%%%%%%%%%%%%%%%%%%%%%%%%%%%%%%%%%%%%%%%%%
%%%%%%     Nastavení polí ve Vlastnostech dokumentu PDF      %%%%%%%%%%%
%%%%%%%%%%%%%%%%%%%%%%%%%%%%%%%%%%%%%%%%%%%%%%%%%%%%%%%%%%%%%%%%%%%%%%%%
%% Při načteném balíčku 'hyperref' lze použít příkaz '\pdfsettings':
\pdfsettings
%  Nastavení polí je možné provést také ručně příkazem:
%\hypersetup{
%  pdftitle={Název studentské práce},    	% Pole 'Document Title'
%  pdfauthor={Autor studenstké práce},   	% Pole 'Author'
%  pdfsubject={Typ práce}, 						  	% Pole 'Subject'
%  pdfkeywords={Klíčová slova}           	% Pole 'Keywords'
%}
%%%%%%%%%%%%%%%%%%%%%%%%%%%%%%%%%%%%%%%%%%%%%%%%%%%%%%%%%%%%%%%%%%%%%%%

\pdfmapfile{=vafle.map}

%%%%%%%%%%%%%%%%%%%%%%%%%%%%%%%%%%%%%%%%%%%%%%%%%%%%%%%%%%%%%%%%%%%%%%%
%%%%%%%%%%%       Začátek dokumentu               %%%%%%%%%%%%%%%%%%%%%
%%%%%%%%%%%%%%%%%%%%%%%%%%%%%%%%%%%%%%%%%%%%%%%%%%%%%%%%%%%%%%%%%%%%%%%
\begin{document}
\pagestyle{empty} %vypnutí číslování stránek

%%% Vložení desek -- od září 2021 na žádost fakulty nepoužíváno
%\includepdf[pages=1]%  buďto generovaných informačním systémem
  %{pdf/student-desky}% název souboru nesmí obsahovat mezery!
%%% NEBO vytvoření desek z balíčku
%%\makecover
%%%
%\oddpage % při dvojstranném tisku přidá prázdnou stránku
%% kazdopádně ale:
%\setcounter{page}{1} %resetovaní čítače stránek -- desky do číslování nezahrnujeme

%% Vložení titulního listu
\includepdf[pages=1]%    buďto generovaného informačním systémem
  {pdf/student-titulka}% název souboru nesmí obsahovat mezery!
%% NEBO vytvoření titulní stránky z balíčku
%\maketitle
%%
\oddpage  % při dvojstranném tisku se přidá prázdná stránka
   
%% Vložení zadání
\includepdf[pages=1]%   buďto generovaného informačním systémem
  {pdf/student-zadani}% název souboru nesmí obsahovat mezery!
%% NEBO lze vytvořit prázdný list příkazem ze šablony
%\patternpage{}%
%	{\sffamily\Huge\centering ZDE VLOŽIT LIST ZADÁNÍ}%
%	{\sffamily\centering Z~důvodu správného číslování stránek}
%%
\oddpage% při dvojstranném tisku se přidá prázdná stránka

%% Vysázení stránky s abstraktem
\makeabstract

% Vysázení stránky s rozšířeným abstraktem
% (pokud píšete práci v češtině či slovenštině, vložení rozšířeného abstraktu zrušte;
%  pro semestrální projekt také není potřeba rozšířený abstrakt uvádět)
% Vysázení stránky s rozšířeným abstraktem
% (týká se pouze bc. a dp. prací psaných v angličtině, viz Směrnice rektora 72/2017)
\cleardoublepage
\noindent
{\large\sffamily\bfseries\MakeUppercase{Rozšířený abstrakt}}
\\
Výtah ze směrnice rektora 72/2017:\\
\emph{Bakalářská a diplomová práce předložená v angličtině musí obsahovat rozšířený abstrakt v češtině
nebo slovenštině (čl. 15). To se netýká studentů, kteří studují studijní program akreditovaný v
angličtině.}
(čl. 3, par. 7)\\
\emph{Nebude-li vnitřní normou stanoveno jinak, doporučuje se rozšířený abstrakt o rozsahu přibližně 3
normostrany, který bude obsahovat úvod, popis řešení a shrnutí a~zhodnocení výsledků.}
(čl. 15, par. 5)

%%% Vysázení citace práce
\makecitation

%%% Vysázení prohlášení o samostatnosti
\makedeclaration

%%% Vysázení poděkování
\makeacknowledgement

%%% Vysázení obsahu
\tableofcontents

%%% Vysázení seznamu obrázků
% (vynechejte, pokud máte dva nebo méně obrázků)
\listoffigures

%%% Vysázení seznamu tabulek
% (vynechejte, pokud máte dvě nebo méně tabulek)
\listoftables

%%% Vysázení seznamu výpisů kódu
% (vynechejte, pokud máte dva nebo méně výpisů)
\lstlistoflistingss

\cleardoublepage\pagestyle{plain}   % zapnutí číslování stránek

%Pro vkládání kapitol i příloh používejte raději \include než \input
%%% Vložení souboru 'text/uvod.tex' s úvodem
\chapter*{Úvod}
\phantomsection
\addcontentsline{toc}{chapter}{Úvod}

Tématem této semestrální práce je návrh modulu pro dvouvodičový \acs{JTAG} protokol a rozšiřujícího protokolu pro přístup na systémovou sběrnici \acs{RISC-V} procesoru.

Cílem semestrální práce je navrhnout modul podporující komunikaci pomocí dvouvodičové varianty \acs{JTAG} protokolu s možností přepínat mezi touto a čtyřvodičovou variantou dle standardu IEEE 1149.7. Dalším cílem je návrh časově optimalizovaný protokol využívající \acs{JTAG} rozhraní pro testovaní procesorů \acs{RISC-V}.

V úvodní části práce je obecně přestaveno testovací rozhraní \acs{JTAG} a základní princip fungování řídicího stavového automatu dle příslušného standardu IEEE 1149.1. Dále je uveden testovací systém pro procesory \acs{RISC-V}, který je použitý pro přístup na systémovou sběrnici procesoru. Princip fungování systémové sběrnice použitého procesorového jádra je také představen v rámci úvodní kapitoly.

Druhá kapitola se věnuje principu komunikace dvouvodičovou variantou \acs{JTAG} protokolu, způsobu aktivace dvouvodičového režimu a také možnosti přepnutí na původní čtyřvodičovou variantu. Dále je v této kapitole popsán způsob návrhu a implementace modulu zajišťujícího uvedenou funkcionalitu.

V poslední části práce je popsán navržený protokol pro přístup na systémovou sběrnici procesoru. Tento protokol bude využitý jako vedlejší možnost přístupu na systémovou sběrnici, která je časově efektivnější oproti stávajícímu způsobu využívajícího systém pro testování \acs{RISC-V} procesorů. Uveden je také způsob zkrácení přenášené adresy registru, ke kterému je přistupováno a možnosti délky přenášených dat. Protokol je navržený ve dvou variantách, které jsou popsány detailněji v jednotlivých částech této kapitoly a jsou zhodnoceny jejich vlastnosti.



%%% Vložení souboru 'text/cile.tex' s úvodem
\chapter*{Cíle práce}
\phantomsection
\addcontentsline{toc}{chapter}{Cíle práce}

Konkrétní specifikace cílů, které má autor v~práci vyřešit.
Tato kapitola je \emph{volitelná} -- pokud váš studijní program nevyžaduje zvláštní kapitolu s cíli,
cíle specifikujte v~rámci Úvodu.

%%% Vložení souboru 'text/reseni' s popisem řešení práce
% (rozdělte na více souborů či kapitol, pokud je vhodné)
\chapter{Teoretická část studentské práce}

Teoretické zázemí studentské práce vhodně rozdělené do částí.

(Struktura navržená v~této šabloně je nejhrubší možná, po konzultaci s~vedoucím je vhodné zvolit přiléhavější.)


%%% Vložení souboru 'text/vysledky' s popisem vysledků práce
% (rozdělte na více souborů či kapitol, pokud je vhodné)
\chapter{Výsledky studentské práce}

Praktická část a výsledky studentské práce vhodně rozdělené do částí.

\section{Programové řešení}
Lorem ipsum dolor sit amet, consectetuer adipiscing elit. Nulla pulvinar eleifend sem. Integer in sapien. Etiam sapien elit, consequat eget, tristique non, venenatis quis, ante. In laoreet, magna id viverra tincidunt, sem odio bibendum justo, vel imperdiet sapien wisi sed libero. Phasellus enim erat, vestibulum vel, aliquam a, posuere eu, velit. Aliquam erat volutpat. Nullam faucibus mi quis velit \cite{sr72/2017}.

\section{Výsledky měření}
Fusce tellus odio, dapibus id fermentum quis, suscipit id erat. Fusce tellus. Morbi scelerisque luctus velit. In laoreet, magna id viverra tincidunt, sem odio bibendum justo, vel imperdiet sapien wisi sed libero. Quisque porta. Fusce suscipit libero eget elit. Nulla non lectus sed nisl molestie malesuada. Phasellus faucibus molestie nisl. Integer vulputate sem a nibh rutrum consequat. Proin mattis lacinia justo. Phasellus et lorem id felis nonummy placerat. Etiam ligula pede, sagittis quis, interdum ultricies, scelerisque eu. Cras elementum. Aenean placerat. Donec ipsum massa, ullamcorper in, auctor et, scelerisque sed, est. Aliquam ante. Integer imperdiet lectus quis justo. Vivamus ac leo pretium faucibus. Nullam faucibus mi quis velit.

\subsection{Etiam quis quam}
Neque porro quisquam est, qui dolorem ipsum quia dolor sit amet, consectetur, adipisci velit, sed quia non numquam eius modi tempora incidunt ut labore et dolore magnam aliquam quaerat voluptatem. Aliquam erat volutpat. Lorem ipsum dolor sit amet, consectetuer adipiscing elit \cite{sr72/2017,pravidla}. Nunc auctor. Neque porro quisquam est, qui dolorem ipsum quia dolor sit amet, consectetur, adipisci velit, sed quia non numquam eius modi tempora incidunt ut labore et dolore magnam aliquam quaerat voluptatem. Maecenas lorem. Maecenas libero. In laoreet, magna id viverra tincidunt, sem odio bibendum justo, vel imperdiet sapien wisi sed libero. Nullam rhoncus aliquam metus.

\subsubsection{Integer rutrum orci vestibulum}
Integer rutrum, orci vestibulum ullamcorper ultricies, lacus quam ultricies odio, vitae placerat pede sem sit amet enim. Ut enim ad minim veniam, quis nostrud exercitation ullamco laboris nisi ut aliquip ex ea commodo consequat. Fusce tellus odio, dapibus id fermentum quis, suscipit id erat. Nullam eget nisl. Nunc auctor. Etiam dui sem, fermentum vitae, sagittis id, malesuada in, quam. Fusce dui leo, imperdiet in, aliquam sit amet, feugiat eu, orci. Curabitur vitae diam non enim vestibulum interdum. Aliquam erat volutpat. Pellentesque sapien. Phasellus enim erat, vestibulum vel, aliquam a, posuere eu, velit.

\subsubsection{Eger rutrum orci westibulum}
Fusce dui leo, imperdiet in, aliquam sit amet, feugiat eu, orci. Maecenas aliquet accumsan leo. Aliquam ornare wisi eu metus. Cum sociis natoque penatibus et magnis dis parturient montes, nascetur ridiculus mus. Aliquam erat volutpat. Donec iaculis gravida nulla. Sed elit dui, pellentesque a, faucibus vel, interdum nec, diam. Temporibus autem quibusdam et aut officiis debitis aut rerum necessitatibus saepe eveniet ut et voluptates repudiandae sint et molestiae non recusandae. Nulla non arcu lacinia neque faucibus fringilla. Phasellus enim erat, vestibulum vel, aliquam a, posuere eu, velit. Praesent vitae arcu tempor neque lacinia pretium
\cite{Walter1999,Svacina1999IEEE,RajmicSysel2002}.

Aliquam erat volutpat. Quisque porta. Integer imperdiet lectus quis justo. Nullam justo enim, consectetuer nec, ullamcorper ac, vestibulum in, elit. Nullam faucibus mi quis velit. Fusce tellus. Fusce consectetuer risus a nunc. Cras pede libero, dapibus nec, pretium sit amet, tempor quis. Morbi imperdiet, mauris ac auctor dictum, nisl ligula egestas nulla, et sollicitudin sem purus in lacus
\cite{CSN_ISO_690-2022,CSN_ISO_7144-1997,CSN_ISO_31-11}.
Mauris elementum mauris vitae tortor. Neque porro quisquam est, qui dolorem ipsum quia dolor sit amet, consectetur, adipisci velit, sed quia non numquam eius modi tempora incidunt ut labore et dolore magnam aliquam quaerat voluptatem. Quisque porta. Integer vulputate sem a nibh rutrum consequat. Nulla pulvinar eleifend sem. Praesent id justo in neque elementum ultrices \cite{Farkasova23:CSNISO6902022komentar}.

Fusce suscipit libero eget elit. Integer vulputate sem a nibh rutrum consequat. Aliquam erat volutpat. Etiam neque. Nulla turpis magna, cursus sit amet, suscipit a, interdum id, felis. Nullam rhoncus aliquam metus. Etiam dui sem, fermentum vitae, sagittis id, malesuada in, quam. Nunc auctor. Nunc dapibus tortor vel mi dapibus sollicitudin. Praesent in mauris eu tortor porttitor accumsan. Nulla non arcu lacinia neque faucibus fringilla. Nullam lectus justo, vulputate eget mollis sed, tempor sed magna. Maecenas lorem. Aenean placerat. Donec vitae arcu. Maecenas lorem. Donec iaculis gravida nulla. Nulla non lectus sed nisl molestie malesuada.

Duis pulvinar. Nulla est. Duis condimentum augue id magna semper rutrum. Integer pellentesque quam vel velit. Aliquam ante. Nulla quis diam. Proin mattis lacinia justo. Aenean fermentum risus id tortor. Nunc auctor. Nullam justo enim, consectetuer nec, ullamcorper ac, vestibulum in, elit. In dapibus augue non sapien. Etiam bibendum elit eget erat. In sem justo, commodo ut, suscipit at, pharetra vitae, orci. Maecenas libero.

Nulla non lectus sed nisl molestie malesuada. Donec vitae arcu. Aenean fermentum risus id tortor. Praesent in mauris eu tortor porttitor accumsan. Nulla pulvinar eleifend sem. Duis viverra diam non justo. Integer imperdiet lectus quis justo. Pellentesque habitant morbi tristique senectus et netus et malesuada fames ac turpis egestas. In rutrum. Excepteur sint occaecat cupidatat non proident, sunt in culpa qui officia deserunt mollit anim id est laborum. Nulla non lectus sed nisl molestie malesuada. Aliquam erat volutpat. Mauris tincidunt sem sed arcu. Duis bibendum, lectus ut viverra rhoncus, dolor nunc faucibus libero, eget facilisis enim ipsum id lacus. Fusce tellus odio, dapibus id fermentum quis, suscipit id erat. In enim a arcu imperdiet malesuada. Nulla non lectus sed nisl molestie malesuada. Proin mattis lacinia justo.

Aliquam in lorem sit amet leo accumsan lacinia. Cum sociis natoque penatibus et magnis dis parturient montes, nascetur ridiculus mus. Duis sapien nunc, commodo et, interdum suscipit, sollicitudin et, dolor. Suspendisse sagittis ultrices augue. Nullam lectus justo, vulputate eget mollis sed, tempor sed magna. In convallis. Praesent id justo in neque elementum ultrices. Neque porro quisquam est, qui dolorem ipsum quia dolor sit amet, consectetur, adipisci velit, sed quia non numquam eius modi tempora incidunt ut labore et dolore magnam aliquam quaerat voluptatem.

Pellentesque pretium lectus id turpis. Nemo enim ipsam voluptatem quia voluptas sit aspernatur aut odit aut fugit, sed quia consequuntur magni dolores eos qui ratione voluptatem sequi nesciunt. Curabitur ligula sapien, pulvinar a vestibulum quis, facilisis vel sapien. Praesent dapibus. Sed elit dui, pellentesque a, faucibus vel, interdum nec, diam. Duis viverra diam non justo. Duis ante orci, molestie vitae vehicula venenatis, tincidunt ac pede. Phasellus rhoncus. Maecenas fermentum, sem in pharetra pellentesque, velit turpis volutpat ante, in pharetra metus odio a lectus. Proin pede metus, vulputate nec, fermentum fringilla, vehicula vitae, justo. Fusce aliquam vestibulum ipsum. Nullam at arcu a est sollicitudin euismod.

%Aliquam ante. Phasellus faucibus molestie nisl. Etiam ligula pede, sagittis quis, interdum ultricies, scelerisque eu. Morbi leo mi, nonummy eget tristique non, rhoncus non leo. Cum sociis natoque penatibus et magnis dis parturient montes, nascetur ridiculus mus. Morbi scelerisque luctus velit. Curabitur bibendum justo non orci. Donec quis nibh at felis congue commodo. Nullam faucibus mi quis velit. Aenean id metus id velit ullamcorper pulvinar. Pellentesque sapien. Fusce nibh. Vestibulum fermentum tortor id mi. Nullam eget nisl. Praesent vitae arcu tempor neque lacinia pretium. Proin in tellus sit amet nibh dignissim sagittis. Donec quis nibh at felis congue commodo.
%
%Nam quis nulla. Proin in tellus sit amet nibh dignissim sagittis. Nullam dapibus fermentum ipsum. Curabitur ligula sapien, pulvinar a vestibulum quis, facilisis vel sapien. Nam libero tempore, cum soluta nobis est eligendi optio cumque nihil impedit quo minus id quod maxime placeat facere possimus, omnis voluptas assumenda est, omnis dolor repellendus. Vivamus ac leo pretium faucibus. Nunc tincidunt ante vitae massa. Maecenas sollicitudin. Ut tempus purus at lorem. Nullam lectus justo, vulputate eget mollis sed, tempor sed magna. Fusce consectetuer risus a nunc. Etiam quis quam.
%
%Donec quis nibh at felis congue commodo. Sed vel lectus. Donec odio tempus molestie, porttitor ut, iaculis quis, sem. Nullam feugiat, turpis at pulvinar vulputate, erat libero tristique tellus, nec bibendum odio risus sit amet ante. Sed elit dui, pellentesque a, faucibus vel, interdum nec, diam. Cras elementum. Sed vel lectus. Donec odio tempus molestie, porttitor ut, iaculis quis, sem. Etiam neque. Integer tempor. Vivamus porttitor turpis ac leo. Nulla non arcu lacinia neque faucibus fringilla.
%
%Etiam posuere lacus quis dolor. Nemo enim ipsam voluptatem quia voluptas sit aspernatur aut odit aut fugit, sed quia consequuntur magni dolores eos qui ratione voluptatem sequi nesciunt. Nullam faucibus mi quis velit. Cum sociis natoque penatibus et magnis dis parturient montes, nascetur ridiculus mus. Phasellus faucibus molestie nisl. Maecenas ipsum velit, consectetuer eu lobortis ut, dictum at dui. Maecenas aliquet accumsan leo. Pellentesque ipsum. Donec vitae arcu. Suspendisse nisl. Morbi imperdiet, mauris ac auctor dictum, nisl ligula egestas nulla, et sollicitudin sem purus in lacus. Pellentesque ipsum. Ut enim ad minima veniam, quis nostrum exercitationem ullam corporis suscipit laboriosam, nisi ut aliquid ex ea commodi consequatur? Nam libero tempore, cum soluta nobis est eligendi optio cumque nihil impedit quo minus id quod maxime placeat facere possimus, omnis voluptas assumenda est, omnis dolor repellendus.


%%% Vložení souboru 'text/zaver' se závěrem
\chapter*{Závěr}
\phantomsection
\addcontentsline{toc}{chapter}{Závěr}

Cílem práce bylo navrhnout modul podporující komunikaci pomocí dvouvodičového \acs{JTAG} protokolu a rozšiřujícího protokolu pro přístup na systémovou sběrnici \acs{RISC-V} procesoru.

V úvodní kapitole je uveden princip fungování \acs{JTAG} rozhraní, systém pro testování \acs{RISC-V} procesorů, který je používaný v projektu pro testování daného procesorového jádra, a princip fungování systémové sběrnice tohoto procesoru. Princip fungování testovacího systému v tomto uspořádání a systémové sběrnice bylo třeba nastudovat, aby bylo možné navrhnout sekundární, časově optimalizovaný protokol pro přístup na systémovou sběrnici, který je popsaný v kapitole \ref{jtag_ap}.

Druhá kapitola popisuje vybraný princip komunikace dvouvodičovou variantou \acs{JTAG} protokolu dle standardu IEEE 1149.7 a způsob implementace modulu podporující tento způsob komunikace. Pro zvolení vhodné varianty dvouvodičové \acs{JTAG} komunikace bylo třeba nastudovat příslušné části standardu IEEE 1149.7, což bylo poměrně náročné, protože tento standard je velmi obsáhlý.

V navazující diplomové práci bude popsán také způsob implementace hardwarového řešení podporujícího navržený optimalizovaný protokol. Dále bude třeba uvést výsledky funkční simulace některých částí navržených modulů a také ověřit správnou funkci v obvodu \acs{FPGA}.

Pro otestování modulu pro podporu komunikace navrženým protokolem v obvodu \acs{FPGA}, bude zapotřebí rozšířit softwarové aplikační prostředí debuggeru o funkce, které budou odesílat a vyhodnocovat data přijatá přes \acs{JTAG} rozhraní dle protokolů navržených v podkapitole \ref{sec:protokoly}. Tato část práce bude z plánovaných činností nejnáročnější z důvodu menších zkušeností s programováním softwaru. Realizace by přesto neměla trvat déle než dva měsíce. Po dokončení realizace softwarových funkcí bude třeba vymyslet vhodný test pro otestování funkčnosti a porovnání časové efektivity nově navrženého protokolu s původním přístupem popsaným v podkapitole \ref{subsec:dm_sba}.

V rámci navrženého modulu, pro podporu nově vytvořeného protokolu, by bylo dobré implementovat také podporu používání obou variant protokolů s automatickou inkrementací adresy. Tato funkcionalita může být využita například pro zápis nebo čtení rozsáhlejších bloků dat z paměti. Tyto varianty protokolů jsou již navrženy a jejich doplnění do stávajícího návrhu by nemělo být příliš časově náročné.

%%% Vložení souboru 'text/literatura' se seznamem zdrojů
% Pro sazbu seznamu literatury použijte jednu z následujících možností

%%%%%%%%%%%%%%%%%%%%%%%%%%%%%%%%%%%%%%%%%%%%%%%%%%%%%%%%%%%%%%%%%%%%%%%%%
%1) Seznam citací definovaný přímo pomocí prostředí literatura / thebibliography

\begin{thebibliography}{99}

\bibitem{IEEE_1149-1}
IEEE Standard for Test Access Port and Boundary-Scan Architecture. Online. In: . S.~1-422. ISBN 978-0-7381-8263-6. Dostupné z: \url{https://doi.org/10.1109/IEEESTD.2013.6515989}. [cit. 2023-11-02].

\bibitem{IEEE_1149-7}
IEEE Standard for Reduced-Pin and Enhanced-Functionality Test Access Port and Boundary-Scan Architecture. Online. In: . Dostupné z: \url{https://doi.org/10.1109/IEEESTD.2022.9919140}. [cit. 2023-11-03].

\bibitem{JTAG_TAP_diagram}
\textit{Joint Test Action Group}. Online. In: Wikipedia: the free encyclopedia. San Francisco (CA): Wikimedia Foundation, 2001-. Dostupné z: \url{https://de.m.wikipedia.org/wiki/Joint\_Test\_Action\_Group}. [cit. 2023-10-31].

\bibitem{JTAG}
\textit{JTAG}. Online. In: Wikipedia: the free encyclopedia. San Francisco (CA): Wikimedia Foundation, 2001-. Dostupné z: \url{https://en.wikipedia.org/wiki/JTAG}. [cit. 2023-11-02].

\bibitem{risc-v_dbg}
\textit{RISC-V External Debug Support Version 0.13.2}. Online. San Mateo, California, U.S.: SiFive, 2019. Dostupné z: \url{https://riscv.org/wp-content/uploads/2019/03/riscv-debug-release.pdf}. [cit. 2023-08-01].

\bibitem{ri5cy}
\textit{RI5CY: User Manual}. Online. 2019. Dostupné z: \url{https://www.pulp-platform.org/docs/ri5cy\_user\_manual.pdf}. [cit. 2023-11-05].

\end{thebibliography}


%%%%%%%%%%%%%%%%%%%%%%%%%%%%%%%%%%%%%%%%%%%%%%%%%%%%%%%%%%%%%%%%%%%%%%%%%
%%2) Seznam citací pomocí BibTeXu
%% Při použití je nutné v TeXnicCenter ve výstupním profilu aktivovat spouštění BibTeXu po překladu.
%% Definice stylu seznamu
%\bibliographystyle{unsrturl}
%% Pro českou sazbu lze použít styl czechiso.bst ze stránek
%% http://www.fit.vutbr.cz/~martinek/latex/czechiso.tar.gz
%%\bibliographystyle{czechiso}
%% Vložení souboru se seznamem citací
%\bibliography{text/literatura}
%
%% Následující příkaz je pouze pro ukázku sazby literatury při použití BibTeXu.
%% Způsobí citaci všech zdrojů v souboru literatura.bib, i když nejsou citovány v textu.
%\nocite{*}

%%% Vložení souboru 'text/zkratky' se seznam použitých symbolů, veličin a zkratek
\cleardoublepage
\chapter*{\listofabbrevname}
\phantomsection
\addcontentsline{toc}{chapter}{\listofabbrevname}

\begin{acronym}[KolikMista]

	\acro{JTAG}
		{Joint Test Action Group}
	\acro{TCK}
		{Test Clock}
	\acro{TMS}
		{Test Mode Select}
	\acro{TDI}
		{Test Data In}
	\acro{TDO}
		{Test Data Out}
	\acro{TCKC}
		{Test Clock Compact}
	\acro{TMSC}
		{Test Mode Select Compact}
	\acro{TRST}
		{Test Reset Input}
	\acro{TAP}
		{Test Access Port}
	\acro{DTS}
		{Debug and Test System}
	\acro{TS}
		{Target System}
	\acro{DM}
		{Debug Module}
	\acro{DMI}
		{Debug Module Interface}
	\acro{OAC}
		{Online Activation Code}	
	\acro{EC}
		{Extension Code}
	\acro{CP}
		{Check Packet}
	\acro{VHDL}
		{VHSIC Hardware Description Language}
	\acro{FPGA}
		{Programovatelné hradlové pole}
	\acro{DPS}
		{Deska plošných spojů}
	\acro{LSB}
		{Nejméně významný bit}
	\acro{MSB}
		{Nejvíce významný bit}
	\acro{RISC-V}
		{Reduced Instruction Set Computer - V}
	\acro{RISC}
		{Reduced Instruction Set Computer}
	\acro{PULP}
		{Parallel Ultra Low Power}
	\acro{MPSSE}
		{Multi-Purpose Synchronous Serial Engine}
\end{acronym}


%%% Začátek příloh
\appendix

%%% Vysázení seznamu příloh
% (vynechejte, pokud máte dvě nebo méně příloh)
\listofappendices

%%% Vložení souboru 'text/prilohy' s přílohami
% Obvykle je přítomen alespoň popis co najdeme na přiloženém médiu
\chapter{Některé příkazy balíčku \texttt{thesis}}

\section{Příkazy pro sazbu veličin a jednotek}

\begin{table}[!h]
  \caption[Přehled příkazů]{Přehled příkazů pro matematické prostředí }
  \begin{center}
  	\small
	  \begin{tabular}{|c|c|c|c|}
	    \hline
	    Příkaz    						& Příklad 					& Zdroj příkladu  							& Význam  \\
	    \hline\hline
	    \verb|\textind{...}|	& $\beta_\textind{max}$ 	& \verb|$\beta_\textind{max}$|	& textový index \\
	    \hline
	    \verb|\const{...}| 		& $\const{U}_\textind{in}$ 				& \verb|$\const{U}_\textind{in}$|		& konstantní veličina \\
	    \hline
	    \verb|\var{...}| 		& $\var{u}_\textind{in}$ & \verb|$\var{u}_\textind{in}$| & proměnná veličina \\
	    \hline
	    \verb|\complex{...}| 	& $\complex{u}_\textind{in}$ & \verb|$\complex{u}_\textind{in}$| & komplexní veličina \\
	    \hline
	    \verb|\vect{...}| 		& $\vect{y}$ 						& \verb|$\vect{y}$| & vektor \\
	    \hline
	    \verb|\mat{...}| 	& $\mat{Z}$ 						& \verb|$\mat{Z}$| & matice \\
	    \hline
	    \verb|\unit{...}| 		& $\unit{kV}$ 						& \verb|$\unit{kV}$|\quad či\ \, \verb|\unit{kV}| & jednotka \\
	    \hline
	  \end{tabular}
  \end{center}
\end{table}



%\newpage
\section{Příkazy pro sazbu symbolů}

\begin{itemize}
  \item
    \verb|\E|, \verb|\eul| -- sazba Eulerova čísla: $\eul$,
  \item
    \verb|\J|, \verb|\jmag|, \verb|\I|, \verb|\imag| -- sazba imaginární jednotky: $\jmag$, $\imag$,
  \item
    \verb|\dif| -- sazba diferenciálu: $\dif$,
  \item
    \verb|\sinc| -- sazba funkce: $\sinc$,
  \item
    \verb|\mikro| -- sazba symbolu mikro stojatým písmem%
			\footnote{znak pochází z~balíčku \texttt{textcomp}}: $\mikro$,
	\item
		\verb|\uppi| -- sazba symbolu $\uppi$
			(stojaté řecké pí, na rozdíl od \verb|\pi|, což sází $\pi$).
\end{itemize}
%
Všechny symboly jsou určeny pro matematický mód, vyjma \verb|\mikro|, jenž je\\ použitelný rovněž v~textovém módu.
%$\upmikro$


\chapter{Druhá příloha}

\begin{figure}[!h]
  \begin{center}
    \includegraphics[scale=0.5]{obrazky/ZlepseneWilsonovoZrcadloNPN}
  \end{center}
  \caption[Alenčino zrcadlo]{Zlepšené Wilsonovo proudové zrcadlo.}
\end{figure}

Pro sazbu vektorových obrázků přímo v~\LaTeX{}u je možné doporučit balíček \href{https://www.ctan.org/pkg/pgf}{\texttt{TikZ}}.
Příklady sazby je možné najít na \href{http://www.texample.net/tikz/examples/}{\TeX{}ample}.
Pro vyzkoušení je možné použít programy QTikz nebo TikzEdt.




\chapter{Příklad sazby zdrojových kódů}

\section{Balíček \texttt{listings}}

Pro vysázení zdrojových souborů je možné použít balíček \href{https://www.ctan.org/pkg/listings}{\texttt{listings}}.
Balíček zavádí nové prostředí \texttt{lstlisting} pro sazbu zdrojových kódů, jako například:
%
\begin{lstlisting}[language={[LaTeX]TeX}]
\section{Balíček lstlistings}
Pro vysázení zdrojových souborů je možné použít
	balíček \href{https://www.ctan.org/pkg/listings}%
	{\texttt{listings}}.
Balíček zavádí nové prostředí \texttt{lstlisting} pro
	sazbu zdrojových kódů.
\end{lstlisting}
%
Podporuje množství programovacích jazyků.
Kód k~vysázení může být načítán přímo ze zdrojových souborů.
Umožňuje vkládat čísla řádků nebo vypisovat jen vybrané úseky kódu.
Např.:

\noindent
Zkratky jsou sázeny v~prostředí \texttt{acronym}:
\label{lst:zkratky}
\lstinputlisting[language={[LaTeX]TeX},nolol,numbers=left, firstnumber=6, firstline=6,lastline=6]{text/zkratky.tex}
%
Šířka textu volitelného parametru \verb|KolikMista| udává šířku prvního sloupce se zkratkami.
Proto by měla být zadávána nejdelší zkratka nebo symbol.
Příklad definice zkratky \acs{symfvz} je na výpisu \ref{lst:symfvz}.

\shorthandoff{-}
\lstinputlisting[language={[LaTeX]TeX},frame=single,caption={Ukázka sazby zkratek},label=lst:symfvz,numbers=left,linerange={bsymfvz-\%\%\%\ esymfvz},includerangemarker=false]{text/zkratky.tex}
\shorthandon{-}

\noindent
Ukončení seznamu je provedeno ukončením prostředí:
\lstinputlisting[language={[LaTeX]TeX},nolol,numbers=left,firstnumber=26,linerange=26]{text/zkratky.tex}

\vspace{\fill}

\noindent
{\bf Poznámka k~výpisům s~použitím volby jazyka \verb|czech| nebo \verb|slovak|:}\newline
Pokud Váš zdrojový kód obsahuje znak spojovníku \verb|-|, pak překlad může skončit chybou.
Ta je způsobená tím, že znak \verb|-| je v~českém nebo slovenském nastavení balíčku \verb|babel| tzv.\ aktivním znakem.
Přepněte znak \verb|-| na neaktivní příkazem \verb|\shorthandoff{-}| těsně před výpisem a hned za ním jej vraťte na aktivní příkazem \verb|\shorthandon{-}|.
Podobně jako to je ukázáno ve zdrojovém kódu šablony.


\clearpage

%\section{Výpis kódu prostředí Matlab}
Na výpisu \ref{lst:priklad.vypis.kodu.Matlab} naleznete příklad kódu pro Matlab, na výpisu \ref{lst:priklad.vypis.kodu.C} zase pro jazyk~C.

\lstnewenvironment{matlab}[1][]{%
\iflanguage{czech}{\shorthandoff{-}}{}%
\iflanguage{slovak}{\shorthandoff{-}}{}%
\lstset{language=Matlab,numbers=left,#1}%
}{%
\iflanguage{slovak}{\shorthandon{-}}{}%
\iflanguage{czech}{\shorthandon{-}}{}%
}

\begin{matlab}[frame=single,float=htbp,caption={Příklad Schur-Cohnova testu stability v~prostředí Matlab.},label=lst:priklad.vypis.kodu.Matlab,numberstyle=\scriptsize, numbersep=7pt]
%% Priklad testovani stability filtru

% koeficienty polynomu ve jmenovateli
a = [ 5, 11.2, 5.44, -0.384, -2.3552, -1.2288];
disp( 'Polynom:'); disp(poly2str( a, 'z'))

disp('Kontrola pomoci korenu polynomu:');
zx = roots( a);
if( all( abs( zx) < 1))
    disp('System je stabilni')
else
    disp('System je nestabilni nebo na mezi stability');
end

disp(' '); disp('Kontrola pomoci Schur-Cohn:');
ma = zeros( length(a)-1,length(a));
ma(1,:) = a/a(1);
for( k = 1:length(a)-2)
    aa = ma(k,1:end-k+1);
    bb = fliplr( aa);
    ma(k+1,1:end-k+1) = (aa-aa(end)*bb)/(1-aa(end)^2);
end

if( all( abs( diag( ma.'))))
    disp('System je stabilni')
else
    disp('System je nestabilni nebo na mezi stability');
end
\end{matlab}

\noindent
\begin{minipage}{\linewidth}


%\section{Výpis kódu jazyka C}

\begin{lstlisting}[frame=single,numbers=right,caption={Příklad implementace první kanonické formy v~jazyce C.},label=lst:priklad.vypis.kodu.C,basicstyle=\ttfamily\small, keywordstyle=\color{black}\bfseries\underbar,]
// první kanonická forma
short fxdf2t( short coef[][5], short sample)
{
	static int v1[SECTIONS] = {0,0},v2[SECTIONS] = {0,0};
	int x, y, accu;
	short k;

	x = sample;
	for( k = 0; k < SECTIONS; k++){
		accu = v1[k] >> 1;
		y = _sadd( accu, _smpy( coef[k][0], x));
		y = _sshl(y, 1) >> 16;

		accu = v2[k] >> 1;
		accu = _sadd( accu, _smpy( coef[k][1], x));
		accu = _sadd( accu, _smpy( coef[k][2], y));
		v1[k] = _sshl( accu, 1);

		accu = _smpy( coef[k][3], x);
		accu = _sadd( accu, _smpy( coef[k][4], y));
		v2[k] = _sshl( accu, 1);

		x = y;
	}
	return( y);
}
\end{lstlisting}
\end{minipage}







\chapter{Obsah elektronické přílohy}
Elektronická příloha je často nedílnou součástí semestrální nebo závěrečné práce.
Vkládá se do informačního systému VUT v~Brně ve vhodném formátu (ZIP, PDF\,\dots).

Nezapomeňte uvést, co čtenář v~této příloze najde.
Je vhodné okomentovat obsah každého adresáře, specifikovat, který soubor obsahuje důležitá nastavení, který soubor je určen ke spuštění, uvést nastavení kompilátoru atd.
Také je dobře napsat, v~jaké verzi software byl kód testován (např.\ Matlab 2018b).
Pokud bylo cílem práce vytvořit hardwarové zařízení,
musí elektronická příloha obsahovat veškeré podklady pro výrobu (např.\ soubory s~návrhem DPS v~Eagle).

Pokud je souborů hodně a jsou organizovány ve více složkách, je možné pro výpis adresářové struktury použít balíček \href{https://www.ctan.org/pkg/dirtree}{\texttt{dirtree}}.

\bigskip

{\small
%
\dirtree{%.
.1 /\DTcomment{kořenový adresář přiloženého archivu}.
.2 logo\DTcomment{loga školy a fakulty}.
.3 BUT\_abbreviation\_color\_PANTONE\_EN.pdf.
.3 BUT\_color\_PANTONE\_EN.pdf.
.3 FEEC\_abbreviation\_color\_PANTONE\_EN.pdf.
.3 FEKT\_zkratka\_barevne\_PANTONE\_CZ.pdf.
.3 UTKO\_color\_PANTONE\_CZ.pdf.
.3 UTKO\_color\_PANTONE\_EN.pdf.
.3 VUT\_barevne\_PANTONE\_CZ.pdf.
.3 VUT\_symbol\_barevne\_PANTONE\_CZ.pdf.
.3 VUT\_zkratka\_barevne\_PANTONE\_CZ.pdf.
.2 obrazky\DTcomment{ostatní obrázky}.
.3 soucastky.png.
.3 spoje.png.
.3 ZlepseneWilsonovoZrcadloNPN.png.
.3 ZlepseneWilsonovoZrcadloPNP.png.
.2 pdf\DTcomment{pdf stránky generované informačním systémem}.
.3 student-desky.pdf.
.3 student-titulka.pdf.
.3 student-zadani.pdf.
.2 text\DTcomment{zdrojové textové soubory}.
.3 literatura.tex.
.3 prilohy.tex.
.3 reseni.tex.
.3 uvod.tex.
.3 vysledky.tex.
.3 zaver.tex.
.3 zkratky.tex.
%.2 navod-sablona\_FEKT.pdf\DTcomment{návod na používání šablony}.
.2 sablona-obhaj.tex\DTcomment{hlavní soubor pro sazbu prezentace k~obhajobě}.
%.2 readme.txt\DTcomment{soubor s~popisem obsahu CD}.
.2 sablona-prace.tex\DTcomment{hlavní soubor pro sazbu kvalifikační práce}.
.2 thesis.sty\DTcomment{balíček pro sazbu kvalifikačních prací}.
}
}


\end{document}