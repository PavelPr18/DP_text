% Soubory musí být v kódování, které je nastaveno v příkazu \usepackage[...]{inputenc}

\documentclass[%        Základní nastavení
%  draft,    				  % Testovací překlad
  12pt,       				% Velikost základního písma je 12 bodů
  a4paper,    				% Formát papíru je A4
  oneside,      			% Jednostranný tisk
	%twoside,      			% Dvoustranný tisk (kapitoly a další důležité části tedy začínají na lichých stranách)
	unicode,						% Záložky a metainformace ve výsledném  PDF budou v kódování unicode
]{report}				    	% Dokument třídy 'zpráva', vhodná pro sazbu závěrečných prací s kapitolami

\usepackage[utf8]		  %	Kódování zdrojových souborů je UTF-8
	{inputenc}					% Balíček pro nastavení kódování zdrojových souborů

\usepackage[				% Nastavení geometrie stránky
	bindingoffset=10mm,		% Hřbet pro vazbu
	hmargin={25mm,25mm},	% Vnitřní a vnější okraj  (jsou nehezky shodné; jakási úroveň estetiky je dosažena pomocí hřbetu)
	vmargin={25mm,34mm},	% Horní a dolní okraj
	footskip=17mm,			  % Velikost zápatí
	nohead,					      % Bez záhlaví
	marginparsep=2mm,		  % Vzdálenost marginálií
	marginparwidth=18mm,	% Šířka marginálií
]{geometry}

\usepackage{sectsty}
	%přetypuje nadpisy všech úrovní na bezpatkové, kromě \chapter, která je přenastavena zvlášť v thesis.sty
	\allsectionsfont{\sffamily}

\usepackage{graphicx} % Balíček 'graphicx' pro vkládání obrázků
											% Nutné pro vložení logotypů školy a fakulty

\usepackage[          % Balíček 'acronym' pro sazby zkratek a symbolů
	nohyperlinks				% Nebudou tvořeny hypertextové odkazy do seznamu zkratek
]{acronym}						
											% Nutné pro použití prostředí 'acronym' balíčku 'thesis'

\usepackage[
	breaklinks=true,		% Hypertextové odkazy mohou obsahovat zalomení řádku
	hypertexnames=false % Názvy hypertext. odkazů budou tvořeny nezávisle na názvech TeXu
]{hyperref}						% Balíček 'hyperref' pro sazbu hypertextových odkazů
											% Nutné pro použití příkazu 'pdfsettings' balíčku 'thesis'

\usepackage{pdfpages} % Balíček umožňující vkládat stránky z PDF souborů
                      % Nutné při vkládání titulních listů a zadání přímo
                      % ve formátu PDF z informačního systému

\usepackage{enumitem} % Balíček pro nastavení mezerování v odrážkách
  \setlist{topsep=0pt,partopsep=0pt,noitemsep} % konkrétní nastavení

\usepackage{cmap} 		% Balíček cmap zajišťuje, že PDF vytvořené `pdflatexem' je
											% plně "prohledávatelné" a "kopírovatelné"

%\usepackage{upgreek}	% Balíček pro sazbu stojatých řeckých písmem
											%% např. stojaté pí: \uppi
											%% např. stojaté mí: \upmu (použitelné třeba v mikrometrech)
											%% pozor, grafická nekompatibilita s fonty typu Computer Modern!
                      
%\usepackage{amsmath} %balíček pro sabu náročnější matematiky                 

\usepackage{dirtree}	% sazba adresářové struktury
                      % vhodné pro prezentaci obsahu elektronické přílohy (např. CD)

\usepackage[formats]{listings}	% Balíček pro sazbu zdrojových textů
\lstset{              % nastavení
%	Definice jazyka použitého ve výpisech
%    language=[LaTeX]{TeX},	% LaTeX
%	language={Matlab},		% Matlab
    basicstyle=\ttfamily,	% definice základního stylu písma
	language={C},           % jazyk C
    tabsize=2,			% definice velikosti tabulátoru
    inputencoding=utf8,         % pro soubory uložené v kódování UTF-8
		columns=fixed,  %fixed nebo flexible,
		fontadjust=true %licovani sloupcu
    extendedchars=true,
    literate=%  definice symbolů s diakritikou
    {á}{{\'a}}1
    {č}{{\v{c}}}1
    {ď}{{\v{d}}}1
    {é}{{\'e}}1
    {ě}{{\v{e}}}1
    {í}{{\'i}}1
    {ň}{{\v{n}}}1
    {ó}{{\'o}}1
    {ř}{{\v{r}}}1
    {š}{{\v{s}}}1
    {ť}{{\v{t}}}1
    {ú}{{\'u}}1
    {ů}{{\r{u}}}1
    {ý}{{\'y}}1
    {ž}{{\v{z}}}1
    {Á}{{\'A}}1
    {Č}{{\v{C}}}1
    {Ď}{{\v{D}}}1
    {É}{{\'E}}1
    {Ě}{{\v{E}}}1
    {Í}{{\'I}}1
    {Ň}{{\v{N}}}1
    {Ó}{{\'O}}1
    {Ř}{{\v{R}}}1
    {Š}{{\v{S}}}1
    {Ť}{{\v{T}}}1
    {Ú}{{\'U}}1
    {Ů}{{\r{U}}}1
    {Ý}{{\'Y}}1
    {Ž}{{\v{Z}}}1
}

%%%%%%%%%%%%%%%%%%%%%%%%%%%%%%%%%%%%%%%%%%%%%%%%%%%%%%%%%%%%%%%%%
%%%%%%      Definice informací o dokumentu             %%%%%%%%%%
%%%%%%%%%%%%%%%%%%%%%%%%%%%%%%%%%%%%%%%%%%%%%%%%%%%%%%%%%%%%%%%%%

% V tomto souboru se nastavují téměř veškeré informace, proměnné mezi studenty:
% jméno, název práce, pohlaví atd.
% Tento soubor je SDÍLENÝ mezi textem práce a prezentací k obhajobě -- netřeba něco nastavovat na dvou místech.

\usepackage[
%%% Z následujících voleb jazyka lze použít pouze jednu
  czech-english,		% originální jazyk je čeština, překlad je anglicky (výchozí)
  %english-czech,	% originální jazyk je angličtina, překlad je česky
  %slovak-english,	% originální jazyk je slovenština, překlad je anglicky
  %english-slovak,	% originální jazyk je angličtina, překlad je slovensky
%
%%% Z následujících voleb typu práce lze použít pouze jednu
  semestral,		  % semestrální práce (výchozí)
  %bachelor,			%	bakalářská práce
  %master,			  % diplomová práce
  %treatise,			% pojednání o disertační práci
  %doctoral,			% disertační práce
%
%%% Z následujících voleb zarovnání objektů lze použít pouze jednu
%  left,				  % rovnice a popisky plovoucích objektů budou zarovnány vlevo
	center,			    % rovnice a popisky plovoucích objektů budou zarovnány na střed (vychozi)
%
]{thesis}   % Balíček pro sazbu studentských prací


%%% Jméno a příjmení autora ve tvaru
%  [tituly před jménem]{Křestní}{Příjmení}[tituly za jménem]
% Pokud osoba nemá titul před/za jménem, smažte celý řetězec '[...]'
\author[Bc.]{Pavel}{Prášil}

%%% Identifikační číslo autora (VUT ID)
\butid{216847}

%%% Pohlaví autora/autorky
% (nepoužije se ve variantě english-czech ani english-slovak)
% Číselná hodnota: 1...žena, 0...muž
\gender{0}

%%% Jméno a příjmení vedoucího/školitele včetně titulů
%  [tituly před jménem]{Křestní}{Příjmení}[tituly za jménem]
% Pokud osoba nemá titul před/za jménem, smažte celý řetězec '[...]'
\advisor[Ing.]{Petr}{Petyovský}[Ph.D.]

%%% Jméno a příjmení oponenta včetně titulů
%  [tituly před jménem]{Křestní}{Příjmení}[tituly za jménem]
% Pokud osoba nemá titul před/za jménem, smažte celý řetězec '[...]'
% Nastavení oponenta se uplatní pouze v prezentaci k obhajobě;
% v případě, že nechcete, aby se na titulním snímku prezentace zobrazoval oponent, pouze příkaz zakomentujte;
% u obhajoby semestrální práce se oponent nezobrazuje (jelikož neexistuje)
% U dizertační práce jsou typicky dva až tři oponenti. Pokud je chcete mít na titulním slajdu, prosím ručně odkomentujte a upravte jejich jména v definici "VUT title page" v souboru thesis.sty.
\opponent[doc.\ Mgr.]{Křestní}{Příjmení}[Ph.D.]

%%% Název práce
%  Parametr ve složených závorkách {} je název v originálním jazyce,
%  parametr v hranatých závorkách [] je překlad (podle toho jaký je originální jazyk).
%  V případě, že název Vaší práce je dlouhý a nevleze se celý do zápatí prezentace, použijte příkaz
%  \def\insertshorttitle{Zkác.\ náz.\ práce}
%  kde jako parametr vyplníte zkrácený název. Pokud nechcete zkracovat název, budete muset předefinovat,
%  jak se vytváří patička slidu. Viz odkaz: https://bit.ly/3EJTp5A
\title[Implementation of system for IC testing via JTAG interface]{Implementace systému pro testování integrovaných obvodů pomocí JTAG rozhraní}

%%% Označení oboru studia
%  Parametr ve složených závorkách {} je název oboru v originálním jazyce,
%  parametr v hranatých závorkách [] je překlad
\specialization[Cybernetics, Control and Measurements]{Kybernetika, automatizace a měření}

%%% Označení ústavu
%  Parametr ve složených závorkách {} je název ústavu v originálním jazyce,
%  parametr v hranatých závorkách [] je překlad
\department[Department of Control and Instrumentation]{Ústav automatizace a měřicí techniky}
%\department[Department of Biomedical Engineering]{Ústav biomedicínského inženýrství}
%\department[Department of Electrical Power Engineering]{Ústav elektroenergetiky}
%\department[Department of Electrical and Electronic Technology]{Ústav elektrotechnologie}
%\department[Department of Physics]{Ústav fyziky}
%\department[Department of Foreign Languages]{Ústav jazyků}
%\department[Department of Mathematics]{Ústav matematiky}
%\department[Department of Microelectronics]{Ústav mikroelektroniky}
%\department[Department of Radio Electronics]{Ústav radioelektroniky}
%\department[Department of Theoretical and Experimental Electrical Engineering]{Ústav teoretické a experimentální elektrotechniky}
%\department[Department of Telecommunications]{Ústav telekomunikací}
%\department[Department of Power Electrical and Electronic Engineering]{Ústav výkonové elektrotechniky a elektroniky}

%%% Označení fakulty
%  Parametr ve složených závorkách {} je název fakulty v originálním jazyce,
%  parametr v hranatých závorkách [] je překlad
%\faculty[Faculty of Architecture]{Fakulta architektury}
\faculty[Faculty of Electrical Engineering and~Communication]{Fakulta elektrotechniky a~komunikačních technologií}
%\faculty[Faculty of Chemistry]{Fakulta chemická}
%\faculty[Faculty of Information Technology]{Fakulta informačních technologií}
%\faculty[Faculty of Business and Management]{Fakulta podnikatelská}
%\faculty[Faculty of Civil Engineering]{Fakulta stavební}
%\faculty[Faculty of Mechanical Engineering]{Fakulta strojního inženýrství}
%\faculty[Faculty of Fine Arts]{Fakulta výtvarných umění}
%
%Nastavení logotypu (v hranatych zavorkach zkracene logo, ve slozenych plne):
\facultylogo[logo/FEKT_zkratka_barevne_PANTONE_CZ]{logo/UAMT_color_PANTONE_CZ}

%%% Rok odevzdání práce
\graduateyear{2024}
%%% Akademický rok odevzdání práce
\academicyear{2023/24}

%%% Datum obhajoby (uplatní se pouze v prezentaci k obhajobě)
\date{11.\,11.\,1980} 

%%% Místo obhajoby
% Na titulních stránkách bude automaticky vysázeno VELKÝMI písmeny (pokud tyto stránky sází šablona)
\city{Brno}

%%% Abstrakt
\abstract[%
This semestral thesis deals with testing IC's containing \acs{RISC-V} processor core by \acs{JTAG} protocol. This thesis objective is to design an IP block for 2-wire \acs{JTAG} protocol support xand design of extended protocol for \acs{RISC-V} processor system bus access. Designed IP block will be used for the IC testing via 2-wire \acs{JTAG} interface, for the purpose of pin count reduction. The extended protocol will be used for IC testing time optimization. The thesis includes descriptions of the \acs{RISC-V} testing system, designed IP block for 2-wire \acs{JTAG} protocol handling and description of its implementation, and design of optimized protocol. 
]{%
Tato semestrální práce se zabývá testováním integrovaných obvodů s procesorem \acs{RISC-V} pomocí \acs{JTAG} protokolu. Cílem práce je návrh modulu pro podporu dvouvodičové varianty \acs{JTAG} protokolu a návrh rozšiřujícího protokolu pro přístup na systémovou sběrnici \acs{RISC-V} procesoru pomocí \acs{JTAG} rozhraní. Navržený modul bude použit pro testování integrovaného obvodu pomocí dvouvodičového \acs{JTAG} rozhraní, za účelem redukce počtu pinů. Rozšiřující protokol bude sloužit pro časovou optimalizaci testování integrovaného obvodu. Práce obsahuje popisy systému pro testování \acs{RISC-V} procesorů, navrženého modulu pro dvouvodičový \acs{JTAG} protokol, včetně způsobu jeho implementace, a návrhu optimalizovaného protokolu.
}

%%% Klíčová slova
\keywrds[%
JTAG, FPGA, VHDL, RISC-V, IC testing
]{%
JTAG, FPGA, VHDL, RISC-V, testování integrovaných obvodů
}

%%% Poděkování
\acknowledgement{%
Rád bych poděkoval vedoucímu diplomové práce
panu Ing.~Petru Petyovskému, Ph.D.\ za odborné vedení,
konzultace, trpělivost a~podnětné návrhy k~práci.
}%  % do tohoto souboru doplňte údaje o sobě, druhu práce, názvu...

%%%%%%%%%%%%%%%%%%%%%%%%%%%%%%%%%%%%%%%%%%%%%%%%%%%%%%%%%%%%%%%%%%%%%%%%

%%%%%%%%%%%%%%%%%%%%%%%%%%%%%%%%%%%%%%%%%%%%%%%%%%%%%%%%%%%%%%%%%%%%%%%%
%%%%%%     Nastavení polí ve Vlastnostech dokumentu PDF      %%%%%%%%%%%
%%%%%%%%%%%%%%%%%%%%%%%%%%%%%%%%%%%%%%%%%%%%%%%%%%%%%%%%%%%%%%%%%%%%%%%%
%% Při načteném balíčku 'hyperref' lze použít příkaz '\pdfsettings':
\pdfsettings
%  Nastavení polí je možné provést také ručně příkazem:
%\hypersetup{
%  pdftitle={Název studentské práce},    	% Pole 'Document Title'
%  pdfauthor={Autor studenstké práce},   	% Pole 'Author'
%  pdfsubject={Typ práce}, 						  	% Pole 'Subject'
%  pdfkeywords={Klíčová slova}           	% Pole 'Keywords'
%}
%%%%%%%%%%%%%%%%%%%%%%%%%%%%%%%%%%%%%%%%%%%%%%%%%%%%%%%%%%%%%%%%%%%%%%%

\pdfmapfile{=vafle.map}
% pridane balicky
\usepackage{array}		% hodí se pro změnu tloušťky tabulek
\newcolumntype{P}[1]{>{\centering\arraybackslash}p{#1}}	% nový typ sloupce odstavcový centrovaný horizontalne
\newcolumntype{M}[1]{>{\centering\arraybackslash}m{#1}}	% nový typ sloupce odstavcový centrovaný i vetikálně
\usepackage{amsmath}	% dalsi matematika (např.: prostredi split)
\usepackage{placeins} % prikaz \FloatBarrier (nedovoli obrazky presunout do jine sekce)
\usepackage{float}		% [H] u obrazku/tabulek
\usepackage{soul}			% prikaz \hl (highlight)
%\sethlcolor{yellow}		% barva pro highlight
%%%%%%%%%%%%%%%%%%%%%%%%%%%%%%%%%%%%%%%%%%%%%%%%%%%%%%%%%%%%%%%%%%%%%%%
%%%%%%%%%%%       Začátek dokumentu               %%%%%%%%%%%%%%%%%%%%%
%%%%%%%%%%%%%%%%%%%%%%%%%%%%%%%%%%%%%%%%%%%%%%%%%%%%%%%%%%%%%%%%%%%%%%%
\begin{document}
\pagestyle{empty} %vypnutí číslování stránek

%%% Vložení desek -- od září 2021 na žádost fakulty nepoužíváno
\includepdf[pages=1]%  buďto generovaných informačním systémem
	{pdf/desky}% název souboru nesmí obsahovat mezery!
%%% NEBO vytvoření desek z balíčku
%%\makecover
%%%
%\oddpage % při dvojstranném tisku přidá prázdnou stránku
%% kazdopádně ale:
%\setcounter{page}{1} %resetovaní čítače stránek -- desky do číslování nezahrnujeme

%% Vložení titulního listu
\includepdf[pages=1]%    buďto generovaného informačním systémem
  {pdf/TitulniList_color}% název souboru nesmí obsahovat mezery!
%% NEBO vytvoření titulní stránky z balíčku
%\maketitle
%%
\oddpage  % při dvojstranném tisku se přidá prázdná stránka
   
%% Vložení zadání
\includepdf[pages=1]%   buďto generovaného informačním systémem
  {pdf/zadani_sem}% název souboru nesmí obsahovat mezery!
%% NEBO lze vytvořit prázdný list příkazem ze šablony
%\patternpage{}%
%	{\sffamily\Huge\centering ZDE VLOŽIT LIST ZADÁNÍ}%
%	{\sffamily\centering Z~důvodu správného číslování stránek}
%%
\oddpage% při dvojstranném tisku se přidá prázdná stránka

%% Vysázení stránky s abstraktem
\makeabstract

% Vysázení stránky s rozšířeným abstraktem
% (pokud píšete práci v češtině či slovenštině, vložení rozšířeného abstraktu zrušte;
%  pro semestrální projekt také není potřeba rozšířený abstrakt uvádět)
%% Vysázení stránky s rozšířeným abstraktem
% (týká se pouze bc. a dp. prací psaných v angličtině, viz Směrnice rektora 72/2017)
\cleardoublepage
\noindent
{\large\sffamily\bfseries\MakeUppercase{Rozšířený abstrakt}}
\\
Výtah ze směrnice rektora 72/2017:\\
\emph{Bakalářská a diplomová práce předložená v angličtině musí obsahovat rozšířený abstrakt v češtině
nebo slovenštině (čl. 15). To se netýká studentů, kteří studují studijní program akreditovaný v
angličtině.}
(čl. 3, par. 7)\\
\emph{Nebude-li vnitřní normou stanoveno jinak, doporučuje se rozšířený abstrakt o rozsahu přibližně 3
normostrany, který bude obsahovat úvod, popis řešení a shrnutí a~zhodnocení výsledků.}
(čl. 15, par. 5)

%%% Vysázení citace práce
\makecitation

%%% Vysázení prohlášení o samostatnosti
\makedeclaration

%%% Vysázení poděkování
%\makeacknowledgement

%%% Vysázení obsahu
\tableofcontents

%%% Vysázení seznamu obrázků
% (vynechejte, pokud máte dva nebo méně obrázků)
\listoffigures

%%% Vysázení seznamu tabulek
% (vynechejte, pokud máte dvě nebo méně tabulek)
\listoftables

%%% Vysázení seznamu výpisů kódu
% (vynechejte, pokud máte dva nebo méně výpisů)
%\lstlistoflistingss

\cleardoublepage\pagestyle{plain}   % zapnutí číslování stránek

%Pro vkládání kapitol i příloh používejte raději \include než \input
%%% Vložení souboru 'text/uvod.tex' s úvodem
\chapter*{Úvod}
\phantomsection
\addcontentsline{toc}{chapter}{Úvod}

Tématem této semestrální práce je návrh modulu pro dvouvodičový \acs{JTAG} protokol a rozšiřujícího protokolu pro přístup na systémovou sběrnici \acs{RISC-V} procesoru.

Cílem semestrální práce je navrhnout modul podporující komunikaci pomocí dvouvodičové varianty \acs{JTAG} protokolu s možností přepínat mezi touto a čtyřvodičovou variantou dle standardu IEEE 1149.7. Dalším cílem je návrh časově optimalizovaný protokol využívající \acs{JTAG} rozhraní pro testovaní procesorů \acs{RISC-V}.

V úvodní části práce je obecně přestaveno testovací rozhraní \acs{JTAG} a základní princip fungování řídicího stavového automatu dle příslušného standardu IEEE 1149.1. Dále je uveden testovací systém pro procesory \acs{RISC-V}, který je použitý pro přístup na systémovou sběrnici procesoru. Princip fungování systémové sběrnice použitého procesorového jádra je také představen v rámci úvodní kapitoly.

Druhá kapitola se věnuje principu komunikace dvouvodičovou variantou \acs{JTAG} protokolu, způsobu aktivace dvouvodičového režimu a také možnosti přepnutí na původní čtyřvodičovou variantu. Dále je v této kapitole popsán způsob návrhu a implementace modulu zajišťujícího uvedenou funkcionalitu.

V poslední části práce je popsán navržený protokol pro přístup na systémovou sběrnici procesoru. Tento protokol bude využitý jako vedlejší možnost přístupu na systémovou sběrnici, která je časově efektivnější oproti stávajícímu způsobu využívajícího systém pro testování \acs{RISC-V} procesorů. Uveden je také způsob zkrácení přenášené adresy registru, ke kterému je přistupováno a možnosti délky přenášených dat. Protokol je navržený ve dvou variantách, které jsou popsány detailněji v jednotlivých částech této kapitoly a jsou zhodnoceny jejich vlastnosti.



%Vlozeni souboru s prvni kapitolou (JTAG rozhraní pro RISC-V degugování)
\chapter{Komunikační protokol JTAG}
Tato kapitola popisuje princip přenosu dat na pomocí komunikačního protokolu JTAG a jeho použití pro testování obvodů s procesorem architektury \acs{RISC-V} (\acl{RISC-V}).

\section{Základní popis}
\acs{JTAG} (\acl{JTAG}) je standardizovaná komunikační sběrnice určená pro testování integrovaných obvodů a \acs{DPS} (\acl{DPS}).
\acs{JTAG} rozhraní definuje tzv. \acs{TAP} (\acl{TAP}), který je skupinou vstupů a výstupů určených k testování. Jelikož \acs{JTAG} je synchronní sběrnice, její rozhraní zahrnuje samostatný vodič pro hodinový signál \textbf{\acs{TCK}} (\acl{TCK}). Dalším vodičem je \textbf{\acs{TMS}} (\acl{TMS}), kterým je přenášen řídicí signál pro stavový automat. Vodiče zajišťující přenos dat jsou označeny \textbf{\acs{TDI}} (\acl{TDI}) a \textbf{\acs{TDO}} (\acl{TDO}). Jako volitelný vodič základního rozhraní je možné připojit také \textbf{\acs{TRST}} (\acl{TRST}), který provádí reset stavového automatu. Reset se využívá především k inicializaci automatu během připojení napájení. \cite {IEEE_1149-1} \cite{JTAG}      

\section{Stavový automat dle IEEE 1149.1}
Standard IEEE 1149.1 definuje stavový automat, který je hlavní součástí \acs{TAP} (\acl{TAP}). Stavový diagram stavového automatu je zobrazen na obrázku \ref{fig:tap_controller}. Všechny přechody mezi stavy toho automatu jsou plně synchronní se vstupním hodinovým signálem \acs{TCK} a jsou určeny řídicím signálem \acs{TMS}. Důležitou vlastností tohoto stavového automatu je způsob resetování jeho stavu, tedy návrat do stavu \textbf{Test-Logic-Reset}. Reset lze provést pomocí minimálně pěti po sobě jdoucích hodinových taktů, kdy hodnota řídicího signálu \acs{TMS} setrvává v log. 1. Tento způsob resetování je možný provést ze kteréhokoliv stavu automatu. Příkladem může být provedení resetu, když se automat nachází ve stavu \textbf{Shift-DR}. Automat tedy projde postupně stavy \textbf{Exit1-DR}, \textbf{Update-DR}, \textbf{Select-DR-Scan} a \textbf{Select-DR-Scan} až do stavu \textbf{Test-Logic-Reset}. V případě kratší cesty do \textbf{Test-Logic-Reset} stavu setrvává při hodnotě log. 1 na signálu \acs{TMS} v tomto stavu.
\begin{figure}[!h]
  \begin{center}
    \includegraphics[scale=0.34]{obrazky/JTAG_TAP_Controller_State_Diagram.png}
  \end{center}
  \caption{Stavový diagram řídicího stavového automatu JTAGu \cite{JTAG_TAP_diagram}}
	\label{fig:tap_controller}
\end{figure}

\section{Systém pro testování \acs{RISC-V}}
Pro procesory architektury \acs{RISC-V} je definován systém pro jejich testování a ladění. Blokové schéma systému je zobrazené na obrázku \ref{fig:blok_sch_risc-v_dbg}. Systém umožňuje vykonávat krátký program 

\begin{figure}[!h]
  \begin{center}
    \includegraphics[scale=0.65]{obrazky/risc-v_debug_system_blok_sch.png}
  \end{center}
  \caption{Blokové schéma systému použitého pro testování \acs{RISC-V} procesorů \cite{risc-v_dbg}}
	\label{fig:blok_sch_risc-v_dbg}
\end{figure}

%Vlozeni souboru s druhou kapitolou (JTAG rozhraní 2w)
% TODO: sjednotit pořadí "dvou nebo čtyřvodič"
%				projít si jak nazývám DTS a TS

\chapter{Návrh a implementace modulu pro dvouvodičový \acs{JTAG} protokol}
Tato kapitola se zaměřuje na princip fungování dvouvodičového \acs{JTAG} rozhraní s možností přepnutí na původní čtyřvodičovou variantu a zpět. Popsána je zde také aktivační sekvence pro výběr jedné z těchto dvou variant komunikace, definovaná standardem IEEE 1149.7. V této kapitole je také uveden způsob implementace modulu pro zpracování aktivační sekvence a převod vstupního dvouvodičového \acs{JTAG} rozhraní na původní čtyřvodičové. %, pokud je zvolena dvouvodičová varianta.
	
\section{Účel redukce počtu pinů \acs{JTAG} rozhraní}	\label{sec:2w_interface}
Důvodem pro snížení počtu pinů potřebných pro testování integrovaných obvodů je možnost využít dva volné piny k jinému účelu během testování, například pro jejich běžnou funkci, kterou mají mimo testovací režim. Dvouvodičové \acs{JTAG} rozhraní je definováno standardem IEEE 1149.7 a je nazýváno \textbf{compact \acs{JTAG}} (c\acs{JTAG}). Dvouvodičové rozhraní využívá ke komunikaci piny \texttt{\acs{TCK}} a \texttt{\acs{TMS}}, které mají pro dvouvodičovou variantu označení s příponou C (compact), tedy \texttt{\acs{TCKC}} a \texttt{\acs{TMSC}}. \cite{IEEE_1149-7} \cite{JTAG}

Dvouvodičová varianta podporuje zapojení ve hvězdicové topologii, jak je vidět na obrázku \ref{fig:star2_sch}. Nicméně návrh modulu popisovaného v této kapitole se omezuje pouze na jedno zařízení, protože funkcionalita podporující připojení více zařízení do hvězdicové topologie není implementována.

\begin{figure}[!h]
  \begin{center}
    \includegraphics[scale=0.9]{obrazky/Example_of_reduced_pin_count_JTAG_interface.pdf}
  \end{center}
  \caption{Blokové schéma hvězdicové topologie \acs{JTAG} rozhraní ve dvouvodičové variantě. \cite{JTAG}.}
	\label{fig:star2_sch}
\end{figure}

Nevýhodou dvouvodičové varianty \acs{JTAG} rozhraní je potřeba serializace jednotlivých signálu původního čtyřvodičového rozhraní. Z tohoto důvodu je přenos dat pomocí dvouvodičové varianty v nejlepším případě 3-krát pomalejší než v případě původní čtyřvodičové. \cite{IEEE_1149-7}

\section{Serializace signálů čtyřvodičového \acs{JTAG} rozhraní na dvouvodičové}	\label{sec:oscan1} 
Pro dvouvodičovou variantu \acs{JTAG} komunikace je zapotřebí odesílat hodnoty \hl{datových signálů {\texttt{\acs{TDI}}}, {\texttt{\acs{TDO}}} a řídicího signálu {\texttt{\acs{TMS}}}} původního čtyřvodičového \acs{JTAG} protokolu sériově, v rámci jediného signálu \texttt{\acs{TMSC}}. Standard IEEE 1149.7 definuje několik verzí formátu serializace dat. Pro návrh popisovaný v této práci byl vybrán základní formát OSCAN1, který definuje prostou serializaci hodnot \texttt{\acs{TDI}}, \texttt{\acs{TMS}} a \texttt{\acs{TDO}} signálů. Formát OSCAN1 byl zvolen pro jeho jednoduchost a konzistenci ve všech stavech řídicího stavového automatu \acs{TAP}. Formát je zobrazen na obrázku \ref{fig:oscan}, kde je obecně naznačena předcházející aktivační sekvence. První bit formátu nese hodnotu negace \texttt{\acs{TDI}} signálu, druhým bitem je přenesena hodnota \texttt{\acs{TMS}} signálu a třetím je přenášena hodnota \texttt{\acs{TDO}} signálu opačným směrem. \cite{IEEE_1149-7}

\begin{figure}[!h]
  \begin{center}
    \includegraphics[scale=0.81]{obrazky/cJTAG_oscan.pdf}
  \end{center}
  \caption{Průběh serializace \acs{JTAG} signálů v OSCAN1 formátu.}
	\label{fig:oscan}
\end{figure}
    
\subsection{Řízení úrovně na \texttt{\acs{TMSC}} pinu}	\label{subsec:oscan1_drive} 
Při komunikaci pomocí dvouvodičového \acs{JTAG} rozhraní se stává pin \texttt{\acs{TMSC}} obousměrným, jelikož jsou přenášeny výstupní testovací data opačným směrem z \acs{JTAG} systému do debuggeru. Z toho důvodu musí být ošetřena možnost buzení pinu současně z obou směrů. Proto je standardem IEEE 1149.7 definováno pravidlo pro buzení \acs{TMSC} pinu během komunikace. Na obrázku \ref{fig:oscan_drive} je znázorněn průběh komunikace v OSCAN1 formátu, kde jsou přehledně znázorněny okamžiky buzení \texttt{\acs{TMSC}} pinu debuggerem (\acs{DTS}) a \acs{JTAG} systémem (\acs{TS}). Znázorněn je také průběh skutečné hodnoty na \texttt{\acs{TMSC}} pinu, kde je možné si všimnout, že při střídání směrů buzení je na pinu \texttt{\acs{TMSC}} zaručen stav vysoké impedance po dobu poloviny periody hodinového signálu \texttt{\acs{TCKC}}. Tímto je ošetřen vznik konfliktu v buzení \texttt{\acs{TMSC}} pinu. \cite{IEEE_1149-7}

\begin{figure}[!h]
  \begin{center}
    \includegraphics[scale=0.75]{obrazky/cJTAG_oscan_drive.pdf}
  \end{center}
  \caption{Průběh buzení \texttt{\acs{TMSC}} pinu v OSCAN1 formátu.}
	\label{fig:oscan_drive}
\end{figure}

%\subsection{Clock-gating}	\label{subsec:oscan1_clk_gate}

\section{Aktivační sekvence dvouvodičové varianty \acs{JTAG} protokolu}
Pro volbu komunikace pomocí dvouvodičové varianty \acs{JTAG} protokolu je zapotřebí provedení aktivace tohoto režimu, která je definována standardem IEEE 1149.7. Jelikož aktivaci musí být možné provést kdykoliv, tedy i při komunikaci dvouvodičovou variantou, jsou k ní využity piny \texttt{\acs{TCKC}} a \texttt{\acs{TMSC}}. Varianta \acs{JTAG} protokolu je volena debuggerem, který vyvolá příslušnou aktivační sekvenci na pinech \texttt{\acs{TCKC}} a \texttt{\acs{TMSC}}. \cite{IEEE_1149-7}

Aktivační sekvence se skládá ze dvou hlavních částí a je uvedena na obrázku \ref{fig:cJTAG_sel}. Nejdříve je \hl{třeba} deaktivovat funkcionalitu probíhajícího komunikačního režimu. Tato část se nazývá "Selection Escape" sekvence. Tato sekvence uvozuje následující sekvenci, určenou pro výběr komunikačního režimu, která se nazývá "Selection Sequence". \cite{IEEE_1149-7}

\begin{figure}[!h]
  \begin{center}
    \includegraphics[scale=0.7]{obrazky/cJTAG_selection.pdf}
  \end{center}
  \caption{Průběh aktivační sekvence pro dvouvodičovou variantu \acs{JTAG} protokolu.}
	\label{fig:cJTAG_sel}
\end{figure}

\subsection{Úvodní sekvence pro přepnutí varianty \acs{JTAG} protokolu}	\label{subsec:sel_escape}
Pro přenastavení varianty \acs{JTAG} komunikace je zapotřebí nejdříve deaktivovat probíhající komunikaci. Sekvence sloužící k deaktivaci je vidět na obrázku \ref{fig:cJTAG_sel} v první části průběhu. Tato sekvence využívá principu přepínání hodnoty na \texttt{\acs{TMSC}} pinu v době, kdy je na \texttt{\acs{TCKC}} signálu hodnota log. \texttt{1}. Díky tomuto principu je možné vyvolat sekvenci během komunikace, protože hodinový signál \texttt{\acs{TCKC}} po dobu sekvence negeneruje hodinové impulsy a nemůže tak nastat situace, kdy jsou hodnoty na pinu \texttt{\acs{TMSC}} dále zpracovány \acs{JTAG} systémem. \cite{IEEE_1149-7}

Pro tuto sekvenci jsou dle standardu IEEE 1149.7 definovány 4 varianty, které jsou odlišeny počtem pulsů vygenerovaných na \texttt{\acs{TMSC}} pinu, během setrvávající úrovně log. \texttt{1} na \texttt{\acs{TCKC}} signálu. \hl{Tyto varianty jsou uvedeny níže:}

\begin{itemize}
	\item 1 puls (2 až 3 hrany) - sekvence je rezervována pro uživatelské rozšíření. Tato možnost není podporována.
	\item 2 pulsy (4 až 5 hran) - nastane deaktivace probíhající komunikace. Tato možnost není podporována.
	\item 3 pulsy (6 až 7 hran) - umožňuje přepnutí na jinou variantu \acs{JTAG} komunikace. Tato možnost je v návrhu implementována a její časový průběh je zobrazen na obrázku \ref{fig:cJTAG_sel}, kde jsou zobrazeny dvě možné varianty s různou výchozí hodnotou. Po této sekvenci následuje sekvence pro volbu varianty \acs{JTAG} protokolu.
	\item 4 a více pulsů (8 a více hran) - dojde k resetování \acs{JTAG} systému. Tato možnost není podporována.
\end{itemize}

%Pro tuto sekvenci jsou dle standardu IEEE 1149.7 definovány 4 její varianty, které jsou odlišeny počtem pulsů vygenerovaných na \acs{TMSC} pinu, během setrvávající úrovně log. 1 na \acs{TCKC} signálu. Pokud debugger během této sekvence vygeneruje 1 puls, tedy 2 až 3 hrany, je sekvence rezervována pro uživatelské rozšíření. V případě 2 pulsů (4 až 5 hran) nastane pouze deaktivace probíhající komunikace. Pro možnost přepínání jsou definovány 3 pulsy (6 až 7 hran), jak je vidět na obrázku \ref{fig:cJTAG_sel}, kde jsou zobrazeny dvě možné varianty s různou výchozí hodnotou. Oba případy mají různý počet hran, ale počet nástupných hran je vždy stejný. Po takto odeslané sekvenci následuje sekvence pro volbu varianty \acs{JTAG} protokolu, což je také jediná podporovaná varianta v rámci návrhu popsaného v této kapitole. Pokud jsou odeslány 4 a více pulsů (8 a více hran), dojde k resetu. Dříve popsané varianty nejsou v rámci návrhu popsaného v této kapitole podporovány, z důvodu jejich postradatelnosti pro přepínání mezi dvou a čtyřvodičovým rozhraním. \cite{IEEE_1149-7}

%Pro tuto sekvenci jsou dle standardu IEEE 1149.7 definovány 4 její varianty, které jsou odlišeny počtem pulsů vygenerovaných na \acs{TMSC} pinu, během setrvávající úrovně log. 1 na \acs{TCKC} signálu. Pokud debugger během této sekvence vygeneruje 1 puls, tedy 2 až 3 hrany, je sekvence rezervována pro uživatelské rozšíření. V případě 2 pulsů (4 až 5 hran) nastane pouze deaktivace probíhající komunikace. Pokud jsou odeslány 4 a více pulsů (8 a více hran), dojde k resetu. Dříve popsané varianty nejsou v rámci návrhu popsaného v této kapitole podporovány, z důvodu jejich postradatelnosti pro přepínání mezi dvou a čtyřvodičovým rozhraním. \cite{IEEE_1149-7}

%Pro možnost přepínání jsou definovány 3 pulsy (6 až 7 hran), jak je vidět na obrázku \ref{fig:cJTAG_sel}, kde jsou zobrazeny dvě možné varianty s různou výchozí hodnotou. Oba případy mají různý počet hran, ale počet nástupných hran je vždy stejný. Po takto odeslané sekvenci následuje sekvence pro volbu varianty \acs{JTAG} protokolu.

\subsection{Sekvence pro výběr varianty \acs{JTAG} protokolu}
Sekvence výběru varianty \acs{JTAG} protokolu je zobrazena v druhé části obrázku \ref{fig:cJTAG_sel}. Sekvence již probíhá běžným způsobem komunikace, kdy debugger generuje hodinový signál \texttt{\acs{TCKC}} a odesílá požadovanou hodnotu sekvence \texttt{\acs{TMSC}} signálem. Sekvence se dělí na tři části, které jsou zobrazeny v tabulce \ref{tab:cJTAG_sel}.

\begin{table}[!h]
  \caption{Formát sekvence pro výběr varianty \acs{JTAG} protokolu \cite{IEEE_1149-7}}
  \begin{center}
  	\small
	  \begin{tabular}{!{\vrule width 1.2pt}c|c|c!{\vrule width 1.2pt}}
	    \noalign{\hrule height 1.2pt}
				\acl{OAC} (\acs{OAC}) [4] & \acl{EC} (\acs{EC}) [4] & \acl{CP} (\acs{CP}) [4-n]\\
			\noalign{\hrule height 1.2pt}
		\end{tabular}
  \end{center}
	\label{tab:cJTAG_sel}
\end{table}

\subsubsection{Část pro výběr konkrétní čtyř nebo dvouvodičové varianty - \acs{OAC}}
První část sekvence má konstantní délku 4 bity. První dva bity musí mít pro správnou aktivaci hodnotu log. \texttt{0}. Další dva bity definují výběr topologie, na které je \acs{JTAG} zařízení připojeno. Na výběr je možnost sériové topologie, tedy čtyřvodičové zapojení s jedním nebo více zařízeními zapojenými do série. Další možností je zapojení ve hvězdicové topologii, a to ve dvou nebo čtyřvodičové variantě. Pro navržené řešení je podstatná sériová topologie a dvouvodičová varianta hvězdicové topologie, které odpovídají hodnotám \acs{OAC} dle tabulky \ref{tab:oac}. \cite{IEEE_1149-7}

\begin{table}[!h]
  \caption{Tabulka významu OAC hodnot. \cite{IEEE_1149-7}}
  \begin{center}
  	\small
	  \begin{tabular}{!{\vrule width 1.2pt}c|c!{\vrule width 1.2pt}}
	    \noalign{\hrule height 1.2pt}
	    Hodnota \acs{OAC} & Význam hodnoty\\
	    \noalign{\hrule height 1.2pt}
			\texttt{0100} & Čtyřvodičová sériová topologie\\
			\hline
			\texttt{1100} & Dvouvodičová hvězdicová topologie\\
			\hline
			\hl{Ostatní} & \hl{Ostatní hodnoty nejsou podporovány}\\
			\hline
			\noalign{\hrule height 1.2pt}
		\end{tabular}
  \end{center}
	\label{tab:oac}
\end{table}

\subsubsection{Část rozšiřující informace o výběru varianty - \acs{EC}}
Druhou částí aktivační sekvence je čtyřbitová hodnota nazvaná \acl{EC}. Tato hodnota pouze doplňuje informace o výběru varianty \acs{JTAG} komunikace. Spodní tři bity udávají informace o výchozím stavu řídicího stavového automatu \acs{TAP} a o dodatečné ochraně před současným buzením pinů \texttt{\acs{TMS}} a \texttt{\acs{TDO}}. \hl{V rámci navrženého řešení je podporován výchozí stav stavového automatu \textit{Run-Test/Idle} nebo \textit{Test-Logic-Reset} a deaktivovaná dodatečná ochrana. Tomu odpovídají očekávané hodnoty log. \texttt{0}.} Nejvíce významný bit (\acs{MSB}) nese informaci o krátké, nebo dlouhé variantě popisované aktivační sekvence. Dlouhá forma aktivační sekvence obsahuje navíc hodnotu registru pro funkcionalitu, která není v návrhu implementována. Z tohoto důvodu musí být vždy zvolena krátká forma aktivační sekvence, které odpovídá hodnota log. \texttt{1} tohoto bitu. Hodnota této části aktivační sekvence musí být tedy vždy \texttt{1000}. \cite{IEEE_1149-7}

\subsubsection{Kontrolní část - \acs{CP}}
Poslední část aktivační sekvence slouží k prodloužení sekvence ze strany debuggeru nebo k vyvolání resetu. Prodloužení sekvence může být pro některé implementace debuggerů výhodné, pokud je třeba prodleva před začátkem samotné komunikace pro přípravu dat. Možnost sekvenci prodloužit je v návrhu popsaném v této kapitole realizováno. Vyvolání resetu v rámci aktivační sekvence není navrženým řešením podporováno. \cite{IEEE_1149-7}

Formát této části sekvence je zobrazen v tabulce \ref{tab:cJTAG_sel_cp}. Jednobitová hodnota první části \textit{Preamble} je \acs{JTAG} systémem ignorována a na její hodnotě nezáleží. Nicméně je doporučeno, aby byla tato hodnota stejná jako hodnota prvního bitu následující části \textit{Body}. Tato perioda hodinového signálu \texttt{\acs{TCKC}} může sloužit debuggeru jako čas pro změnu zdroje buzení \texttt{\acs{TMSC}} pinu. Následuje část těla kontrolní části sekvence, která může nabývat hodnot dle tabulky \ref{tab:cp_body}, kde je definován také význam těchto hodnot. Poslední částí je jednobitová hodnota \textit{Postamble}, která je \acs{JTAG} systémem také ignorována, přičemž je doporučeno použít stejnou hodnotu jako poslední bit části \textit{Body}. \cite{IEEE_1149-7}              

\begin{table}[H]
  \caption{Formát kontrolní části sekvence pro výběr varianty \acs{JTAG} protokolu. \cite{IEEE_1149-7}}
  \begin{center}
  	\small
	  \begin{tabular}{!{\vrule width 1.2pt}c|c|c!{\vrule width 1.2pt}}
	    \noalign{\hrule height 1.2pt}
				Preamble [1] & Body [2-n] & Postamble [1]\\
			\noalign{\hrule height 1.2pt}
		\end{tabular}
  \end{center}
	\label{tab:cJTAG_sel_cp}
\end{table}

\begin{table}[H]
  \caption{Tabulka významu CP hodnot. \cite{IEEE_1149-7}}
  \begin{center}
  	\small
	  \begin{tabular}{!{\vrule width 1.2pt}c|c|c!{\vrule width 1.2pt}}
	    \noalign{\hrule height 1.2pt}
	    Hodnota CP\_BODY & Označení & Význam hodnoty\\
	    \noalign{\hrule height 1.2pt}
			\texttt{00} & CP\_END & Ukončení sekvence\\
			\hline
			\texttt{01} nebo \texttt{10} & CP\_NOP & Rozšíření aktivační sekvence o jeden bit\\
			\hline
			\texttt{11} & CP\_RES & Reset \acs{JTAG} systému\\
			\hline
			\noalign{\hrule height 1.2pt}
		\end{tabular}
  \end{center}
	\label{tab:cp_body}
\end{table}

Jelikož navrženým systémem je podporováno pouze prodloužení aktivační sekvence, tedy hodnota CP\_BODY = CP\_NOP zakončení sekvence hodnotou \\CP\_BODY = CP\_END, je na obrázku \ref{fig:cJTAG_sel_cp_nop} zobrazen průběh sekvence v případě jejího prodloužení. Hodnota těla kontrolní části pro prodloužení sekvence o jeden bit může nabývat hodnot \texttt{01} nebo \texttt{10} dle tabulky \ref{tab:cp_body}. Proto je tedy sekvence prodlužována střídáním hodnot log. \texttt{0} a \texttt{1} na \texttt{\acs{TMSC}} pinu, přičemž se uvažují poslední dva bity CP\_BODY, které nabývají požadovaných hodnot CP\_NOP. Sekvence bude ukončena odesláním hodnoty CP\_END (\texttt{00}), jak je zobrazeno na konci vyznačené části \textit{Body}.

\begin{figure}[!h]
  \begin{center}
    \includegraphics[scale=0.66]{obrazky/cJTAG_sel_sequence_cp_nop.pdf}
  \end{center}
  \caption{Průběh kontrolní části aktivační sekvence varianty \acs{JTAG} protokolu.}
	\label{fig:cJTAG_sel_cp_nop}
\end{figure}

\section{Implementace modulu pro dvouvodičový \acs{JTAG} protokol}
Navržený obvod pro detekci aktivační sekvence a převod vstupních signálů je realizován jako převodník, který je předřazen stávajícímu systému pro testování \acs{RISC-V} popsanému v kapitole \ref{sec:risc-v_dbg} a také navrženému rozšiřujícímu modulu popsanému v kapitole \ref{jtag_ap}. Navržené řešení bylo popsáno v jazyce \acs{VHDL} a jeho funkcionalita byla ověřena pomocí simulace. \hl{Verifikační prostředí pro simulaci bylo vytvořeno v jazyce SystemVerilog a simulace obvodu byla provedena v simulátoru \textit{Xcelium} od firmy \textit{Cadence}.}

\subsection{Základní popis návrhu}	\label{subsec:cJTAG_adapter}
%TODO: - nakreslit blok a v nem muxování a clk gating 
%			- zduraznit ze to nema rychlejsi hod signal
%			- výchozí stav je 4 protoze... 			
Navržený modul je obecně zobrazen na obrázku \ref{fig:cJTAG_bridge}, kde je možné vidět princip převodu mezi čtyřvodičovou variantou \acs{JTAG} rozhraní a dvou nebo čtyřvodičovou variantou na vstupně/výstupních pinech čipu. Jakmile je přijata aktivační sekvence, je obvodem nastaven signál \texttt{jtag\_2w\_en} podle vybrané varianty protokolu.

Pokud je vybrána čtyřvodičová varianta (\texttt{jtag\_2w\_en = 0}) jsou signály vstupního rozhraní pouze propojeny skrz tento modul. U signálů \texttt{\acs{TDI}}, \texttt{\acs{TMS}} a \texttt{TDO\_OEN} je toho docíleno pomocí multiplexorů. Hodinový signál \texttt{TCK(C)} je propagován dále do obvodu, prostřednictvím buňky pro "clock gating", nastavením povolovacího signálu \texttt{tckc\_en}. Vstupně/výstupní pin \texttt{\acs{TMSC}} je v případě čtyřvodičové varianty rozdělen na vstup \texttt{\acs{TMS}} a výstup \texttt{\acs{TDO}}, což je řízeno dle hodnoty výstupního signálu \texttt{jtag\_2w\_en} bloku \textit{cJTAG\_adapter}. Na obrázku je pin označen \texttt{TMS(C)}/\texttt{TDO}, čímž je naznačeno rozdělení pinů dle vybrané varianty. Výstupy modulu \texttt{tmsc\_tdo\_out} a \texttt{tmsc\_tdo\_oen} nesou tedy společnou funkci pro \texttt{\acs{TMSC}} a \texttt{\acs{TDO}} signály.

V případě výběru dvouvodičové varianty jsou multiplexory přepnuty na interní signály, které jsou generované dle formátu serializace dat na \texttt{\acs{TMSC}} pin uvedeného v podkapitole \ref{sec:oscan1}. Signály \texttt{tms\_4w\_i} a \texttt{tdi\_4w\_i} nesou hodnotu \texttt{\acs{TMSC}} navzorkovanou nástupnou hranou \texttt{\acs{TCKC}} hodinového signálu, dle aktuálního stavu stavového automatu popsaného v podkapitole \ref{subsec:oscan1_fsm}. Hodinový signál \texttt{\acs{TCK}} je propagován nastavením povolovacího signálu \texttt{tckc\_en} pouze v rámci posledního bitu (TDO) formátu OSCAN1. Aby bylo docíleno buzení \texttt{\acs{TMSC}} pinu pouze po dobu poloviny periody hodinového signálu, jak je popsáno v podkapitole \ref{subsec:oscan1_drive}, musí být výstup \texttt{tmsc\_tdo\_oen} deaktivován s nástupnou hranou hodinového signálu \texttt{\acs{TCKC}}. Pro docílení této funkcionality je na obrázku naznačena část s klopným obvodem a logickým hradlem \textit{AND}. Signál \texttt{tmsc\_oen\_i} je výstupem stavového automatu popsaného v podkapitole \ref{subsec:oscan1_fsm}.

Výchozím stavem modulu je varianta čtyřvodičové komunikace, přičemž veškeré části obvodu jsou po uvolnění asynchronního resetu \texttt{TRST\_N} nastaveny, jakoby aktivační sekvence této varianty již proběhla. Důvodem pro volbu této varianty je především zachování stávající funkcionality obvodu při použití stávajících ladících nástrojů. To je také důvodem, proč není vyžadováno odeslání aktivační sekvence, jelikož není ve stávajícím debuggeru implmentována.

Navržený obvod je taktován pouze testovacím hodinovým signálem \texttt{\acs{TCKC}}, a není tedy třeba ošetřovat přechody mezi hodinovými doménami. Toto řešení ovšem přineslo při návrhu nemožnost využít vyšší hodinové frekvence k vzorkování hodnot na signálech \acs{JTAG} rozhraní, což je vidět například na části obvodu pro detekci úvodní sekvence, jejíž schéma je vidět na obrázku \ref{fig:cJTAG_escape_circuit}, kde se pro detekci pulsů na \texttt{\acs{TMSC}} pinu využívá použití \texttt{\acs{TMSC}} jako hodinového signálu pro některé klopné obvody.

\begin{figure}[!h]
  \begin{center}
    \includegraphics[scale=0.76]{obrazky/cJTAG_bridge.pdf}
  \end{center}
  \caption{Principiální schéma navrženého modulu pro převod na čtyřvodičové \acs{JTAG} rozhraní.}
	\label{fig:cJTAG_bridge}
\end{figure}

% TODO: mezera za uvozovkami!!!
\subsection{Detekce úvodní sekvence pro přepnutí \acs{JTAG} protokolu}	\label{subsec:sel_escape_det}
Pro detekci sekvence určené pro přepínání mezi dvou a čtyřvodičovou variantou \acs{JTAG} protokolu, popsané v kapitole \ref{subsec:sel_escape}, byl navržen obvod, jehož schéma je zobrazeno na obrázku \ref{fig:cJTAG_escape_circuit}. Tato implementace obvodu pro detekci byla inspirována příkladem možné implementace uvedeném ve standardu IEEE 1149.7.

Jelikož "Escape" sekvence spočívá v generování pulsů na signálu \texttt{\acs{TMSC}} v době, kdy je signál \texttt{\acs{TCKC}} držen v log. 1. je třeba pulsy detekovat. V obvodu na obrázku se nachází 4-bitový posuvný registr, na jehož sériový vstup je přivedena hodnota log. 1. Na tento posuvný registr je přiveden hodinový signál \texttt{sr\_clk}, který je generován logickou funkcí \textit{XNOR}, tedy ekvivalencí referenčního signálu \texttt{tmsc\_fq} navzorkovaného poslední sestupnou hranou hodinového signálu \texttt{\acs{TCKC}}. \hl{Podle počtu nástupných hran vygenerovaných na vstupu {\texttt{\acs{TMSC}}}, během {\texttt{\acs{TCKC}}} signálu setrvávajícího na úrovni log. 1, je určena hodnota posuvného registru. Podle hodnoty posuvného registru je poté nastaven vždy jeden ze signálů \texttt{reset\_detect}, \texttt{select\_detect}, \texttt{deselect\_detect} nebo \texttt{custom\_detect}.} % podle počtu nástupných hran vygenerovaných na vstupu \acs{TMSC}, během \acs{TCKC} signálu setrvávajícího na úrovni log. 1. 

\hl{Neméně důležitou částí obvodu jsou dva klopné obvody generující asynchronní reset \texttt{(sr\_rst\_n)} pro posuvný registr. Reset je třeba generovat vždy v případě hodnoty log. 0 na hodinovém signálu {\texttt{\acs{TCKC}}}, protože posuvný registr musí být resetován vždy mimo Escape sekvenci.} %Jelikož reset posuvného registru je aktivní při hodnotě log. 0, je tedy generován sestupnou hranou signálu \acs{TCKC}, která překlopí první klopný obvod a hodnoty na vstupu hradla \textit{XNOR} budou rozdílné. Nástupná hrana signálu \acs{TCKC} překlopí druhý klopný obvod a vstupy hradla budou vždy totožné, čímž je reset vždy uvolněn.

Jelikož pro návrh popsaný v této kapitole je uvažována pouze detekce tří pulsů ("Selection Sequence"), je dále obvodem zpracováván pouze signál \texttt{select\_detect}. Ostatní signály příslušející jiným variantám nejsou dále využívány, ale jsou generovány dle obvodu na obrázku pro případné budoucí využití.

\begin{figure}[!h]
  \begin{center}
    \includegraphics[scale=0.7]{obrazky/cJTAG_escape_circuit.pdf}
  \end{center}
  \caption{Schéma obvodu pro detekci úvodní sekvence pro přepnutí \acs{JTAG} protokolu.}
	\label{fig:cJTAG_escape_circuit}
\end{figure}

\subsection{Zpracování sekvence pro výběr varianty \acs{JTAG} protokolu}	\label{subsec:sel_seq_det}
Za účelem zpracování sekvence pro výběr dvou či čtyřvodičové varianty \acs{JTAG} protokolu byl návrh obvodu rozdělen na dvě části pro zpracování konkrétních úseků sekvence uvedených v tabulce \ref{tab:cJTAG_sel}. První část je prostým porovnáním přijatých hodnot \acs{OAC} a \acs{EC} s konstantou. Druhá část realizuje detekci prodloužení aktivační sekvence podle přijatých hodnot v části \acs{CP}.

\subsubsection{Část obvodu pro výběr konkrétní dvou nebo čtyřvodičové varianty a rozšiřující části}
Pro porovnání prvních dvou částí aktivační sekvence byla navržena část obvodu znázorněna na obrázku \ref{fig:cJTAG_sel_oac_ec_circuit}. Tato část obvodu realizuje porovnání přijatých hodnot \acs{OAC} a \acs{EC} s podporovanými možnostmi dle tabulky \ref{tab:oac} (hodnota části \acs{EC} může nabývat pouze hodnoty \texttt{1000}).

Obvod obsahuje 8-bitový posuvný registr (\texttt{sel\_sreg}) pro příjem hodnoty sekvence a 3-bitový čítač (\texttt{sel\_cnt}) pro počítání 8 přijatých bitů. Tento registr a čítač má vstup povolující hodinový signál \texttt{CE} (Clock Enable). Signál povolující příjem hodnoty je nastaven do hodnoty log. \texttt{1} po skončení "Escape" sekvence, a to se sestupnou hranou hodinového signálu \texttt{\acs{TCKC}}, kdy je hodnota signálu \texttt{select\_detect} log. \texttt{1} vygenerovaná předchozí částí obvodu uvedené na obrázku \ref{fig:cJTAG_escape_circuit}. Jakmile čítač napočítá 8 přijatých bitů a je posuvný registr deaktivován, setrvává v něm přijatá 8-bitová hodnota. Signál CE je nastaven na hodnotu log. \texttt{0} po dokončení čítání, kdy čítač nastaví výstup TC (Terminal Count) do log. \texttt{1} a tím je na vstup prvního klopného obvodu přivedena log. \texttt{0}.

Hodnota přijatá posuvným registrem je porovnávána se dvěma konstantami, které určují zvolenou variantu \acs{JTAG} protokolu. Pokud je zvolena dvouvodičová varianta, je nastaven signál \texttt{jtag\_2w\_en} povolující funkcionalitu pro zpracování komunikace v dvouvodičové variantě. Zejména se jedná o multiplexory propojující \acs{JTAG} signály pro další zpracování a stavový automat pro  zpracování formátu OSCAN1 popsaného v podkapitole \ref{sec:oscan1}. Pokud je hodnota posuvného registru shodná s jednou z konstant a čítač napočítal 8 nasunutých bitů, je hodnota signálu \texttt{sel\_seq\_match} nastavena do log. \texttt{1} a je tak aktivován obvod pro zpracování poslední části aktivační sekvence, jehož schéma je uvedeno na obrázku \ref{fig:cJTAG_sel_cp_circuit}. 

\begin{figure}[H]s
  \begin{center}
    \includegraphics[scale=0.78]{obrazky/cJTAG_sel_oac_ec_circuit.pdf}
  \end{center}
  \caption{Schéma obvodu pro zpracování sekvence pro výběr varianty \acs{JTAG} protokolu.}
	\label{fig:cJTAG_sel_oac_ec_circuit}
\end{figure}

\subsubsection{Část obvodu pro zpracování kontrolní části aktivační sekvence}
Pro poslední část aktivační sekvence byl navržen a implementován obvod zobrazený na obrázku \ref{fig:cJTAG_sel_cp_circuit}. Popsaný obvod podporuje prodloužení sekvence při příjmu hodnoty CP\_NOP a následné ukončení sekvence hodnotou CP\_END, které jsou definované v tabulce \ref{tab:cp_body}. % Nejkratší možnou variantou kontrolní části aktivační sekvence je 

Obvod na obrázku je aktivován jakmile předcházející část obvodu uvedená na obrázku \ref{fig:cJTAG_sel_oac_ec_circuit} vyhodnotí shodu hodnoty s jednou z konstant pro výběr varianty \acs{JTAG} protokolu (nastaví se signál \texttt{sel\_seq\_match}). V tomto hodinovém taktu je povolen 2 a 3-bitový posuvný registr signálem \texttt{CE}. Dvoubitový posuvný registr, jehož vstupem je signál \texttt{\acs{TMSC}}, slouží k vyhodnocení hodnoty CP\_END. Jakmile je hodnota tohoto posuvného registru \texttt{00}, což odpovídá hodnotě CP\_END ukončující sekvenci, je výstupem logické funkce \textit{NOR} log. \texttt{1}.

Tříbitový posuvný registr slouží k povolení detekce hodnoty CP\_END nejdříve po třech hodinových taktech \texttt{\acs{TCKC}}, protože nejkratší možná délka této části sekvence jsou 4 bity. Platná hodnota CP\_BODY je tudíž nasunuta do 2-bitového posuvného registru nejdříve po třech taktech \texttt{\acs{TCKC}}, dle průběhu na obrázku \ref{fig:cJTAG_sel_cp_nop}. Tento posuvný registr je vždy před aktivační sekvencí synchronně resetován signálem \texttt{sel\_rst}, který je generován vždy po skončení úvodní sekvence popsané v kapitole \ref{subsec:sel_escape_det}. Tuto deaktivaci po dobu tří taktů by bylo možné realizovat také čítačem od 0 do 3. Tato varianta řešení by ušetřila jeden klopný obvod, ale přidala jedno logické hradlo. Z hlediska velikosti čipu je tato úprava zanedbatelná, a proto bylo v návrhu ponecháno prvotní řešení.

V případě, že jsou výše popsané podmínky pro ukončení aktivační sekvence splněny, je signálem \texttt{CE} povoleno překlopení posledního klopného obvodu s následující nástupnou hranou \texttt{\acs{TCKC}}. Signál \texttt{online\_i} je tedy nastaven v rámci hodinového taktu, kdy je přijímán poslední bit aktivační sekvence označený \textit{Postamble}, dle formátu kontrolní části aktivační sekvence uvedeného v tabulce \ref{tab:cJTAG_sel_cp}. Signál \texttt{CE} všech klopných obvodů je opět deaktivován po dokončení aktivace, a tím je zamezeno dalšímu překlápění. Signál \texttt{online\_i} je využíván pro povolení funkcionality stavového automatu pro serializaci dat v případě dvouvodičové varianty protokolu a také povoluje propagaci hodinového signálu \texttt{\acs{TCKC}} pro další části obvodu. 

Logické hradlo na výstupu registru pro nastavení \texttt{online\_i} signálu využívá resetovací signál \texttt{sel\_rst} k zapsání log. 0 na signál \texttt{online\_i}, ihned se sestupnou hranou \texttt{\acs{TCKC}} ukončující "Escape" sekvenci. Bez této logické funkce docházelo v případě přepínání z dvouvodičové varianty k propagaci jedné periody hodinového signálu \texttt{\acs{TCKC}} i po "Escape" sekvenci, proto bylo třeba zaručit deaktivaci online režimu po celou dobu aktivační sekvence.

\begin{figure}[H]
  \begin{center}
    \includegraphics[scale=0.78]{obrazky/cJTAG_sel_cp_circuit.pdf}
  \end{center}
  \caption{Schéma obvodu pro zpracování kontrolní části aktivační sekvence.}
	\label{fig:cJTAG_sel_cp_circuit}
\end{figure}

\subsection{Stavový automat pro serializaci signálů čtyřvodičového rozhraní}	\label{subsec:oscan1_fsm}
Zpracování serializovaných hodnot signálů odesílaných přes \texttt{\acs{TMSC}} pin dle formátu OSCAN1 popsaného v kapitole \ref{sec:oscan1} je realizováno pomocí jednoduchého stavového automatu, jehož stavový diagram je uvedený na obrázku \ref{fig:cJTAG_oscan1_fsm}. Stavový automat svůj stav mění se sestupnou hranou hodinového signálu \texttt{\acs{TCKC}}, protože hodnota \texttt{\acs{TMSC}} signálu se mění vždy se sestupnou hranou a na nástupnou je vzorkována.

Výchozím stavem je stav \textbf{S\_IDLE}, ve kterém automat setrvává a je do něj vždy navrácen pokud nejsou signály \texttt{online\_i} a \texttt{jtag\_2w\_en} ve stavu log. \texttt{1}. Tímto je zajištěna nečinnost automatu v průběhu aktivační sekvence a také, když je zvolena čtyřvodičová varianta \acs{JTAG} protokolu. Jakmile je aktivována dvouvodičová varianta přechází stavový automat do stavu \textbf{S\_nTDI}. V tomto stavu je navzorkována hodnota TDI na \texttt{\acs{TMSC}} pinu, přičemž hodnota je negována dle specifikace formátu OSCAN1. S další sestupnou hranou hodinového signálu přechází automat do stavu \textbf{S\_TMS}, kdy se vzorkuje hodnota TMS. Následuje přechod do posledního stavu \textbf{S\_TDO}, který povoluje signálem \texttt{tck\_en\_2w} propagaci hodinového signálu \texttt{\acs{TCKC}}. Dále je také na interní signál \texttt{tmsc\_oen\_i}, povolující výstup pinu, přiřazena hodnota z převedeného čtyřvodičového rozhraní. Stavy \textbf{S\_nTDI}, \textbf{S\_TMS} a \textbf{S\_TDO} se cyklicky opakují, dokud není zahájena další aktivační sekvence.

\begin{figure}[!h]
  \begin{center}
    \includegraphics[scale=0.95]{obrazky/cJTAG_oscan1_fsm.pdf}
  \end{center}
  \caption{Stavový diagram stavového automatu pro implementaci serializace signálů na pinu \acs{TMSC}.}
	\label{fig:cJTAG_oscan1_fsm}
\end{figure}

\section{Zhodnocení návrhu modulu pro dvouvodičový \acs{JTAG} protokol}
V této kapitole byl popsán formát přenosu dat přes dvouvodičové \acs{JTAG} rozhraní, definice aktivační sekvence pro deaktivování probíhající komunikace a výběr jednotlivých variant \acs{JTAG} rozhraní, definovaná standardem IEEE 1149.7. V rámci popisu možností byly zdůrazněny vlastnosti implementované v rámci této práce, přičemž standard IEEE 1149.7. definuje mnoho dalších možností. Dle standardu by byl návrh celého systému výrazně komplexnější a velmi náročný na implementaci. \hl{Celkový systém definovaný standardem by bylo navíc zbytečné realizovat, protože takový systém již existuje a jeho pokročilé funkce nejsou v rámci návrhu uvedeného v této práci potřebné.}

Pro implementaci byla tedy zvolena podpora dvouvodičového \acs{JTAG} rozhraní ve formátu OSCAN1 a možnost přepnou mezi dvou a čtyřvodičovým rozhraním. Zapojení do hvězdicové topologie, resetování v rámci deaktivační sekvence nebo pomocí hodnoty kontrolní části aktivační sekvence, jiné formáty serializace dat a další funkce definované standardem nebyly implementovány z důvodu jejich nepotřebnosti. \cite{IEEE_1149-7}

Navržený modul byl implementován jako převodník \acs{JTAG} rozhraní na původní čtyřvodičovou variantu a je taktován pouze hodinovým signálem \texttt{\acs{TCKC}} generovaným debuggerem. Toto řešení vyžadovalo poměrně malý zásah do stávajícího obvodu a díky výchozímu čtyřvodičovému rozhraní nemá rozšíření systému vliv na stávající funkcionalitu.

%Vlozeni souboru s třetí kapitolou (Advanced protocol)
%\chapter{Návrh a implementace rozšiřujícího modulu pro přístup na systémovou sběrnici pomocí JTAG protokolu}	\label{jtag_ap}
\chapter{Popis modulu pro přístup na systémovou sběrnici pomocí JTAG protokolu}	\label{jtag_ap}
Pro přístup na systémovou sběrnici procesoru jsou navrženy dvě varianty komunikačního protokolu přenášeného pomocí JTAG rozhraní. Obě varianty protokolu jsou popsány pro čtyřvodičovou variantu \acs{JTAG} komunikace mezi debuggerem (aplikační prostředí pro testování obvodů) a \acs{JTAG} systémem (navržený systém pro testování). \hl{Navržený protokol bude sloužit jako vedlejší možnost přístupu na systémovou sběrnici, která je časově optimálnější.}

\section{Návrh optimalizovaného protokolu pro přístup na systémovou sběrnici}	\label{sec:protokoly}
%\section{Návrh časově optimalizovaného protokolu pro přístup na systémovou sběrnici}	\label{sec:protokoly}
Pro přístup na systémovou sběrnici byly navrženy dvě varianty protokolu využívajícího JTAG rozhraní. Obě varianty se od sebe liší především ve způsobu signalizace nepřístupnosti systémové sběrnice z důvodu probíhající operace na sběrnici. První variantou je využití obdobného principu jak u stávajícího řešení. Tedy pokud se operace na systémovou sběrnici nestihne provést do okamžiku dalšího požadavku, je skrze JTAG komunikaci odeslán zpět status \textit{busy}. Druhou variantou je využití výstupního \texttt{\acs{TDO}} portu k signalizaci dokončení operace na systémové sběrnici dynamicky.

\subsection{Zkrácení adresy a přenášených dat}
V případě obou variant protokolu je adresa zkrácena na 21 bitů namísto původních 32 bitů. Důvodem pro zkrácení adresy je velikost adresního prostoru celého systému, ve kterém není z adresy využito 11 bitů. Z tohoto důvodu je výhodné nevyužívané bity adresy neodesílat a adresu korektně rozšířit na původních 32 bitů až po jejím příjmu. Tato optimalizace ušetří 11 taktů testovacího hodinového signálu \texttt{\acs{TCK}} pro každý přístup na sběrnici. V případě většího adresního prostoru systému by se pouze odesílaná adresa rozšířila.

Délka přenášených dat může být volena jako \textit{Word} (32-bitů), \textit{Half-Word} (16-bitů) a \textit{Byte} (8-bitů). Obvyklým způsobem je přístup na sběrnici v celé šířce 32 bitů. Pro specifické požadavky je možné využít také kratších variant. Příkladem může být potřeba přepsání logické hodnoty určitého bitu registru, či určité skupiny bitů. Délka přenesených dat je dekódována podle adresy na signál \texttt{be} systémové sběrnice.

\subsection{Varianta protokolu Busy-wait} \label{subsec:busy-wait}
Varianta protokolu \textbf{Busy-wait} je založena na principu signalizace probíhající operace na sběrnici před odesíláním požadovaných dat. Tato informace je předávána debuggeru jako 3-bitová hodnota \textit{status}, prostřednictvím signálu \texttt{\acs{TDO}}. \textit{Status} může nabývat hodnot dle tabulky \ref{tab:status_vals}, kde je vidět, že je využitý pouze jeden bit ze tří. Důvodem je to, že pole \textit{status} je vysíláno současně s hodnotou \textit{op}, která je vysílána debuggerem, a ta již 3-bitová být musí. Tato varianta protokolu využívá toho, že debugger prochází řídicím stavovým automatem (\acs{TAP}) po spuštění požadavku na systémové sběrnici vždy přes stav \textit{Run-Test/Idle}, kde čeká po několik taktů hodinového signálu \texttt{\acs{TCK}}. Tato prodleva dává možnost systémové sběrnici požadavek dokončit. V případě nedokončeného požadavku je při dalším požadavku signalizováno pomocí hodnoty \textit{status} = 1 (\textit{busy}), že nebude akceptován. \hl{Oproti přístupu na systémovou sběrnici pomocí stávajícího testovacího systému není třeba tento stav nijak resetovat.}

\begin{table}[!h]
  \caption{Tabulka status hodnot.}
  \begin{center}
  	\small
	  \begin{tabular}{!{\vrule width 1.2pt}M{1.5cm}|M{1.8cm}|M{9cm}!{\vrule width 1.2pt}}
	    \noalign{\hrule height 1.2pt}
	    Status & Význam & Popis\\
	    \noalign{\hrule height 1.2pt}
	    0 & no\_error & Systémová sběrnice je připravena k přístupu\\
			\hline
			1 & busy & Na systémové sběrnici probíhá komunikace\\
			\hline
			2 - 7 & res & Rezervováno pro budoucí využití\\
			\hline
			\noalign{\hrule height 1.2pt}
		\end{tabular}
  \end{center}
	\label{tab:status_vals}
\end{table}

Význam hodnot \textit{op} (operace) je popsán v tabulce \ref{tab:op_vals}. Výběr, zda bude následovat zápis nebo čtení, je určen nejnižším bitem pole \textit{op}, kde hodnota log. \texttt{0} určuje čtení a log. \texttt{1} zápis, což je shodné s významem signálu \texttt{we} systémové sběrnice popsaného v tabulce \ref{tab:ri5cy_bus}. Hodnota \textit{op} určuje také délku přenášených dat, kterou je tak možné kdykoliv změnit.

\begin{table}[!h]
  \caption{Tabulka možných hodnot operace.}
  \begin{center}
  	\small
	  \begin{tabular}{!{\vrule width 1.2pt}M{0.7cm}|M{2.5cm}|M{10.4cm}!{\vrule width 1.2pt}}
	    \noalign{\hrule height 1.2pt}
	    \textit{op} & Význam & Popis\\
	    \noalign{\hrule height 1.2pt}
	    0 & byte / R & Čtení bajtu (následuje odesílání adresy)\\
			\hline	    
			1 & byte / W & Zápis bajtu (následuje odesílání adresy a zapisovaných dat)\\
			\hline
			2 & half-word / R & Čtení půlslova (následuje odesílání adresy)\\
			\hline	    
			3 & half-word / W & Zápis půlslova (následuje odesílání adresy a zapisovaných dat)\\
			\hline
			4 & word / R & Čtení slova (následuje odesílání adresy)\\
			\hline	   
			5 & word / W & Zápis slova (následuje odesílání adresy a zapisovaných dat)\\
			\hline
			6 & data & Uvozuje přenos vyčtených dat (pro Busy-mode). Rezervováno (pro Dynamic-mode)\\
			\hline
			7 & cmd & Uvozuje změnu módu\\
			\hline
			\noalign{\hrule height 1.2pt}
		\end{tabular}
  \end{center}
	\label{tab:op_vals}
\end{table}

%\subsubsection{Busy-wait mód - zápis} 
\subsubsection{Zápis na systémovou sběrnici prostřednictvím Busy-wait módu} 
Časový průběh JTAG komunikace pro zápis na systémovou sběrnici je zobrazen na obrázku \ref{fig:busy_w}. Po přechodu řídicího stavového automatu do stavu \textit{Shift-DR} je nejdříve odeslána 3-bitová hodnota \textit{op}, která může v případě zápisu nabývat hodnot 1, 3 nebo 5, podle délky zapisovaných dat. Následuje odesílání 21-bitové adresy, na kterou se budou data zapisovat. Po odeslání adresy následuje přenos požadovaných dat k zápisu, jejichž délka je určena hodnotou \textit{op}. Pro zápis na další adresu je třeba provést průchod řídicím stavovým automatem. Pokud je navrácena hodnota \textit{status} = 1 (\textit{busy}), nebudou následující data zapsána, protože se musí nejdříve dokončit probíhající zápis na systémovou sběrnici. Debugger by v tomto případě měl požadavek pro zapsání dat opakovat, dokud nebude navrácen \textit{status} = 0 (\textit{no\_err}).

\begin{figure}[H]
  \begin{center}
    \includegraphics[scale=0.575]{obrazky/busy_w.pdf}
  \end{center}
  \caption{Průběh zápisu na systémovou sběrnici v Busy-wait módu.}
	\label{fig:busy_w}
\end{figure}

%\subsubsection{Busy-wait mód - čtení} 
\subsubsection{Čtení ze systémové sběrnice prostřednictvím Busy-wait módu} 
Časový průběh JTAG komunikace v případě čtení ze systémové sběrnice je zobrazen na obrázku \ref{fig:busy_r}. Nejprve je odeslána hodnota \textit{op} = 0, 2 nebo 4, která dle tabulky \ref{tab:op_vals} uvozuje odesílání adresy, ze které bude vyčtena hodnota registru a délka čtených dat. Jakmile je adresa odeslána, spustí se čtení ze systémové sběrnice. Debugger projde pomocí signálu \texttt{\acs{TMS}} řídicím stavovým automatem dle stavového diagramu na obrázku \ref{fig:tap_controller} zpět do stavu \textit{Shift-DR} a odešle hodnotu \textit{op} = 6, kterou je požadováno odeslání dat vyčtených z požadované adresy zpět prostřednictvím \texttt{\acs{TDO}} pinu. \hl{Délka odesílaných dat je určena předchozí hodnotou \textit{op} odesílanou v rámci přenosu adresy.} V případě, že je navrácen \textit{status} = 1 (busy), odpovídající nedokončenému čtení ze systémové sběrnice, nejsou následující data platná. V dalším průchodu má debugger možnost žádat data znovu (\textit{op} = 6), nebo zažádat o vyčtení z jiné adresy.

\begin{figure}[H]
  \begin{center}
    \includegraphics[scale=0.68]{obrazky/busy_r.pdf}
  \end{center}
  \caption{Průběh čtení ze systémové sběrnice v Busy-wait módu.}
	\label{fig:busy_r}
\end{figure}

\subsection{Varianta protokolu Dynamic-wait}	\label{subsec:dyn-wait}
Varianta protokolu \textbf{Dynamic-wait} je založena na principu signalizace probíhající operace na sběrnici prostřednictvím \texttt{\acs{TDO}} signálu v rámci stavu \textit{Shift-DR} řídicího stavového automatu. V případě probíhající operace na systémové sběrnici je tato informace signalizována úrovní log. \texttt{0} na \texttt{\acs{TDO}} pinu v definovaném úseku komunikace, tedy po spuštění požadavku na systémovou sběrnici. Jakmile je požadavek na systémovou sběrnici obsloužen dochází k nastavení \texttt{\acs{TDO}} pinu na úroveň log. \texttt{1} po jeden takt hodinového signálu \texttt{\acs{TCK}}. Poté následuje další komunikace, jak je dále popsáno pro zápis a čtení.

Použití tohoto komunikačního módu má velkou výhodu z hlediska časové optimalizace přenosu dat. Tato výhoda spočívá v nejkratší možné prodlevě způsobené obsluhou požadavku systémovou sběrnicí, protože je tato informace signalizována dynamicky ihned po dokončení požadavku. Další aspekt, že je tento způsob časově optimálnější, spočívá v setrvávání řídicího stavového automatu ve stavu \textit{Shift-DR} pro přístup k libovolnému počtu registrů. Při komunikaci se tak neztrácí čas průchodem stavového automatu přes stav \textit{Run-Test/Idle}. Další výhodou je možnost libovolně střídat zápis a čtení v rámci jednoho přechodu do stavu \textit{Shift-DR}. Tato vlastnost může výrazně časově optimalizovat případy komunikace, kdy je zapotřebí přečíst hodnotu registru a změnit v něm hodnoty pouze některých bitů.

K výběru, zda jde o zápis nebo čtení a délky odesílaných dat, je využita hodnota \textit{op} (operace), stějně jako v případě \textbf{Busy-wait} módu popsaného v podkapitole \ref{subsec:busy-wait}. Význam hodnot \textit{op} je popsán v tabulce \ref{tab:op_vals}. Rozdílem je nevyužití hodnoty \textit{op} = 6, protože v dynamickém režimu není potřeba, a zůstává tak volná pro budoucí využití.

%\subsubsection{Dynamic-wait mód - zápis} 
\subsubsection{Zápis na systémovou sběrnici prostřednictvím Dynamic-wait módu} 
Na obrázku \ref{fig:wait_w} je znázorněn časový průběh \acs{JTAG} komunikace popisující způsob zápisu na systémovou sběrnici v dynamickém  režimu. Po přechodu řídicího stavového automatu do stavu \textit{Shift-DR} je odeslána 3-bitová hodnota \textit{op}, která může nabývat hodnot 1, 3 nebo 5, podle délky zapisovaných dat, stejně jako při zápisu v \textbf{Busy-wait} módu. Následuje odeslání 21-bitové adresy, na kterou je požadováno data zapsat a ta se poté odesílají dle zvolené délky. Jakmile jsou data odeslána, následuje fáze zpracování zápisu systémovou sběrnicí, kdy je \acs{JTAG} systémem vystavena hodnota log. \texttt{0} na signálu \texttt{\acs{TDO}} a je držena, dokud zápis na systémové sběrnici není proveden. Po dokončení zápisu \acs{JTAG} systém nastaví signál \texttt{\acs{TDO}} na úroveň log. \texttt{1} po dobu jednoho taktu. Po takto zapsaných datech může být proveden zápis na další adresu okamžitě bez nutnosti průchodu stavovým automatem, jak je uvedeno na obrázku nebo může být přenos ukončen přechodem do stavu \textit{Exit1-DR}, což je zobrazeno na konci průběhu. V případě ukončení přenosu by měla být poslední hodnota na \texttt{\acs{TDO}} nastavena na hodnotu log. \texttt{0}. Důvodem je, že \acs{JTAG} systém po přechodu ze stavu \textit{Shift-DR} získává informaci právě v tomto taktu, a tudíž by pro případ pokračování v komunikaci hodnota log. \texttt{1} v tomto taktu znamenala \textit{status} = 1 (\textit{busy}), který v tomto dynamickém módu nemůže nastat.

\begin{figure}[!h]
  \begin{center}
    \includegraphics[scale=0.56]{obrazky/wait_w.pdf}
  \end{center}
  \caption{Průběh zápisu na systémovou sběrnici v Dynamic-wait módu.}
	\label{fig:wait_w}
\end{figure}

%\subsubsection{Dynamic-wait mód - čtení} 
\subsubsection{Čtení ze systémové sběrnice prostřednictvím Dynamic-wait módu}
Časový průběh JTAG komunikace v případě čtení v dynamickém režimu je zobrazen na obrázku \ref{fig:wait_r}. Komunikace začíná odesláním hodnoty operace, která může nabývat hodnot \textit{op} = 0, 2 nebo 4, stejně jako pro čtení v \textbf{Busy-wait} módu. Následuje odesílání 21-bitové adresy. Po odeslání adresy je spuštěn požadavek čtení ze systémové sběrnice a následuje signalizace dokončení čtení. Jakmile je operace dokončena \acs{JTAG} systém nastaví signál \texttt{\acs{TDO}} na úroveň log. 1 a začne vysílat přečtená data. Po odeslání dat může debugger pokračovat v dalším čtením či zápisem s možností volby délky dat dle hodnoty \textit{op}, nebo komunikaci ukončit. 

\begin{figure}[!h]
  \begin{center}
    \includegraphics[scale=0.6]{obrazky/wait_r.pdf}
  \end{center}
  \caption{Průběh čtení ze systémové sběrnice v Dynamic-wait módu.}
	\label{fig:wait_r}
\end{figure}

\section{Zhodnocení návrhu optimalizovaného protokolu pro přístup na systémovou sběrnici}
V této kapitole byly popsány dvě varianty protokolu pro přístup na systémovou sběrnici pomocí \acs{JTAG} rozhraní. Obě varianty jsou časově optimalizované, oproti přístupu na systémovou sběrnici pomocí původního testovacího systému popsaného v podkapitole \ref{sec:risc-v_dbg}.

První varianta je inspirována řešením původního testovacího systému, kde se využívá signalizace probíhajícího přístupu na systémovou sběrnici prostřednictvím \textit{status} kódu. Tato varianta je oproti původnímu přístupu časově optimálnější, protože není třeba procházet řídicím stavovým automatem v takové míře. \hl{Počet průchodů je snížen zejména, protože adresa registru a data se odesílají přímo, zatímco v případě použití původního přístupu je třeba zapisovat adresu registru a data postupně do registrů testovacího systému}.

Druhá varianta protokolu je nejvíce časově optimální, protože využívá signalizace probíhajícího přístupu na systémovou sběrnici prostřednictvím \texttt{\acs{TDO}}. Díky této vlastnosti také není třeba procházet řídicím stavovým automatem během komunikace a po zpracování požadavku na systémové sběrnici nevzniká výrazná prodleva před započetím další komunikace.

%\section{Implementace}

%\subsection{Busy-wait mód} 

%\begin{figure}[!h]
%  \begin{center}
%    \includegraphics[scale=1.5]{obrazky/busy_wait_fsm.pdf}
%  \end{center}
%  \caption{Stavový diagram popisující část hlavního stavového automatu pro busy-wait režim.}
%	\label{fig:busy_wait_fsm}
%\end{figure}

%%% Vložení souboru 'text/cile.tex' s úvodem
%\chapter*{Cíle práce}
\phantomsection
\addcontentsline{toc}{chapter}{Cíle práce}

Konkrétní specifikace cílů, které má autor v~práci vyřešit.
Tato kapitola je \emph{volitelná} -- pokud váš studijní program nevyžaduje zvláštní kapitolu s cíli,
cíle specifikujte v~rámci Úvodu.

%%% Vložení souboru 'text/reseni' s popisem řešení práce
% (rozdělte na více souborů či kapitol, pokud je vhodné)
%\chapter{Teoretická část studentské práce}

Teoretické zázemí studentské práce vhodně rozdělené do částí.

(Struktura navržená v~této šabloně je nejhrubší možná, po konzultaci s~vedoucím je vhodné zvolit přiléhavější.)


%%% Vložení souboru 'text/vysledky' s popisem vysledků práce
% (rozdělte na více souborů či kapitol, pokud je vhodné)
%\chapter{Výsledky studentské práce}

Praktická část a výsledky studentské práce vhodně rozdělené do částí.

\section{Programové řešení}
Lorem ipsum dolor sit amet, consectetuer adipiscing elit. Nulla pulvinar eleifend sem. Integer in sapien. Etiam sapien elit, consequat eget, tristique non, venenatis quis, ante. In laoreet, magna id viverra tincidunt, sem odio bibendum justo, vel imperdiet sapien wisi sed libero. Phasellus enim erat, vestibulum vel, aliquam a, posuere eu, velit. Aliquam erat volutpat. Nullam faucibus mi quis velit \cite{sr72/2017}.

\section{Výsledky měření}
Fusce tellus odio, dapibus id fermentum quis, suscipit id erat. Fusce tellus. Morbi scelerisque luctus velit. In laoreet, magna id viverra tincidunt, sem odio bibendum justo, vel imperdiet sapien wisi sed libero. Quisque porta. Fusce suscipit libero eget elit. Nulla non lectus sed nisl molestie malesuada. Phasellus faucibus molestie nisl. Integer vulputate sem a nibh rutrum consequat. Proin mattis lacinia justo. Phasellus et lorem id felis nonummy placerat. Etiam ligula pede, sagittis quis, interdum ultricies, scelerisque eu. Cras elementum. Aenean placerat. Donec ipsum massa, ullamcorper in, auctor et, scelerisque sed, est. Aliquam ante. Integer imperdiet lectus quis justo. Vivamus ac leo pretium faucibus. Nullam faucibus mi quis velit.

\subsection{Etiam quis quam}
Neque porro quisquam est, qui dolorem ipsum quia dolor sit amet, consectetur, adipisci velit, sed quia non numquam eius modi tempora incidunt ut labore et dolore magnam aliquam quaerat voluptatem. Aliquam erat volutpat. Lorem ipsum dolor sit amet, consectetuer adipiscing elit \cite{sr72/2017,pravidla}. Nunc auctor. Neque porro quisquam est, qui dolorem ipsum quia dolor sit amet, consectetur, adipisci velit, sed quia non numquam eius modi tempora incidunt ut labore et dolore magnam aliquam quaerat voluptatem. Maecenas lorem. Maecenas libero. In laoreet, magna id viverra tincidunt, sem odio bibendum justo, vel imperdiet sapien wisi sed libero. Nullam rhoncus aliquam metus.

\subsubsection{Integer rutrum orci vestibulum}
Integer rutrum, orci vestibulum ullamcorper ultricies, lacus quam ultricies odio, vitae placerat pede sem sit amet enim. Ut enim ad minim veniam, quis nostrud exercitation ullamco laboris nisi ut aliquip ex ea commodo consequat. Fusce tellus odio, dapibus id fermentum quis, suscipit id erat. Nullam eget nisl. Nunc auctor. Etiam dui sem, fermentum vitae, sagittis id, malesuada in, quam. Fusce dui leo, imperdiet in, aliquam sit amet, feugiat eu, orci. Curabitur vitae diam non enim vestibulum interdum. Aliquam erat volutpat. Pellentesque sapien. Phasellus enim erat, vestibulum vel, aliquam a, posuere eu, velit.

\subsubsection{Eger rutrum orci westibulum}
Fusce dui leo, imperdiet in, aliquam sit amet, feugiat eu, orci. Maecenas aliquet accumsan leo. Aliquam ornare wisi eu metus. Cum sociis natoque penatibus et magnis dis parturient montes, nascetur ridiculus mus. Aliquam erat volutpat. Donec iaculis gravida nulla. Sed elit dui, pellentesque a, faucibus vel, interdum nec, diam. Temporibus autem quibusdam et aut officiis debitis aut rerum necessitatibus saepe eveniet ut et voluptates repudiandae sint et molestiae non recusandae. Nulla non arcu lacinia neque faucibus fringilla. Phasellus enim erat, vestibulum vel, aliquam a, posuere eu, velit. Praesent vitae arcu tempor neque lacinia pretium
\cite{Walter1999,Svacina1999IEEE,RajmicSysel2002}.

Aliquam erat volutpat. Quisque porta. Integer imperdiet lectus quis justo. Nullam justo enim, consectetuer nec, ullamcorper ac, vestibulum in, elit. Nullam faucibus mi quis velit. Fusce tellus. Fusce consectetuer risus a nunc. Cras pede libero, dapibus nec, pretium sit amet, tempor quis. Morbi imperdiet, mauris ac auctor dictum, nisl ligula egestas nulla, et sollicitudin sem purus in lacus
\cite{CSN_ISO_690-2022,CSN_ISO_7144-1997,CSN_ISO_31-11}.
Mauris elementum mauris vitae tortor. Neque porro quisquam est, qui dolorem ipsum quia dolor sit amet, consectetur, adipisci velit, sed quia non numquam eius modi tempora incidunt ut labore et dolore magnam aliquam quaerat voluptatem. Quisque porta. Integer vulputate sem a nibh rutrum consequat. Nulla pulvinar eleifend sem. Praesent id justo in neque elementum ultrices \cite{Farkasova23:CSNISO6902022komentar}.

Fusce suscipit libero eget elit. Integer vulputate sem a nibh rutrum consequat. Aliquam erat volutpat. Etiam neque. Nulla turpis magna, cursus sit amet, suscipit a, interdum id, felis. Nullam rhoncus aliquam metus. Etiam dui sem, fermentum vitae, sagittis id, malesuada in, quam. Nunc auctor. Nunc dapibus tortor vel mi dapibus sollicitudin. Praesent in mauris eu tortor porttitor accumsan. Nulla non arcu lacinia neque faucibus fringilla. Nullam lectus justo, vulputate eget mollis sed, tempor sed magna. Maecenas lorem. Aenean placerat. Donec vitae arcu. Maecenas lorem. Donec iaculis gravida nulla. Nulla non lectus sed nisl molestie malesuada.

Duis pulvinar. Nulla est. Duis condimentum augue id magna semper rutrum. Integer pellentesque quam vel velit. Aliquam ante. Nulla quis diam. Proin mattis lacinia justo. Aenean fermentum risus id tortor. Nunc auctor. Nullam justo enim, consectetuer nec, ullamcorper ac, vestibulum in, elit. In dapibus augue non sapien. Etiam bibendum elit eget erat. In sem justo, commodo ut, suscipit at, pharetra vitae, orci. Maecenas libero.

Nulla non lectus sed nisl molestie malesuada. Donec vitae arcu. Aenean fermentum risus id tortor. Praesent in mauris eu tortor porttitor accumsan. Nulla pulvinar eleifend sem. Duis viverra diam non justo. Integer imperdiet lectus quis justo. Pellentesque habitant morbi tristique senectus et netus et malesuada fames ac turpis egestas. In rutrum. Excepteur sint occaecat cupidatat non proident, sunt in culpa qui officia deserunt mollit anim id est laborum. Nulla non lectus sed nisl molestie malesuada. Aliquam erat volutpat. Mauris tincidunt sem sed arcu. Duis bibendum, lectus ut viverra rhoncus, dolor nunc faucibus libero, eget facilisis enim ipsum id lacus. Fusce tellus odio, dapibus id fermentum quis, suscipit id erat. In enim a arcu imperdiet malesuada. Nulla non lectus sed nisl molestie malesuada. Proin mattis lacinia justo.

Aliquam in lorem sit amet leo accumsan lacinia. Cum sociis natoque penatibus et magnis dis parturient montes, nascetur ridiculus mus. Duis sapien nunc, commodo et, interdum suscipit, sollicitudin et, dolor. Suspendisse sagittis ultrices augue. Nullam lectus justo, vulputate eget mollis sed, tempor sed magna. In convallis. Praesent id justo in neque elementum ultrices. Neque porro quisquam est, qui dolorem ipsum quia dolor sit amet, consectetur, adipisci velit, sed quia non numquam eius modi tempora incidunt ut labore et dolore magnam aliquam quaerat voluptatem.

Pellentesque pretium lectus id turpis. Nemo enim ipsam voluptatem quia voluptas sit aspernatur aut odit aut fugit, sed quia consequuntur magni dolores eos qui ratione voluptatem sequi nesciunt. Curabitur ligula sapien, pulvinar a vestibulum quis, facilisis vel sapien. Praesent dapibus. Sed elit dui, pellentesque a, faucibus vel, interdum nec, diam. Duis viverra diam non justo. Duis ante orci, molestie vitae vehicula venenatis, tincidunt ac pede. Phasellus rhoncus. Maecenas fermentum, sem in pharetra pellentesque, velit turpis volutpat ante, in pharetra metus odio a lectus. Proin pede metus, vulputate nec, fermentum fringilla, vehicula vitae, justo. Fusce aliquam vestibulum ipsum. Nullam at arcu a est sollicitudin euismod.

%Aliquam ante. Phasellus faucibus molestie nisl. Etiam ligula pede, sagittis quis, interdum ultricies, scelerisque eu. Morbi leo mi, nonummy eget tristique non, rhoncus non leo. Cum sociis natoque penatibus et magnis dis parturient montes, nascetur ridiculus mus. Morbi scelerisque luctus velit. Curabitur bibendum justo non orci. Donec quis nibh at felis congue commodo. Nullam faucibus mi quis velit. Aenean id metus id velit ullamcorper pulvinar. Pellentesque sapien. Fusce nibh. Vestibulum fermentum tortor id mi. Nullam eget nisl. Praesent vitae arcu tempor neque lacinia pretium. Proin in tellus sit amet nibh dignissim sagittis. Donec quis nibh at felis congue commodo.
%
%Nam quis nulla. Proin in tellus sit amet nibh dignissim sagittis. Nullam dapibus fermentum ipsum. Curabitur ligula sapien, pulvinar a vestibulum quis, facilisis vel sapien. Nam libero tempore, cum soluta nobis est eligendi optio cumque nihil impedit quo minus id quod maxime placeat facere possimus, omnis voluptas assumenda est, omnis dolor repellendus. Vivamus ac leo pretium faucibus. Nunc tincidunt ante vitae massa. Maecenas sollicitudin. Ut tempus purus at lorem. Nullam lectus justo, vulputate eget mollis sed, tempor sed magna. Fusce consectetuer risus a nunc. Etiam quis quam.
%
%Donec quis nibh at felis congue commodo. Sed vel lectus. Donec odio tempus molestie, porttitor ut, iaculis quis, sem. Nullam feugiat, turpis at pulvinar vulputate, erat libero tristique tellus, nec bibendum odio risus sit amet ante. Sed elit dui, pellentesque a, faucibus vel, interdum nec, diam. Cras elementum. Sed vel lectus. Donec odio tempus molestie, porttitor ut, iaculis quis, sem. Etiam neque. Integer tempor. Vivamus porttitor turpis ac leo. Nulla non arcu lacinia neque faucibus fringilla.
%
%Etiam posuere lacus quis dolor. Nemo enim ipsam voluptatem quia voluptas sit aspernatur aut odit aut fugit, sed quia consequuntur magni dolores eos qui ratione voluptatem sequi nesciunt. Nullam faucibus mi quis velit. Cum sociis natoque penatibus et magnis dis parturient montes, nascetur ridiculus mus. Phasellus faucibus molestie nisl. Maecenas ipsum velit, consectetuer eu lobortis ut, dictum at dui. Maecenas aliquet accumsan leo. Pellentesque ipsum. Donec vitae arcu. Suspendisse nisl. Morbi imperdiet, mauris ac auctor dictum, nisl ligula egestas nulla, et sollicitudin sem purus in lacus. Pellentesque ipsum. Ut enim ad minima veniam, quis nostrum exercitationem ullam corporis suscipit laboriosam, nisi ut aliquid ex ea commodi consequatur? Nam libero tempore, cum soluta nobis est eligendi optio cumque nihil impedit quo minus id quod maxime placeat facere possimus, omnis voluptas assumenda est, omnis dolor repellendus.


%%% Vložení souboru 'text/zaver' se závěrem
\chapter*{Závěr}
\phantomsection
\addcontentsline{toc}{chapter}{Závěr}

Cílem práce bylo navrhnout modul podporující komunikaci pomocí dvouvodičového \acs{JTAG} protokolu a rozšiřujícího protokolu pro přístup na systémovou sběrnici \acs{RISC-V} procesoru.

V úvodní kapitole je uveden princip fungování \acs{JTAG} rozhraní, systém pro testování \acs{RISC-V} procesorů, který je používaný v projektu pro testování daného procesorového jádra, a princip fungování systémové sběrnice tohoto procesoru. Princip fungování testovacího systému v tomto uspořádání a systémové sběrnice bylo třeba nastudovat, aby bylo možné navrhnout sekundární, časově optimalizovaný protokol pro přístup na systémovou sběrnici, který je popsaný v kapitole \ref{jtag_ap}.

Druhá kapitola popisuje vybraný princip komunikace dvouvodičovou variantou \acs{JTAG} protokolu dle standardu IEEE 1149.7 a způsob implementace modulu podporující tento způsob komunikace. Pro zvolení vhodné varianty dvouvodičové \acs{JTAG} komunikace bylo třeba nastudovat příslušné části standardu IEEE 1149.7, což bylo poměrně náročné, protože tento standard je velmi obsáhlý.

V navazující diplomové práci bude popsán také způsob implementace hardwarového řešení podporujícího navržený optimalizovaný protokol. Dále bude třeba uvést výsledky funkční simulace některých částí navržených modulů a také ověřit správnou funkci v obvodu \acs{FPGA}.

Pro otestování modulu pro podporu komunikace navrženým protokolem v obvodu \acs{FPGA}, bude zapotřebí rozšířit softwarové aplikační prostředí debuggeru o funkce, které budou odesílat a vyhodnocovat data přijatá přes \acs{JTAG} rozhraní dle protokolů navržených v podkapitole \ref{sec:protokoly}. Tato část práce bude z plánovaných činností nejnáročnější z důvodu menších zkušeností s programováním softwaru. Realizace by přesto neměla trvat déle než dva měsíce. Po dokončení realizace softwarových funkcí bude třeba vymyslet vhodný test pro otestování funkčnosti a porovnání časové efektivity nově navrženého protokolu s původním přístupem popsaným v podkapitole \ref{subsec:dm_sba}.

V rámci navrženého modulu, pro podporu nově vytvořeného protokolu, by bylo dobré implementovat také podporu používání obou variant protokolů s automatickou inkrementací adresy. Tato funkcionalita může být využita například pro zápis nebo čtení rozsáhlejších bloků dat z paměti. Tyto varianty protokolů jsou již navrženy a jejich doplnění do stávajícího návrhu by nemělo být příliš časově náročné.

%%% Vložení souboru 'text/literatura' se seznamem zdrojů
% Pro sazbu seznamu literatury použijte jednu z následujících možností

%%%%%%%%%%%%%%%%%%%%%%%%%%%%%%%%%%%%%%%%%%%%%%%%%%%%%%%%%%%%%%%%%%%%%%%%%
%1) Seznam citací definovaný přímo pomocí prostředí literatura / thebibliography

\begin{thebibliography}{99}

\bibitem{IEEE_1149-1}
IEEE Standard for Test Access Port and Boundary-Scan Architecture. Online. In: . S.~1-422. ISBN 978-0-7381-8263-6. Dostupné z: \url{https://doi.org/10.1109/IEEESTD.2013.6515989}. [cit. 2023-11-02].

\bibitem{IEEE_1149-7}
IEEE Standard for Reduced-Pin and Enhanced-Functionality Test Access Port and Boundary-Scan Architecture. Online. In: . Dostupné z: \url{https://doi.org/10.1109/IEEESTD.2022.9919140}. [cit. 2023-11-03].

\bibitem{JTAG_TAP_diagram}
\textit{Joint Test Action Group}. Online. In: Wikipedia: the free encyclopedia. San Francisco (CA): Wikimedia Foundation, 2001-. Dostupné z: \url{https://de.m.wikipedia.org/wiki/Joint\_Test\_Action\_Group}. [cit. 2023-10-31].

\bibitem{JTAG}
\textit{JTAG}. Online. In: Wikipedia: the free encyclopedia. San Francisco (CA): Wikimedia Foundation, 2001-. Dostupné z: \url{https://en.wikipedia.org/wiki/JTAG}. [cit. 2023-11-02].

\bibitem{risc-v_dbg}
\textit{RISC-V External Debug Support Version 0.13.2}. Online. San Mateo, California, U.S.: SiFive, 2019. Dostupné z: \url{https://riscv.org/wp-content/uploads/2019/03/riscv-debug-release.pdf}. [cit. 2023-08-01].

\bibitem{ri5cy}
\textit{RI5CY: User Manual}. Online. 2019. Dostupné z: \url{https://www.pulp-platform.org/docs/ri5cy\_user\_manual.pdf}. [cit. 2023-11-05].

\end{thebibliography}


%%%%%%%%%%%%%%%%%%%%%%%%%%%%%%%%%%%%%%%%%%%%%%%%%%%%%%%%%%%%%%%%%%%%%%%%%
%%2) Seznam citací pomocí BibTeXu
%% Při použití je nutné v TeXnicCenter ve výstupním profilu aktivovat spouštění BibTeXu po překladu.
%% Definice stylu seznamu
%\bibliographystyle{unsrturl}
%% Pro českou sazbu lze použít styl czechiso.bst ze stránek
%% http://www.fit.vutbr.cz/~martinek/latex/czechiso.tar.gz
%%\bibliographystyle{czechiso}
%% Vložení souboru se seznamem citací
%\bibliography{text/literatura}
%
%% Následující příkaz je pouze pro ukázku sazby literatury při použití BibTeXu.
%% Způsobí citaci všech zdrojů v souboru literatura.bib, i když nejsou citovány v textu.
%\nocite{*}

%%% Vložení souboru 'text/zkratky' se seznam použitých symbolů, veličin a zkratek
\cleardoublepage
\chapter*{\listofabbrevname}
\phantomsection
\addcontentsline{toc}{chapter}{\listofabbrevname}

\begin{acronym}[KolikMista]

	\acro{JTAG}
		{Joint Test Action Group}
	\acro{TCK}
		{Test Clock}
	\acro{TMS}
		{Test Mode Select}
	\acro{TDI}
		{Test Data In}
	\acro{TDO}
		{Test Data Out}
	\acro{TCKC}
		{Test Clock Compact}
	\acro{TMSC}
		{Test Mode Select Compact}
	\acro{TRST}
		{Test Reset Input}
	\acro{TAP}
		{Test Access Port}
	\acro{DTS}
		{Debug and Test System}
	\acro{TS}
		{Target System}
	\acro{DM}
		{Debug Module}
	\acro{DMI}
		{Debug Module Interface}
	\acro{OAC}
		{Online Activation Code}	
	\acro{EC}
		{Extension Code}
	\acro{CP}
		{Check Packet}
	\acro{VHDL}
		{VHSIC Hardware Description Language}
	\acro{FPGA}
		{Programovatelné hradlové pole}
	\acro{DPS}
		{Deska plošných spojů}
	\acro{LSB}
		{Nejméně významný bit}
	\acro{MSB}
		{Nejvíce významný bit}
	\acro{RISC-V}
		{Reduced Instruction Set Computer - V}
	\acro{RISC}
		{Reduced Instruction Set Computer}
	\acro{PULP}
		{Parallel Ultra Low Power}
	\acro{MPSSE}
		{Multi-Purpose Synchronous Serial Engine}
\end{acronym}


%%% Začátek příloh
\appendix

%%% Vysázení seznamu příloh
% (vynechejte, pokud máte dvě nebo méně příloh)
\listofappendices

%%% Vložení souboru 'text/prilohy' s přílohami
% Obvykle je přítomen alespoň popis co najdeme na přiloženém médiu
%\chapter{Některé příkazy balíčku \texttt{thesis}}

\section{Příkazy pro sazbu veličin a jednotek}

\begin{table}[!h]
  \caption[Přehled příkazů]{Přehled příkazů pro matematické prostředí }
  \begin{center}
  	\small
	  \begin{tabular}{|c|c|c|c|}
	    \hline
	    Příkaz    						& Příklad 					& Zdroj příkladu  							& Význam  \\
	    \hline\hline
	    \verb|\textind{...}|	& $\beta_\textind{max}$ 	& \verb|$\beta_\textind{max}$|	& textový index \\
	    \hline
	    \verb|\const{...}| 		& $\const{U}_\textind{in}$ 				& \verb|$\const{U}_\textind{in}$|		& konstantní veličina \\
	    \hline
	    \verb|\var{...}| 		& $\var{u}_\textind{in}$ & \verb|$\var{u}_\textind{in}$| & proměnná veličina \\
	    \hline
	    \verb|\complex{...}| 	& $\complex{u}_\textind{in}$ & \verb|$\complex{u}_\textind{in}$| & komplexní veličina \\
	    \hline
	    \verb|\vect{...}| 		& $\vect{y}$ 						& \verb|$\vect{y}$| & vektor \\
	    \hline
	    \verb|\mat{...}| 	& $\mat{Z}$ 						& \verb|$\mat{Z}$| & matice \\
	    \hline
	    \verb|\unit{...}| 		& $\unit{kV}$ 						& \verb|$\unit{kV}$|\quad či\ \, \verb|\unit{kV}| & jednotka \\
	    \hline
	  \end{tabular}
  \end{center}
\end{table}



%\newpage
\section{Příkazy pro sazbu symbolů}

\begin{itemize}
  \item
    \verb|\E|, \verb|\eul| -- sazba Eulerova čísla: $\eul$,
  \item
    \verb|\J|, \verb|\jmag|, \verb|\I|, \verb|\imag| -- sazba imaginární jednotky: $\jmag$, $\imag$,
  \item
    \verb|\dif| -- sazba diferenciálu: $\dif$,
  \item
    \verb|\sinc| -- sazba funkce: $\sinc$,
  \item
    \verb|\mikro| -- sazba symbolu mikro stojatým písmem%
			\footnote{znak pochází z~balíčku \texttt{textcomp}}: $\mikro$,
	\item
		\verb|\uppi| -- sazba symbolu $\uppi$
			(stojaté řecké pí, na rozdíl od \verb|\pi|, což sází $\pi$).
\end{itemize}
%
Všechny symboly jsou určeny pro matematický mód, vyjma \verb|\mikro|, jenž je\\ použitelný rovněž v~textovém módu.
%$\upmikro$


\chapter{Druhá příloha}

\begin{figure}[!h]
  \begin{center}
    \includegraphics[scale=0.5]{obrazky/ZlepseneWilsonovoZrcadloNPN}
  \end{center}
  \caption[Alenčino zrcadlo]{Zlepšené Wilsonovo proudové zrcadlo.}
\end{figure}

Pro sazbu vektorových obrázků přímo v~\LaTeX{}u je možné doporučit balíček \href{https://www.ctan.org/pkg/pgf}{\texttt{TikZ}}.
Příklady sazby je možné najít na \href{http://www.texample.net/tikz/examples/}{\TeX{}ample}.
Pro vyzkoušení je možné použít programy QTikz nebo TikzEdt.




\chapter{Příklad sazby zdrojových kódů}

\section{Balíček \texttt{listings}}

Pro vysázení zdrojových souborů je možné použít balíček \href{https://www.ctan.org/pkg/listings}{\texttt{listings}}.
Balíček zavádí nové prostředí \texttt{lstlisting} pro sazbu zdrojových kódů, jako například:
%
\begin{lstlisting}[language={[LaTeX]TeX}]
\section{Balíček lstlistings}
Pro vysázení zdrojových souborů je možné použít
	balíček \href{https://www.ctan.org/pkg/listings}%
	{\texttt{listings}}.
Balíček zavádí nové prostředí \texttt{lstlisting} pro
	sazbu zdrojových kódů.
\end{lstlisting}
%
Podporuje množství programovacích jazyků.
Kód k~vysázení může být načítán přímo ze zdrojových souborů.
Umožňuje vkládat čísla řádků nebo vypisovat jen vybrané úseky kódu.
Např.:

\noindent
Zkratky jsou sázeny v~prostředí \texttt{acronym}:
\label{lst:zkratky}
\lstinputlisting[language={[LaTeX]TeX},nolol,numbers=left, firstnumber=6, firstline=6,lastline=6]{text/zkratky.tex}
%
Šířka textu volitelného parametru \verb|KolikMista| udává šířku prvního sloupce se zkratkami.
Proto by měla být zadávána nejdelší zkratka nebo symbol.
Příklad definice zkratky \acs{symfvz} je na výpisu \ref{lst:symfvz}.

\shorthandoff{-}
\lstinputlisting[language={[LaTeX]TeX},frame=single,caption={Ukázka sazby zkratek},label=lst:symfvz,numbers=left,linerange={bsymfvz-\%\%\%\ esymfvz},includerangemarker=false]{text/zkratky.tex}
\shorthandon{-}

\noindent
Ukončení seznamu je provedeno ukončením prostředí:
\lstinputlisting[language={[LaTeX]TeX},nolol,numbers=left,firstnumber=26,linerange=26]{text/zkratky.tex}

\vspace{\fill}

\noindent
{\bf Poznámka k~výpisům s~použitím volby jazyka \verb|czech| nebo \verb|slovak|:}\newline
Pokud Váš zdrojový kód obsahuje znak spojovníku \verb|-|, pak překlad může skončit chybou.
Ta je způsobená tím, že znak \verb|-| je v~českém nebo slovenském nastavení balíčku \verb|babel| tzv.\ aktivním znakem.
Přepněte znak \verb|-| na neaktivní příkazem \verb|\shorthandoff{-}| těsně před výpisem a hned za ním jej vraťte na aktivní příkazem \verb|\shorthandon{-}|.
Podobně jako to je ukázáno ve zdrojovém kódu šablony.


\clearpage

%\section{Výpis kódu prostředí Matlab}
Na výpisu \ref{lst:priklad.vypis.kodu.Matlab} naleznete příklad kódu pro Matlab, na výpisu \ref{lst:priklad.vypis.kodu.C} zase pro jazyk~C.

\lstnewenvironment{matlab}[1][]{%
\iflanguage{czech}{\shorthandoff{-}}{}%
\iflanguage{slovak}{\shorthandoff{-}}{}%
\lstset{language=Matlab,numbers=left,#1}%
}{%
\iflanguage{slovak}{\shorthandon{-}}{}%
\iflanguage{czech}{\shorthandon{-}}{}%
}

\begin{matlab}[frame=single,float=htbp,caption={Příklad Schur-Cohnova testu stability v~prostředí Matlab.},label=lst:priklad.vypis.kodu.Matlab,numberstyle=\scriptsize, numbersep=7pt]
%% Priklad testovani stability filtru

% koeficienty polynomu ve jmenovateli
a = [ 5, 11.2, 5.44, -0.384, -2.3552, -1.2288];
disp( 'Polynom:'); disp(poly2str( a, 'z'))

disp('Kontrola pomoci korenu polynomu:');
zx = roots( a);
if( all( abs( zx) < 1))
    disp('System je stabilni')
else
    disp('System je nestabilni nebo na mezi stability');
end

disp(' '); disp('Kontrola pomoci Schur-Cohn:');
ma = zeros( length(a)-1,length(a));
ma(1,:) = a/a(1);
for( k = 1:length(a)-2)
    aa = ma(k,1:end-k+1);
    bb = fliplr( aa);
    ma(k+1,1:end-k+1) = (aa-aa(end)*bb)/(1-aa(end)^2);
end

if( all( abs( diag( ma.'))))
    disp('System je stabilni')
else
    disp('System je nestabilni nebo na mezi stability');
end
\end{matlab}

\noindent
\begin{minipage}{\linewidth}


%\section{Výpis kódu jazyka C}

\begin{lstlisting}[frame=single,numbers=right,caption={Příklad implementace první kanonické formy v~jazyce C.},label=lst:priklad.vypis.kodu.C,basicstyle=\ttfamily\small, keywordstyle=\color{black}\bfseries\underbar,]
// první kanonická forma
short fxdf2t( short coef[][5], short sample)
{
	static int v1[SECTIONS] = {0,0},v2[SECTIONS] = {0,0};
	int x, y, accu;
	short k;

	x = sample;
	for( k = 0; k < SECTIONS; k++){
		accu = v1[k] >> 1;
		y = _sadd( accu, _smpy( coef[k][0], x));
		y = _sshl(y, 1) >> 16;

		accu = v2[k] >> 1;
		accu = _sadd( accu, _smpy( coef[k][1], x));
		accu = _sadd( accu, _smpy( coef[k][2], y));
		v1[k] = _sshl( accu, 1);

		accu = _smpy( coef[k][3], x);
		accu = _sadd( accu, _smpy( coef[k][4], y));
		v2[k] = _sshl( accu, 1);

		x = y;
	}
	return( y);
}
\end{lstlisting}
\end{minipage}







\chapter{Obsah elektronické přílohy}
Elektronická příloha je často nedílnou součástí semestrální nebo závěrečné práce.
Vkládá se do informačního systému VUT v~Brně ve vhodném formátu (ZIP, PDF\,\dots).

Nezapomeňte uvést, co čtenář v~této příloze najde.
Je vhodné okomentovat obsah každého adresáře, specifikovat, který soubor obsahuje důležitá nastavení, který soubor je určen ke spuštění, uvést nastavení kompilátoru atd.
Také je dobře napsat, v~jaké verzi software byl kód testován (např.\ Matlab 2018b).
Pokud bylo cílem práce vytvořit hardwarové zařízení,
musí elektronická příloha obsahovat veškeré podklady pro výrobu (např.\ soubory s~návrhem DPS v~Eagle).

Pokud je souborů hodně a jsou organizovány ve více složkách, je možné pro výpis adresářové struktury použít balíček \href{https://www.ctan.org/pkg/dirtree}{\texttt{dirtree}}.

\bigskip

{\small
%
\dirtree{%.
.1 /\DTcomment{kořenový adresář přiloženého archivu}.
.2 logo\DTcomment{loga školy a fakulty}.
.3 BUT\_abbreviation\_color\_PANTONE\_EN.pdf.
.3 BUT\_color\_PANTONE\_EN.pdf.
.3 FEEC\_abbreviation\_color\_PANTONE\_EN.pdf.
.3 FEKT\_zkratka\_barevne\_PANTONE\_CZ.pdf.
.3 UTKO\_color\_PANTONE\_CZ.pdf.
.3 UTKO\_color\_PANTONE\_EN.pdf.
.3 VUT\_barevne\_PANTONE\_CZ.pdf.
.3 VUT\_symbol\_barevne\_PANTONE\_CZ.pdf.
.3 VUT\_zkratka\_barevne\_PANTONE\_CZ.pdf.
.2 obrazky\DTcomment{ostatní obrázky}.
.3 soucastky.png.
.3 spoje.png.
.3 ZlepseneWilsonovoZrcadloNPN.png.
.3 ZlepseneWilsonovoZrcadloPNP.png.
.2 pdf\DTcomment{pdf stránky generované informačním systémem}.
.3 student-desky.pdf.
.3 student-titulka.pdf.
.3 student-zadani.pdf.
.2 text\DTcomment{zdrojové textové soubory}.
.3 literatura.tex.
.3 prilohy.tex.
.3 reseni.tex.
.3 uvod.tex.
.3 vysledky.tex.
.3 zaver.tex.
.3 zkratky.tex.
%.2 navod-sablona\_FEKT.pdf\DTcomment{návod na používání šablony}.
.2 sablona-obhaj.tex\DTcomment{hlavní soubor pro sazbu prezentace k~obhajobě}.
%.2 readme.txt\DTcomment{soubor s~popisem obsahu CD}.
.2 sablona-prace.tex\DTcomment{hlavní soubor pro sazbu kvalifikační práce}.
.2 thesis.sty\DTcomment{balíček pro sazbu kvalifikačních prací}.
}
}


\end{document}